% Empfehlung

\documentclass[11pt]{article}

\usepackage[german]{babel}

\usepackage[autostyle=true]{csquotes}

\usepackage[a4paper, margin=1in]{geometry}

\usepackage{libertine}

\setlength{\parindent}{0pt}

\addtolength{\parskip}{\baselineskip}

\newcommand{\extrapar}{\par\vspace{\baselineskip}}

\newcommand{\heading}[1]{\begin{center}\Huge \textbf{#1} \end{center}}

\newcommand{\sub}[1]{{\Large \textbf{#1}}\par}

\newcommand{\subsub}[1]{{\large \textbf{#1}}\par}

\newcommand{\zitat}[1]{\emph{\foreignquote{german}{#1}}}

\newcommand{\titleitem}[1]{\item \textbf{#1} \par}

\begin{document}
\thispagestyle{plain}

\heading{Die Empfehlung}

\sub{Definition}

Bei einer Empfehlung handelt es sich um eine Befundung einer Sachfrage mit anschlie\ss{}enden Schlussfolgerungen und argumentativer Bewertung, die als Best\"{a}rkung bzw. als Entscheidungshilfe (bei einer Auswahl) dient. In der Regel ist sie eine Mischform aus Schreibkompetenzen, bei der sowohl analytisch-interpretative, appellierend-kritisierende als auch argumentativ-bewertende Kompetenz gefragt ist. Die pers\"{o}nliche Meinung ist in der Empfehlung wichtig und sollte zu einer klaren, subjektiven jedoch objektiv und logisch begr\"{u}ndeten Bewertung und Empfehlung f\"{u}hren.

\sub{Aufbau}

\begin{itemize}

	\titleitem{\"{U}berschrift}

	Eine Empfehlung braucht im Grunde genommen keine \"{U}berschrift, jedoch sicherlich keine kreative oder reizende. Es reicht: \zitat{Empfehlung zum Thema / zur Auswahl von ...}

	\titleitem{Gru\ss{}formel}

	Eine Empfehlung ist eine Form des Briefes, braucht also eine direkte Anrede an das Publikum, z.B.: \zitat{Sehr geehrter Schulgemeinschaftsausschuss!}

	\titleitem{Einleitung}

	In der Einleitung sollten die Empfehlsituation und die zur Auswahl stehenden Texte oder Entscheidungs- bzw. Handlungsm\"{o}glichkeiten genannt werden. Ebenso kann schon ein Ausblick auf die pers\"{o}nliche Preferenz bzw. Bewertung der Auswahloptionen gegeben werden, um somit schlie\ss{}lich zum Hauptteil \"{u}berzuleiten. Beispiel:

	\zitat{Sie haben mich gebeten, Ihnen einen lyrischen Text passend zu Werner Wintersteiners Kommentar \zitat{Emp\"{o}rt euch!} zu empfehlen. Nach intensiver Recherche und Analyse einer Vielzahl von Gedichten kristallisierten sich drei passende Werke heraus, n\"{a}mlich \zitat{Niemand sucht aus} von Gioconda Belli, Peter Turrinis \zitat{Das Nein} sowie, letztlich, \zitat{Die Abnehmer} von Erich Fried. Von diesen \"{u}bertrifft meiner Meinung nach das Letztere die beiden anderen in seinem lyrischen Stil, seinem Inhalt sowie seiner Wirkung bei Weitem. Die Gr\"{u}nde daf\"{u}r m\"{o}chte ich Ihnen erl\"{a}utern.}

	\titleitem{Hauptteil}

	Der Hauptteil variiert je nach Operatoren bzw. Aufgabenstellung, sollte aber grunds\"{a}tzlich eine Referenz zu einem Bezugstext herstellen und diesen zusammenfassen und / oder die zur Auswahl stehenden Optionen beleuchten, vergleichen und argumentieren bzw. klar und logisch begr\"{u}nden, wieso die pers\"{o}nlich bevorzugte Auswahl die beste ist. Eventuell muss dazu ein lyrischer Text nach Stil, Sprache und Aufbau analysiert werden. Es ist ebenso m\"{o}glich, im Hauptteil noch keinen Bezug auf die nicht gew\"{a}hlten Optionen zu nehmen, sondern erst im Schluss ein paar Worte dar\"{u}ber zu verlieren, wieso sie nicht so gut wie die bevorzugte Option sind.

\pagebreak

	\titleitem{Schluss}

	Im Schluss sollte das Urteil kurz zusammengefasst werden um eine klare, begr\"{u}ndete und aussagekr\"{a}ftige pers\"{o}nliche Empfehlung zu liefern. Beispiel:

	\zitat{Nat\"{u}rlich sind auch die beiden anderen Werke in meiner engeren Auswahl erw\"{a}hnenswert, doch schafft es weder Turrinis \zitat{Das Nein} noch Bellis \zitat{Niemand sucht aus} Wirkung, Stil und Inhalt auf solch elegante Weise zu vereinen, wie \zitat{Die Abnehmer} von Erich Fried. Auch empfinde ich \zitat{Die Abnehmer} als das kritischste dieser drei Werke, weswegen ich es Ihnen mit gr\"{o}\ss{}ter Freude empfehle. }

	\titleitem{Abschiedsformel}

	So wie eine Empfehlung eine Gru\ss{}formel hat, muss sie auch eine Abschiedsformel aufweisen:

	\zitat{Mit freundlichen Gr\"{u}\ss{}en\extrapar Peter Goldsborough}

\end{itemize}

\sub{Stil}

\begin{itemize}
	\item Pers\"{o}nliche, subjektive Empfehlung --- \zitat{Ich} erlaubt (in Ma\ss{}en)
	\item Neutrale, objektive Analyse bzw. Argumentation und Begr\"{u}ndung
	\item Addressatenbezogen, an Leser angepasst
	\item Fachjargon bei Analyse von Texten
\end{itemize}

\end{document}