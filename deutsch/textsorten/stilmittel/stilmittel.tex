% Stilmittel

\documentclass[11pt]{article}

\usepackage[german]{babel}

\usepackage[autostyle=true]{csquotes}

\usepackage[a4paper, margin=1in]{geometry}

\usepackage{libertine}

\setlength{\parindent}{0pt}

\addtolength{\parskip}{\baselineskip}

\newcommand{\extrapar}{\par\vspace{\baselineskip}}

\newcommand{\heading}[1]{\begin{center}\Huge \textbf{#1} \end{center}}

\newcommand{\sub}[1]{{\Large \textbf{#1}}\par}

\newcommand{\subsub}[1]{{\large \textbf{#1}}\par}

\newcommand{\zitat}[1]{\emph{\foreignquote{german}{#1}}}

\newcommand{\titleitem}[1]{\item \textbf{#1} \par}

\begin{document}

\heading{Stilmittel}

\extrapar
\extrapar

{ \renewcommand{\arraystretch}{2}

\begin{longtable}{|>{\columncolor[gray]{0.8}}p{3.5cm}|p{6cm}|p{5cm}|}

	\hline

	\textbf{Allegorie} & Konkrete Darstellung abstrakter Begriffe & \emph{Gott Amor = Liebe; Justitia = Gerechtigkeit}

	\\ \hline

	\textbf{Metapher} & Verbindung zweier Bedeutungsbereiche / Verbildlichung & \emph{Redefluss; Warteschlange; jmd. das Herz brechen; eine Mauer des Schweigens}

	\\ \hline

	\textbf{Personifikation} & Zuschreibung menschlicher Eigenschaften f\"{u}r Dinge / abstrakte Begriffe & \emph{Die Gr\"{a}ser tanzen; die Sonne lacht}

	\\ \hline

	\textbf{Vergleich / Analogie} & Vergleich zwischen zwei Dingen / Hervorhebung von Gemeinsamkeiten (\zitat{\ldots \textbf{wie} \ldots}) & \emph{Er k\"{a}mpft \textbf{wie} ein L\"{o}we; stark \textbf{wie} ein B\"{a}r}

	\\ \hline

	\textbf{Euphemismus} & Besch\"{o}nigung / sanftere Ausdr\"{u}cksweise f\"{u}r etwas dramatisches & \emph{\zitat{entschlafen} anstatt \zitat{sterben}}

	\\ \hline

	\textbf{Hyperbel} & \"{U}bertreibung (mehr / gr\"{o}\ss{}er / dramatischer scheinen lassen) & \emph{Zu 120\%; Schneckentempo; so schnell wie der Wind; unendlich lang}

	\\ \hline

	\textbf{Litotes} & Doppelte Verneinung / Untertreibung & \emph{nicht schlecht = gut; er hat nicht Unrecht = er hat Recht;}

	\\ \hline

	\textbf{Neologismus} & Wortneusch\"{o}pfung / Neue Verbindung von Begriffsgruppen & \emph{Literaturpapst; Augenkrebs; Ohrgasmus}

	\\ \hline

	\textbf{Pleonasmus} & Wiederholung eines charakteristischen Merkmals / doppelte Darstellung einer Eigenschaft & \emph{runde Kugel; alter Greis; nasser Regen; fl\"{u}ssiges Getr\"{a}nk}

	\\ \hline

	\textbf{Trikolon} & Dreigliedriger Ausdruck & \emph{Veni, vidi, vici; Verliebt, verlobt, verheiratet}

	\\ \hline

	\textbf{Ellipse} & Auslassung selbstverst\"{a}ndlicher, unwichtiger W\"{o}rter $\rightarrow$ grammatikalisch unvollst\"{a}ndiger Satz & \emph{Todesstille f\"{u}rchterlich; Im Zweifel f\"{u}r den Angeklagten}

	\\ \hline

	\textbf{Chiasmus} & \"{U}berkreuzung von Sinneinheiten & \emph{Er ist arm, reich ist sie.}

	\\ \hline

	\textbf{Parallelismus} & Wiederholung gleicher Satzbaumuster & \emph{Das Wasser rauscht, das Wasser schwoll ...}

	\\ \hline

	\textbf{Zeugma} & Zuordnung eines Verbes zu zwei Satzf\"{u}gungen & \emph{Er warf die Nudeln aus dem Topf und einen Blick aus dem Fenster}

	\\ \hline

	\textbf{Antithese} & Betonter Gegensatz & \emph{Des einen Freud, des anderen Leid}

	\\ \hline

	\textbf{Oxymoron} & Verbindung von zwei wiederspr\"{u}chlichen Begriffen & \emph{junger Greis, vielsagendes Schweigen}

	\\ \hline

	\textbf{Rhetorische Frage} & Frage, auf die keine Antwort erwartet wird & \emph{Und das sollen wir zulassen? Bist du verr\"{u}ckt?}

	\\ \hline

	\textbf{Alliteration} & Stabreim / Anreihung von Begriffen mit demselbem Anfangslaut & \emph{Mit Kind und Kegel; Manner mag man eben;}

	\\ \hline

	\textbf{Anapher} & Wiederholung eines Wortes oder einer Wortgruppe am Vers- oder Satzanfang & \emph{Du bist schuld, du hast das getan, du wirst b"{u}\ss{}en!}

	\\ \hline

	\textbf{Asyndeton} & Anreihung von W\"{o}rtern / S\"{a}tzen ohne Bindew\"{o}rter & \emph{Ich kam, sah, siegte; Freiheit, Gleiheit, Br\"{u}derlichkeit; }

	\\ \hline

	\textbf{Polysyndeton} & Anreihung von W\"{o}rtern / S\"{a}tzen mit vielen Bindew\"{o}rtern & \emph{Ich kam und sah und siegte; Er sang und tanzte und spielte und lachte}

	\\ \hline

	\textbf{Onomatopoesie} & Lautmalerei & \emph{Kuckuck; quaken; quietschen}

	\\ \hline

	\textbf{Ironie} & Das Gegenteil des Gesagten ist gemeint & \emph{Na toll!; Eine sch\"{o}ne Bescherung!}

	\\ \hline

	\textbf{Ausruf / Exclamatio} & Ein Satz (oftmals eine Ellipse) der mit einem Ausrufezeichen endet & \emph{Immer gib ihm!}

	\\ \hline

	\textbf{Paradoxon} & Widerspr\"{u}chliche Aussage & \emph{Weniger ist mehr}

	\\ \hline

	\textbf{Paronomasie} & Gleichlautende oder \"{a}hnliche W\"{o}rter werden miteinander verbunden & \emph{Lieber arm dran als Arm ab}

	\\ \hline

	\textbf{Figura etymologica / Polyptoton} & Verbindung zweier W\"{o}rter aus verwandten Wortfamilien aber verschiedenen Wortarten (Verb, Nomen) bzw. mit verschiedener Bedeutung oder in anderen F\"{a}llen & \emph{jmd. eine Grube graben; Spiele spiel ich mit dir; ... noch nicht bezahlt, aber nicht unbezahlbar; meines Herzens Herz}

	\\ \hline

	\textbf{Paronomastischer Intensit\"{a}tsgenitiv} & Steigerung durch Verbindung eines Wortes mit seinem Genetiv & \emph{der K\"{o}nig der K\"{o}nige; das Spiel der Spiele}

	\\ \hline

	\textbf{Epipher} & Anreihung von S\"{a}tzen / Satzteilen mit demselben Wort(-gruppe) am Ende --- Gegenteil der Anapher & \emph{Er lachte, als er das sagte. Er spuckte, als er das sagte.}

	\\ \hline

	\textbf{Homoioteleuton} & Gegenteil der Alliteration / Endreim / Anreihung von W\"{o}rtern mit demselben Endlaut & \emph{Gleich\textbf{heit}, Frei\textbf{heit}, Br\"{u}derlich\textbf{keit}}

	\\ \hline

\end{longtable}

} % end

\end{document}