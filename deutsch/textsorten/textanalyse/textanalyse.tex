% Textanalyse

\documentclass[11pt]{article}

\usepackage[a4paper, margin=1in]{geometry}

\usepackage{amsmath}

\usepackage{amssymb}

\usepackage[german]{babel}

\usepackage[autostyle=true]{csquotes}

\usepackage{libertine}

\usepackage[libertine]{newtxmath}

\usepackage{tikz}

\usepackage{gensymb}

\usepackage{fancyhdr}

\usepackage{amsfonts}

\usepackage{pgfplots}

\pgfplotsset{compat=1.10}

\usepackage{multicol}

\usepackage{caption}

\usepackage{floatrow}

\everymath{\displaystyle}

% Header / footer settings

\pagestyle{fancy}
\fancyhf{}
\renewcommand{\headrulewidth}{0.2mm}
\fancyhead[C]{Funktionen}
\renewcommand{\footrulewidth}{0.2mm}
\fancyfoot[L]{Peter Goldsborough}
\fancyfoot[C]{\thepage}
\fancyfoot[R]{\today}

\fancypagestyle{plain}{%
	\fancyhf{}
	\renewcommand{\headrulewidth}{0mm}%
	\renewcommand{\footrulewidth}{0.2mm}%
	\fancyfoot[L]{Peter Goldsborough}
	\fancyfoot[C]{\thepage}
	\fancyfoot[R]{\today}
}


\setlength{\headheight}{15pt}

\setlength{\parindent}{0pt}

\addtolength{\parskip}{\baselineskip}


\newcommand{\overbar}[1]{\mkern 1.5mu\overline{\mkern-1.5mu#1\mkern-1.5mu}\mkern 1.5mu}

\newcommand{\heading}[1]{\begin{center}\Huge \textbf{#1}\end{center}\par}

\newcommand{\sub}[1]{\vspace{\parskip}{\LARGE\textbf{#1}}}

\newcommand{\subsub}[1]{{\Large \textbf{#1}}}

\newcommand{\subsubsub}[1]{\textbf{#1}}

\newcommand{\colvec}[1]{\begin{pmatrix}#1\end{pmatrix}}

\newcommand{\extrapar}{\par\vspace{\baselineskip}}

\newcommand{\zitat}[1]{\foreignquote{german}{#1}}

\newcommand{\bolditem}[1]{\item \textbf{#1}}

\newcommand{\titleitem}[1]{\bolditem{#1}\par}

\newcommand{\defas}{ \dots \,\,}

\begin{document}
\thispagestyle{plain}

\heading{Die Textanalyse}

\sub{Definition}

Das Ziel einer \textbf{Textanalyse} ist es, einen sachlichen, epischen oder lyrischen Text nach seinem Aufbau, seiner Sprache, seinem Stil, seiner Verwendung von rhetorischen Stilmitteln sowie seinem Inhalt nach zu analysieren, zu beschreiben und zu bewerten. Die Textanalyse ist eine \emph{informierende, beschreibende, analysierende, zusammenfassende, bewertende sowie beurteilende} Textsorte, jedoch keine argumentierende oder meinungs\"{a}\ss{}ernde. 

\sub{Aufbau}

\begin{enumerate}

	\titleitem{\"{U}berschrift}

	Eine Textanalyse erfordert keine besondere \"{U}berschrift. Es gen\"{u}gt: \zitat{Textanalyse zu \zitat{\ldots}}.

	\titleitem{Einleitung}

	In der Einleitung ist es essentiell, eine Referenz zum analysierten Text herzustellen. Es sollten nach M\"{o}glichkeit Titel, Autor, Textsorte, Erscheinungsdatum sowie Erscheinungsort/-medium des Textes genannt werden. Ebenso ist es ratsam, den Inhalt des Textes in einem kurzen, b\"{u}ndigen Satz zusammenzufassen und schon einige bewertende Worte zum Text zu \"{a}\ss{}ern, um dem Leser der Textanalyse einen Ausblick auf die Analyse zu geben. Es ist jedoch auch m\"{o}glich, in der Einleitung auf einen aktuellen, passenden Anlass, ein historisches Geschehen oder ein Zitat Bezug zu nehmen. Letztlich ist es in der Einleitung noch wichtig, eine gute \"{U}berleitung zum Hauptteil herzustellen.

	\titleitem{Hauptteil}

		Im Hauptteil der Textanalyse sollen der Inhalt, der Aufbau, der Stil, die Sprache sowie sonstige analytische Informationen betreffend des Textes besprochen werden. Auch m\"{u}ssen in diesem Teil die weiteren, je nach Textaufgabe verschiedenen, Operatoren bearbeitet werden. Der Hauptteil besteht daher \"{u}blicherweise aus den folgenden Teilen:

		\begin{enumerate}

			\titleitem{Zusammenfassung}

			Hierbei wird der Inhalt des Textes kurz zusammengefasst und dessen wichtigste Aussagen besprochen und gedeutet.

			\titleitem{Analyse}

			Nach der Zusammenfassung des Textes muss die eigentliche \zitat{Analyse} durchgef\"{u}hrt werden. Diese sollte eine Nennung der verwendeten rhetorischen Stilmittel inklusive Textzitat und Deutung der Verwendung im Bezug auf Inhalt sowie der Intention des Autors beinhalten. Weiters sollten das Publikum bzw. der Addressat des Textes (Leser, Gott, bestimmte Gruppe) sowie das Niveau der Sprache (komplex, allt\"{a}glich, umgangssprachlich) abgehandelt werden. 

			\extrapar

			Beispiel:

			\extrapar

			\zitat{Tucholsky arbeitet in seinem Text sehr stark mit den Mitteln des Sarkasmus, der Ironie sowie des Zynismus, wodurch dem gesamten Essay ein ausgepr\"{a}gt sp\"{o}ttischer Unterton verliehen wird. Dies erkennt man beispielsweise an den Zeilen 14 und 15: \zitat{Sei \"{u}berhaupt unliebensw\"{u}rdig --- daran erkennt man den Mann.}. Hiermit will der Autor dem Leser durch Kontrast und Provokation vermitteln, wie er sich im Urlaub m\"{o}glichst nicht zu verhalten hat. Es lassen sich auch weitere rhetorische bzw. sprachliche Stilmittel im Text finden, von Repetitionen --- \zitat{[\ldots] durch die fremde Stadt. In der fremden Stadt [\ldots]}, Correctionen --- \zitat{Hast du keinen Titel … Verzeihung … ich meine [\ldots]}, Hyperbeln --- \zitat{Bedenke, dass es von ungeheurer Wichtigkeit ist, dass du einen Fensterplatz hast} --- um Klischees und schlechtes Benehmen hervorzuheben, bis hin zu Exclamationen --- \zitat{Immer gib ihm!} oder \zitat{\"{A}rgere dich!}. Letztlich ist noch anzumerken, dass die Sprache des Autors im Essay grunds\"{a}tzlich als \zitat{allt\"{a}glich} einzustufen ist und dass sich Tucholsky direkt an den Leser wendet, welcher f\"{u}r ihn, dem Inhalt nach, m\"{a}nnlich ist.}

			\titleitem{Bearbeitung weiterer Operatoren}

			Danach folgt die Bearbeitung der weiteren gestellten Operatoren, welche eine Verbindung des Textes mit einem aktuellen Anlass, die Verkn\"{u}pfung mit einer Grafik bzw. einem Diagramm oder sonstiges erfordern k\"{o}nnen.

		\end{enumerate}

	\titleitem{Schluss}

	Der Schluss einer Textanalyse sollte die gefundenen und besprochenen Ergebnisse der Analyse nochmals kurz f\"{u}r den Leser zusammenfassen. Danach ist es m\"{o}glich, eine pers\"{o}nliche Meinung oder Bewertung des Textes beizuf\"{u}gen, welche durchaus die Verwendung von \zitat{Ich} zul\"{a}sst. Ebenso ist es m\"{o}glich, einen Zukunftausblick auf Basis des Textes bzw. der Analyse zu geben oder einen Appell bzw. einen Ausruf zu \"{a}\ss{}ern.

\end{enumerate}

\sub{Stil der Textanalyse}

Der Stil und die Sprache einer Textanalyse sollten sachlich, objektiv, neutral und unpers\"{o}nlich sein. Die Verwendung von \zitat{Ich} sollte auf jeden Fall vermieden werden, ist aber im Schluss, zum Ausdr\"{u}cken des pers\"{o}nlichen Empfindendns bez\"{u}glich des Textes, erlaubt. 

\sub{Beschreibung von Sprache und Aufbau}

Die Sprache eines Textes kann sein:

\begin{itemize}
	\item (stilistisch) gehoben / intellektuell / komplex
	\item (stilistisch) neutral / allt\"{a}glich
	\item umgangsprachlich / vulg\"{a}r
\end{itemize}

Der Aufbau eines Textes kann sein:

\begin{itemize}
	\item parataktisch (viele, b\"{u}ndige, kurze S\"{a}tze)
	\item hypotaktisch (wenige, komplexe, lange S\"{}atze)
\end{itemize}

\pagebreak

\sub{Wichtigste Stilmittel}

\begin{itemize}
	\titleitem{Analogie} Vergleich zwischen zwei Dingen / Hervorhebung von Gemeinsamkeiten (\emph{Er k\"{a}mpft \textbf{wie} ein L\"{o}we; stark \textbf{wie} ein B\"{a}r})

	\titleitem{Metapher} Verbindung zweier Bedeutungsbereiche / Verbildlichung (\emph{Redefluss; Warteschlange; jmd. das Herz brechen; eine Mauer des Schweigens})

	\titleitem{Euphemismus} Besch\"{o}nigung / sanftere Ausdr\"{u}cksweise f\"{u}r etwas dramatisches (\emph{\zitat{entschlafen} anstatt \zitat{sterben}})

	\titleitem{Hyperbel} \"{U}bertreibung (mehr / gr\"{o}\ss{}er / dramatischer scheinen lassen) (\emph{Zu 120\%; Schneckentempo; so schnell wie der Wind; unendlich lang})

	\titleitem{Ellipse} Auslassung selbstverst\"{a}ndlicher, unwichtiger W\"{o}rter $\rightarrow$ grammatikalisch unvollst\"{a}ndiger Satz (\emph{Todesstille f\"{u}rchterlich; Im Zweifel f\"{u}r den Angeklagten})

	\titleitem{Alliteration} Stabreim / Anreihung von Begriffen mit demselbem Anfangslaut (\emph{Mit Kind und Kegel; Manner mag man eben;})

	\titleitem{Anapher} Wiederholung eines Wortes oder einer Wortgruppe am Vers- oder Satzanfang (\emph{Du bist schuld, du hast das getan, du wirst b"{u}\ss{}en!}) 
\end{itemize}


\end{document}