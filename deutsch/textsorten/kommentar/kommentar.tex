% Kommentar

\documentclass[11pt]{article}

\usepackage[german]{babel}

\usepackage[autostyle=true]{csquotes}

\usepackage[a4paper, margin=1in]{geometry}

\usepackage{libertine}

\setlength{\parindent}{0pt}

\addtolength{\parskip}{\baselineskip}

\newcommand{\extrapar}{\par\vspace{\baselineskip}}

\newcommand{\heading}[1]{\begin{center}\Huge \textbf{#1} \end{center}}

\newcommand{\sub}[1]{{\Large \textbf{#1}}\par}

\newcommand{\subsub}[1]{{\large \textbf{#1}}\par}

\newcommand{\zitat}[1]{\emph{\foreignquote{german}{#1}}}

\newcommand{\titleitem}[1]{\item \textbf{#1} \par}

\begin{document}
\thispagestyle{plain}

\heading{Der Kommentar}

\sub{Definition}

Ein Kommentar ist eine subjektive, wertende Stellungnahme und \"{A}u\ss{}erung der pers\"{o}nlichen Meinung des Schreibers zu einem Text, Thema oder Ereiginis. Der Kommentar geh\"{o}rt zu den untersuchenden, bewertenden, kritisierenden, meinungs\"{a}u\ss{}ernden, argumentierenden, \"{u}berzeugenden sowie appellierenden Textsorten. Die pers\"{o}nliche, subjektive jedoch objektiv und logisch begr\"{u}ndete Meinung des Autors sollte klar und koh\"{a}rent aus dem Text zu entnehmen sein. Die Sprache kann gehoben oder auch allt\"{a}glich sein (nicht umgangsprachlich), sollte jedoch rhetorische Mittel enthalten und Variationen in Wortschatz und Satzstruktur aufweisen. Wichtig ist es auch, zu m\"{o}glichen Bezugstexten eine Referenz herzustellen.

\sub{Aufbau}

Der Aufbau eines Kommentar folgt meist einem bestimmten Schema, bestehend aus einer Mixtur zwischen Analyse und Diskussion des Bezugstextes sowie der Darstellung der eigenen Meinung und Argumente.

\begin{enumerate}

	\titleitem{\"{U}berschrift}

	Die \"{U}berschrift eines Kommentars sollte kreativ und reizend sein um das Leserinteresse zu wecken.

	\titleitem{Einleitung}

	In der Einleitung muss zun\"{a}chst ein passender Einstieg und somit eine passende Einf\"{u}hrung zum Thema gew\"{a}hlt werden. Dieser Einstieg kann entweder durch eine Begriffsdefinition, eine Nennung bzw. Diskussion eines aktuellen oder auch historischen Ereignisses oder durch ein Zitat eines Politikers, Philosophen, Wissenschaftlers oder sonstigen Experten erfolgen. Das Ziel einer Einleitung ist es, das Leserinteresse weiter wecken bzw. zu bewahren und kann einen kurzen Ausblick auf den weiteren Inhalt des Kommentars geben. Letztlich sollte eine \"{U}berleitung zum Hauptteil erfolgen. 

	\titleitem{Hauptteil}

	Der Hauptteil eines Kommentars sollte zun\"{a}chst eine Referenz zum Bezugstext hergestellen --- samt Titel, Autor, Erscheinungsdatum sowie -ort und dessen Aussagen kurz und b\"{u}ndig zusammenfassen. Aussagen des Bezugstextes, die nicht relevant zum Thema des Kommentars sind, k\"{o}nnen vernachl\"{a}ssigt werden. Danach sollten pers\"{o}nliche, subjektive Argumente und Ansichten des Autors linear geschildert sowie dessen logische, objektive Basis durch dem \emph{Behauptung --- Beweis --- Beispiel} Modell klargestellt werden. Die eigene Meinung sollte auf jeden Fall klar dargestellt und erkennbar sein. Es ist g\"{u}nstig, ihre Aussagekr\"{a}ftigkeit durch rhetorische Mittel wie Anaphern, Alliterationen, Analogien oder Ironie bzw. Sarkasmus zu st\"{a}rken.

	\titleitem{Schluss}

	Im Schluss sollten die genannten Meinungen und Ansichten zusammgefasst werden, um somit eine logische Schlussfolgerung zu ziehen. Ebenso ist es m\"{o}glich, einen Ausblick auf die Zukunft zu geben, m\"{o}gliche L\"{o}sungen vorzuschlagen und / oder einen Appell an den Leser, eine bestimmte Gruppe oder die gesamte Gesellschaft zu richten.

\end{enumerate}

\pagebreak

\sub{Stil}

\begin{itemize}

	\item Verwendung vieler rhetorischer Stilmittel

	\item Gehobene oder allt\"{a}gliche Sprache --- \zitat{peppig}

	\item Variation in Wortschatz und Satzstruktur

	\item Sehr subjektiv und pers\"{o}nlich, jedoch kein \zitat{Ich}

	\item Redewendungen oder Zitate k\"{o}nnen vorkommen

	\item Provokation erw\"{u}nscht: \zitat{Entschuldigung, aber haben wir noch alle Tassen im Schrank?}

	\item Logische Argumentation: \emph{Behauptung --- Beweis --- Beispiel}

	\item Ausrufe bzw. eine gewisser Verdruss erlaubt, z.B. \zitat{Es hat ja so kommen m\"{u}ssen}

\end{itemize}

\end{document}