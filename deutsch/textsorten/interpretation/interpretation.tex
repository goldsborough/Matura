% Textinterpretation

\documentclass[11pt]{article}

\usepackage[a4paper, margin=1in]{geometry}

\usepackage{amsmath}

\usepackage{amssymb}

\usepackage[german]{babel}

\usepackage[autostyle=true]{csquotes}

\usepackage{libertine}

\usepackage[libertine]{newtxmath}

\usepackage{tikz}

\usepackage{gensymb}

\usepackage{fancyhdr}

\usepackage{amsfonts}

\usepackage{pgfplots}

\pgfplotsset{compat=1.10}

\usepackage{multicol}

\usepackage{caption}

\usepackage{floatrow}

\everymath{\displaystyle}

% Header / footer settings

\pagestyle{fancy}
\fancyhf{}
\renewcommand{\headrulewidth}{0.2mm}
\fancyhead[C]{Funktionen}
\renewcommand{\footrulewidth}{0.2mm}
\fancyfoot[L]{Peter Goldsborough}
\fancyfoot[C]{\thepage}
\fancyfoot[R]{\today}

\fancypagestyle{plain}{%
	\fancyhf{}
	\renewcommand{\headrulewidth}{0mm}%
	\renewcommand{\footrulewidth}{0.2mm}%
	\fancyfoot[L]{Peter Goldsborough}
	\fancyfoot[C]{\thepage}
	\fancyfoot[R]{\today}
}


\setlength{\headheight}{15pt}

\setlength{\parindent}{0pt}

\addtolength{\parskip}{\baselineskip}


\newcommand{\overbar}[1]{\mkern 1.5mu\overline{\mkern-1.5mu#1\mkern-1.5mu}\mkern 1.5mu}

\newcommand{\heading}[1]{\begin{center}\Huge \textbf{#1}\end{center}\par}

\newcommand{\sub}[1]{\vspace{\parskip}{\LARGE\textbf{#1}}}

\newcommand{\subsub}[1]{{\Large \textbf{#1}}}

\newcommand{\subsubsub}[1]{\textbf{#1}}

\newcommand{\colvec}[1]{\begin{pmatrix}#1\end{pmatrix}}

\newcommand{\extrapar}{\par\vspace{\baselineskip}}

\newcommand{\zitat}[1]{\foreignquote{german}{#1}}

\newcommand{\bolditem}[1]{\item \textbf{#1}}

\newcommand{\titleitem}[1]{\bolditem{#1}\par}

\newcommand{\defas}{ \dots \,\,}

\begin{document}
\thispagestyle{plain}

\heading{Die Textinterpretation}

\sub{Definition}

Die Textinterpretation ist eine analysiserende, beschreibende, zusammenfassende, deutende und erl\"{a}uternde Textsorte, bei der es gilt, einen Prosa- oder Lyriktext auf wesentliche Merkmale wie Sprache, Stil sowie Aufbau zu analysieren. Ein besonderer Schwerpunkt liegt auf der Analyse, Beschreibung und Interpretation des Inhalts, wof\"{u}r man Interpretationshypothesen darlegen sollte. Wichtig ist ebenso, f\"{u}r alle Behauptungen zum Stil, Inhalt oder sonstigen Merkmalen des Werkes Textzitate und Beweise darzulegen.

\sub{Aufbau}

Der Aufbau ist stark von den Operatoren der Aufgabe abh\"{a}ngig, sollte jedoch, wenn auch in begrenzter Form, folgenden Aufbau aufweisen:

\begin{enumerate}
	\titleitem{Einleitung}

	Die Einleitung der Textinterpretation sollte wesentliche Informationen zum vorliegenden und zu deutenden Werk nennen, einschlie\ss{}lich Titel, Textsorte, Autor, Erscheinungdatum und in wenigen, b\"{u}ndigen Worten den Inhalt. Der Einstieg in die Einleitung bzw. der Textinterpretation kann durch einen aktuellen Anlass, ein Zitat oder eine allgemeine, zum Textinhalt passende Aussage oder Beobachtung erfolgen. Letztlich sollte noch ein Ausblick auf die Interpretationshypothese gegeben und passend zum Hauptteil \"{u}bergeleitet werden.

	\titleitem{Hauptteil}

	Es gibt f\"{u}nf Hauptkategorien, in welche man Merkmale oder Aussagen bez\"{u}glich eines Textes eingliedern k\"{o}nnte, diese sind unten angef\"{u}hrt. Inhaltliche und deutende Aspekte sind besonders wichtig, andere tragen je nach Aufgabenstellung verschiedene Gewichtungen. Der erste Absatz des Hauptteils wird mit Sicherheit den Inhalt abhandeln m\"{u}ssen, folgende in irgendeiner Form die Sprache, den Stil sowie die Deutung der Motive des Werkes. Ebenso sollte nicht vergessen werden, Bezug auf den Titel des Textes zu nehmen!

		\begin{itemize}
			\bolditem{Inhalt}: Die Analyse des Inhalts bezieht sich auf das Verst\"{a}ndnis sowie die Deutung des Textes, rein nach deren Kernaussagen. Stilmittel oder Sprache sind bei der Diskussion des Inhalts nicht relevant. Der inhaltliche Teil der Interpretation sollte die f\"{u}nf \zitat{W} Fragen beantworten: \emph{Wer} sind die Hauptfiguren bzw. das lyrische Ich? \emph{Was} geschieht? \emph{Wo} nimmt die Handlung statt? \emph{Wann} nimmt die Handlung statt? \emph{Wie} geschieht die Handlung? Nicht alle Texte werden Antworten f\"{u}r alle Fragen bieten, sie sollten jedoch im Kopf behalten werden.

			\bolditem{Thema und Motive}: Diese Kategorie umfasst die Deutung der Motive sowie der Beschreibung des \"{u}bergeordneten Themas. Man sollte sowohl die Macro Ebene des Werkes nach Motiven untersuchen, also \"{u}berliegende Kernaussagen und Themen des Werkes, als auch auf der Micro Ebene einzelne Verse oder Strophen, deren Motive, Aussagen sowie Bedeutung f\"{u}r den Text darstellen. Die Deutung der Motive beeinschlie\ss{}t ebenso die Einbezugnahme des Titels. Dieser kann die Bedeutung der Macro als auch der Micro Ebene bedeutend beeinflussen und sollte daher gr\"{u}ndlich untersucht werden.

\pagebreak

			\bolditem{Aufbau}: Der Aufbau des Textes beschreibt alle formalen Merkmale des Werkes. Bei einem lyrischen Text sind dies Versma\ss{}, Reimschema sowie Gliederung --- Vers- und Strophenzahl. Kontempor\"{a}re Lyrik weist oftmals kein Reimschema auf, Reimschemen wie ABAB (Kreuzreim), ABBA (Umarmender Reim), AAAA (Haufenreim) oder AABB (Paarreim) sollte man dennoch erkennen k\"{o}nnen. Bei Prosa Texten umfasst diese Kategorie die Beschreibung der Gliederung sowie des inhaltlichen Aufbaus (nach Abs\"{a}tzen).

			\bolditem{Stil und Sprache}: Auch die Untersuchung stilistischer und sprachlicher Merkmale sollte unternommen werden. Stilmittel wie Ringkomposita, Alliterationen, Metaphern usw. sind ein Grundwerkzeug aller Lyriker. Bei rhetorischen Mitteln ist es wichtig, auf passende Textstellen hinzuweisen und diese gegebenfalls zu zitieren. Sprachliche Merkmale umfassen den Satzbau --- hypotaktisch oder parataktisch --- das sprachliche bzw. stilistische Niveau --- allt\"{a}glich, gehoben oder vulg\"{a}r --- die Position des lyrischen Ichs (Erz\"{a}hlperspektive) sowie des Addressaten, die Zeit des Textes --- meist Pr\"{a}sens oder Perfekt --- als auch die Nennung der erz\"{a}hlten Zeit (Handlungszeitraum des Werkes).
		\end{itemize}

		Immer beachten: Behauptung, Beweis, Beispiel!

		\titleitem{Schluss}

		Der Schluss der Textinterpretation sollte die gemachten Erkenntnisse zusammenfassen. Ebenso ist es m\"{o}glich, eine pers\"{o}nliche Wertung oder Stellungnahme zum Werk zu \"{a}u\ss{}ern. Oft ist gefragt, einen Bezug zur Gegenwart herzustellen.
\end{enumerate}

\sub{Anmerkungen}

\begin{itemize}

	\item Im Pr\"{a}sens verfassen.

	\item Sachlich-distanziertes Register. Kein \zitat{Ich}, aus\ss{}er es ist nach einer pers\"{o}nlichen Stellungnahme gefragt (z.B. im Schluss).

	\item Korrekt Zitieren.

	\item Gehobene, analytische Sprache und Verweundung von Fachtermini.

\end{itemize}

\end{document}