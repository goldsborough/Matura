% Eroerterung

\documentclass[11pt]{article}

\usepackage[a4paper, margin=1in]{geometry}

\usepackage{amsmath}

\usepackage{amssymb}

\usepackage[german]{babel}

\usepackage[autostyle=true]{csquotes}

\usepackage{libertine}

\usepackage[libertine]{newtxmath}

\usepackage{tikz}

\usepackage{gensymb}

\usepackage{fancyhdr}

\usepackage{amsfonts}

\usepackage{pgfplots}

\pgfplotsset{compat=1.10}

\usepackage{multicol}

\usepackage{caption}

\usepackage{floatrow}

\everymath{\displaystyle}

% Header / footer settings

\pagestyle{fancy}
\fancyhf{}
\renewcommand{\headrulewidth}{0.2mm}
\fancyhead[C]{Funktionen}
\renewcommand{\footrulewidth}{0.2mm}
\fancyfoot[L]{Peter Goldsborough}
\fancyfoot[C]{\thepage}
\fancyfoot[R]{\today}

\fancypagestyle{plain}{%
	\fancyhf{}
	\renewcommand{\headrulewidth}{0mm}%
	\renewcommand{\footrulewidth}{0.2mm}%
	\fancyfoot[L]{Peter Goldsborough}
	\fancyfoot[C]{\thepage}
	\fancyfoot[R]{\today}
}


\setlength{\headheight}{15pt}

\setlength{\parindent}{0pt}

\addtolength{\parskip}{\baselineskip}


\newcommand{\overbar}[1]{\mkern 1.5mu\overline{\mkern-1.5mu#1\mkern-1.5mu}\mkern 1.5mu}

\newcommand{\heading}[1]{\begin{center}\Huge \textbf{#1}\end{center}\par}

\newcommand{\sub}[1]{\vspace{\parskip}{\LARGE\textbf{#1}}}

\newcommand{\subsub}[1]{{\Large \textbf{#1}}}

\newcommand{\subsubsub}[1]{\textbf{#1}}

\newcommand{\colvec}[1]{\begin{pmatrix}#1\end{pmatrix}}

\newcommand{\extrapar}{\par\vspace{\baselineskip}}

\newcommand{\zitat}[1]{\foreignquote{german}{#1}}

\newcommand{\bolditem}[1]{\item \textbf{#1}}

\newcommand{\titleitem}[1]{\bolditem{#1}\par}

\newcommand{\defas}{ \dots \,\,}

\begin{document}
\thispagestyle{plain}

\heading{Die Er\"{o}rterung}

\sub{Definition}

Eine Er\"{o}rterung ist eine schriftliche Auseinandersetzung mit einem Thema. Sie geh\"{o}rt zu den argumentierenden, untersuchenden, bewertenden, kritisierenden sowie \"{u}berzeugenden Textsorten, und stellt eine kritische, analytische, objektive jedoch meingungsbetonte und -bildende Untersuchung einer Frage oder einer Hypothese dar. Eine Er\"{o}rterung kann entweder linear --- wobei es gilt, eine unstrittige Frage einseitig zu beleuchten und zu beantworten --- oder dialektisch sein --- wo \emph{pro} \textbf{und} \emph{contra}, These \textbf{und} \emph{Gegen}thsese eines Themas bearbeitet werden m\"{u}ssen. Nach der Analyse und der Auseinandersetzung mit dem Kernthema, soll eine Er\"{o}rterung auch eine L\"{o}sung bzw. eine abschlie\ss{}ende Bewertung liefern.

\sub{Aufbau}

Der Aufbau eine Er\"{o}rterung ist im gro\ss{}en und ganzen fur sowohl die lineare als auch die dialektische Variante gleich und folgt immer einem bestimmten, vordefinierten sowie operatoren- bzw. aufgabenstellungsunabh\"{a}ngigen Schema:

\begin{enumerate}

	\titleitem{\"{U}berschrift}

	Die \"{U}berschrift einer Er\"{o}rterung muss nicht kreativ oder speziell sein. Es gen\"{u}gt, die gestellte Frage der Er\"{o}rterung zu nennen --- z.B. \zitat{Wird Umweltechnologie durch den \"{o}sterreichischen Staat ausreichend gef\"{o}rdert?}

	\titleitem{Einleitung}

	Die Einleitung sollte einen gelungenen Einstieg bzw. eine Einf\"{u}hrung zum Thema beinhalten, beispielsweise durch eine Begriffsdefinition, durch Nennung eines aktuellen Anlasses, durch ein passendes Zitat eines Politikers, Philosophen, Wissenschaftlers oder sonstigem Experten oder durch Beschreibung eines historischen Ereignises oder Phenom\"{a}ns. Das Thema --- sowie im Falle einer dialektischen Er\"{o}rterung die beiden polarisierenden Facetten der Debatte --- sollten genannt werden. Ebenso ist es wichtig, eine Referenz zu Zeitungsartikeln oder sonstigen zur Aufgabenstellung geh\"{o}renden Texten herzustellen, durch Nennung des Titels, des Autors, des Erscheinungsortes sowie -datums. Letztlich sollte die Einleitung noch zum Hauptteil \"{u}berleiten.

	\titleitem{Hauptteil}

	Der Aufbau des Hauptteils variiert hier zwischen der linearen und der dialektischen Variante. 

	\begin{itemize}

		\titleitem{Linear}

		Der Hauptteil einer linearen Er\"{o}rterung sollte zwischen zwei und drei Argumente bzw. Abs\"{a}tze enthalten, in welchen der eigene Standpunkt auf objektive, kritische und analytische Weise genannt sowie durch Beispiele und Beweise gest\"{a}rkt wird. Der erste Absatz sollte das schw\"{a}chste Argument behandeln sodass die folgenden Abs\"{a}tze bzw. die darin pr\"{a}sentierten Argumente in St\"{a}rke steigen. Der letzte Absatz enth\"{a}lt das st\"{a}rkste Argument.

		\titleitem{Dialektisch}

		Bei der dialektischen Er\"{o}rterung sollte der Hauptteil das Sanduhrmodell befolgen. Dabei nennt der erste Absatz des Hauptteils das st\"{a}rkere Gegenargument bzw. das st\"{a}rkere Argument f\"{u}r die Gegenthese --- also \emph{contra} --- und der zweite dann das schw\"{a}chere Gegenargument. Danach folgt die Umkehrung der Sanduhr. Der dritte Absatz nennt das schw\"{a}chere Argument f\"{u}r die These --- \emph{pro} --- und der vierte das st\"{a}rkere Argument \emph{pro}.

	\end{itemize}

	\titleitem{Schluss}

	Der Schluss dient dazu, die er\"{o}rterten Standpunkte nochmals zusammenzufassen, sie abzuw\"{a}gen und schlussendlich eine Bewertung bzw. eine Konklusion abzugeben. Es sollten hier keine weiteren Argumente genannt werden! Allerdings ist es m\"{o}glich, einen Ausblick auf die Zukunft zu geben oder einen Appell an den Leser bzw. an die Gesellschaft zu richten.

\end{enumerate}

\sub{Stil}

\begin{itemize}
	\item variantenreicher Wortschatz und Aufbau

	\item neutrale, objektive, gehobene jedoch nicht unn\"{o}tig komplexe Sprache

	\item logische, koh\"{a}rente Schlussfolgerung bzw. Argumentierung

	\item Argument bestehen aus \textbf{Behauptung --- Beweis --- Beispiel}

	\item Objektivit\"{a}t bewahren jedoch eigenen Standpunkt auch herausbringen

	\item \zitat{Ich} vermeiden
\end{itemize}

\end{document}