% Leserbrief

\documentclass[11pt]{article}

\usepackage[a4paper, margin=1in]{geometry}

\usepackage{amsmath}

\usepackage{amssymb}

\usepackage[german]{babel}

\usepackage[autostyle=true]{csquotes}

\usepackage{libertine}

\usepackage[libertine]{newtxmath}

\usepackage{tikz}

\usepackage{gensymb}

\usepackage{fancyhdr}

\usepackage{amsfonts}

\usepackage{pgfplots}

\pgfplotsset{compat=1.10}

\usepackage{multicol}

\usepackage{caption}

\usepackage{floatrow}

\everymath{\displaystyle}

% Header / footer settings

\pagestyle{fancy}
\fancyhf{}
\renewcommand{\headrulewidth}{0.2mm}
\fancyhead[C]{Funktionen}
\renewcommand{\footrulewidth}{0.2mm}
\fancyfoot[L]{Peter Goldsborough}
\fancyfoot[C]{\thepage}
\fancyfoot[R]{\today}

\fancypagestyle{plain}{%
	\fancyhf{}
	\renewcommand{\headrulewidth}{0mm}%
	\renewcommand{\footrulewidth}{0.2mm}%
	\fancyfoot[L]{Peter Goldsborough}
	\fancyfoot[C]{\thepage}
	\fancyfoot[R]{\today}
}


\setlength{\headheight}{15pt}

\setlength{\parindent}{0pt}

\addtolength{\parskip}{\baselineskip}


\newcommand{\overbar}[1]{\mkern 1.5mu\overline{\mkern-1.5mu#1\mkern-1.5mu}\mkern 1.5mu}

\newcommand{\heading}[1]{\begin{center}\Huge \textbf{#1}\end{center}\par}

\newcommand{\sub}[1]{\vspace{\parskip}{\LARGE\textbf{#1}}}

\newcommand{\subsub}[1]{{\Large \textbf{#1}}}

\newcommand{\subsubsub}[1]{\textbf{#1}}

\newcommand{\colvec}[1]{\begin{pmatrix}#1\end{pmatrix}}

\newcommand{\extrapar}{\par\vspace{\baselineskip}}

\newcommand{\zitat}[1]{\foreignquote{german}{#1}}

\newcommand{\bolditem}[1]{\item \textbf{#1}}

\newcommand{\titleitem}[1]{\bolditem{#1}\par}

\newcommand{\defas}{ \dots \,\,}

\begin{document}

\pagenumbering{gobble}

\heading{Der Leserbrief}

\sub{Definition}

Schriftliche, entweder kritische oder best\"{a}rkende, jedenfalls appellierende, \"{u}berzeugende und meinungs\"{a}u\ss{}ernde Stellungnahme zu einem Artikel, einem aktuellen Geschehen oder sonstigem Beitrag in einer Zeitung oder Zeitschrift. Es soll eine klar verst\"{a}ndliche, einheitliche (nicht dialektische) Meinung zu einem Thema geschildert werden, welche durch sprachlich-stilistiche und rhetorische Mittel sowie Ironie oder Sarkasmus das Interesse des Lesers bindet. Man muss eine Referenz zum urspr\"{u}nglichen Medium herstellen und dessen Aussagen auch kurz zusammenfassen.

\sub{Aufbau}

\begin{enumerate}
  \item \textbf{\"{U}berschrift}
  \par
        Provozierende \"{U}berschrift die das Interesse des Lesers / der Leserin erweckt und ihn / sie zum lesen anregt.

  \item \textbf{Einleitung}
  \par
  Klare Referenz zum urspr\"{u}nglichen Text herstellen, inklusive Titel, Datum, Erscheinungsort und Autor (wenn jeweils vorhanden). In ein paar Worten ausdr\"{u}cken, ob man f\"{u}r oder gegen die Meinung des Autors ist bzw. welche Ansicht in dem folgenden Leserbrief vertreten werden wird (\zitat{Ich teile die Meinung des Autors nicht!}).

  \item \textbf{Hauptteil}
  \par
  Je nach Operatoren den Inhalt und die vertretenen Ansichten des Referenztexts kurz zusammefassen und kritisch bel\"{a}uchten. Die eigene Meinung durch klare, kompakte Argumente ausdr\"{u}cken und den Leser bestm\"{o}glichst davon \"{u}berzeugen. Gegenargumente nicht nennen, wenn nur um sie zu entkr\"{a}ften. Der Haupttteil sollte Wortvariation sowie rhetorische Stilfiguren enthalten um die Aufmerksamkeit des Lesers / der Leserein aufrecht zu erhalten.

  \item \textbf{Schluss}
  \par
  Meinungen und Ansichten in einem Satz zusammenfassen und letzten, impr\"{a}gnanten Appell oder Ausblick \"{a}u\ss{}ern.

\end{enumerate}

\sub{Stil}

Ein Leserbrief ist addressatenorientiert, jedoch gegen\"{u}ber der gesamten Leserschaft und nicht gegen\"{u}ber einer einzigen Person. Das Interesse der Leser muss durch sprachlisch-stilistische und rhetorische Stilmittel aufrecht erhalten werden. Wortvariation und ein gehobener Wortschatz sowie Ironie oder Sarkasmus k\"{o}nnen den Text ebenso lesenswerter machen. Zur Bekr\"{a}ftigung der eigenen Meinung k\"{o}nnen Beispiele und Vergleiche verwendet werden.

\end{document}
