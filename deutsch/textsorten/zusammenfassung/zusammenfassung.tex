% Zusammenfassung

\documentclass[11pt]{article}

\usepackage[german]{babel}

\usepackage[autostyle=true]{csquotes}

\usepackage[a4paper, margin=1in]{geometry}

\usepackage{libertine}

\setlength{\parindent}{0pt}

\addtolength{\parskip}{\baselineskip}

\newcommand{\extrapar}{\par\vspace{\baselineskip}}

\newcommand{\heading}[1]{\begin{center}\Huge \textbf{#1} \end{center}}

\newcommand{\sub}[1]{{\Large \textbf{#1}}\par}

\newcommand{\subsub}[1]{{\large \textbf{#1}}\par}

\newcommand{\zitat}[1]{\emph{\foreignquote{german}{#1}}}

\newcommand{\titleitem}[1]{\item \textbf{#1} \par}

\begin{document}
\thispagestyle{plain}

\heading{Die Zusammenfassung}

\sub{Definition}

Eine Zusammenfassung dient dazu, einen Ausgangstext auf seine wesentlichen und wichtigen Aussagen zu reduzieren, zu verdichten, zusammenzufassen. Es handelt sich um eine stark textbezogene, beschreibende und weniger um eine argumentierende, er\"{o}rternde oder kritisierende Textsorte. 

\sub{Aufbau}

Eine Zusammenfassung hat normalerweise eine vordefinierte Struktur, von der aber auch eventuell abgewichen werden kann (z.B. muss es keinen expliziten Schluss geben, es kann auch die Zusammenfassung der letzten Aussage des Textes den Schluss bilden).

\begin{itemize}

	\titleitem{\"{U}berschrift}

	Eine Zusammenfassung braucht keine kreative oder reizende \"{U}berschrift. Es gen\"{u}gt: \zitat{Zusammenfassung des Textes: ...}. Ebenso kann man auch \"{u}berhaupt keine \"{U}berschrift schreiben.

	\titleitem{Einleitung}

	Die Einleitung muss Referenz zum Bezugstext herstellen und dessen Titel, Autor, Textsorte, Erscheinungsdatum- sowie ort nennen. Auch sollte die Essenz des Textes kurz und b\"{u}ndig in einem Satz zusammengefasst werden.

	\titleitem{Hauptteil}

	Im Hauptteil werden die wichtigsten Aussagen des Textes chronologisch, strukturiert und inhaltsgetr\"{a}u im Pr\"{a}sens wiedergegeben. Die Gliederung des Hauptteils sollte die Gliederung des Ausgangstextes wiederspiegeln. Es soll keine pers\"{o}nliche Interpretation und kein subjektiver Kommentar zum Text gegeben werden --- das Sprachregister ist neutral, objektiv. Man sollte versuchen, den Text zu paraphrasieren bzw. in eigenen Worten zusammenzufassen und direkte oder indirekte Zitate zu vermeiden.

	\titleitem{Schluss}

	Im Schluss einer Zusammenfassung ist es m\"{o}glich, die Hauptaussage des Textes nochmals zu unterstreichen. Ebenso kann der Schluss aber auch einfach die Zusammenfassung der letzten Aussage des Textes sein. Der Schluss w\"{a}re somit nicht explizit, sondern im Grunde der Abschluss des Hauptteils.

\end{itemize}

\sub{Stil}

\begin{itemize}

	\item Objektives, unpers\"{o}nliches, neutrales Sprachregister --- kein \zitat{Ich}

	\item Immer Pr\"{a}sens bzw. Konjunktiv bei Paraphrasierung einer Aussage

	\item Eigenst\"{a}ndige Formulierungen (wenn m\"{o}glich) 

	\item Korrekte Zitierung

	\item Keine Bewertung, keine Deutung: Sachlichkeit

\end{itemize}

\end{document}