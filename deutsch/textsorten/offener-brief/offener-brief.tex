% Offener Brief

\documentclass[11pt]{article}

\usepackage[german]{babel}

\usepackage[autostyle=true]{csquotes}

\usepackage[a4paper, margin=1in]{geometry}

\usepackage{libertine}

\setlength{\parindent}{0pt}

\addtolength{\parskip}{\baselineskip}

\newcommand{\extrapar}{\par\vspace{\baselineskip}}

\newcommand{\heading}[1]{\begin{center}\Huge \textbf{#1} \end{center}}

\newcommand{\sub}[1]{{\Large \textbf{#1}}\par}

\newcommand{\subsub}[1]{{\large \textbf{#1}}\par}

\newcommand{\zitat}[1]{\emph{\foreignquote{german}{#1}}}

\newcommand{\titleitem}[1]{\item \textbf{#1} \par}

\begin{document}
\thispagestyle{plain}

\heading{Offener Brief}

\sub{Definition}

Der offene Brief z\"{a}hlt zu den Sonderformen der Textsorte \zitat{Brief} und ist zur Ver\"{o}ffentlichung gedacht. Im Gegensatz zu allen anderen Briefformen wird der offene Brief als Grundlage f\"{u}r einen \"{o}ffentlichen
Diskussionsprozess konzipiert und ist somit sowohl empf\"{a}nger- als auch \"{o}ffentlichkeitsorientiert. Sein offensiver, appellierender, \"{u}berzeugender und meinungsbetonter Charakter verlangt nach einer Reaktion des Empf\"{a}ngers, die im Idealfall \"{o}ffentlich erfolgt. Um den pers\"{o}nlichen Standpunkt gut zu positionieren, muss man seine Argumente logisch, sachlich sowie kritisch aufbauen. Da es sich um einen Brief handelt, m\"{u}ssen bestimmte Formalismen eingehalten werden.

\sub{Aufbau}

\begin{enumerate}

	\titleitem{Empf\"{a}nger und Absender}

	Ein Briefkopf f\"{u}hrt meistens die Kontaktdaten des Absenders sowie jene des Empf\"{a}ngers an (zuerst Absender, dann Empf\"{a}nger).

	\titleitem{Datum}

	Das Datum steht rechtsb\"{u}ndig unter den Kontaktdaten des Empf\"{a}ngers und \"{u}ber der Gru\ss{}formel. 

	\titleitem{Gru\ss{}formel}

	Ein Brief hat immer einen bestimmten Adressaten, daher auch eine formale Anrede bzw. eine Gru\ss{}formel vor der Einleitung. 

	\titleitem{Einleitung}

	In der Einleitung muss genannt werden, wer den Brief schreibt und aus welchem Anlass (Referenz zu Text/Zeitungsartikel, Ereignis, aktuelle Diskussion, Forderung, Richtigstellung). Der Sachverhalt bzw. die Thematik kann kurz und b\"{u}ndig in einem Satz zusammengefasst werden, um dann eine fl\"{u}ssige \"{U}berleitung zum Hauptteil zu erstellen. Eine Einleitung soll vor allem das Interesse des Lesers wecken und dazu anregen, weiterzulesen. Dies gelingt durch korrekte Anwendung sprachlicher Mittel und exakter Anf\"{u}hrung des Schreibanlasses.

	\titleitem{Hauptteil}

	Im Hauptteil werden die eigentlichen Thesen genannt und durch logische Argumente sowie durch Beispiele und Beweise gest\"{a}rkt. Die pers\"{o}nlichen Forderungen, W\"{u}nsche oder Appelle m\"{u}ssen klar genannt und begr\"{u}ndet werden, um somit eine Reaktion des Empf\"{a}ngers einfordern zu k\"{o}nnen --- beispielsweise eine \"{o}ffentliche Stellungnahme oder die Korrektur einer politischen Entscheidung. Der letzte Absatz des Hauptteils sollte einen Ma\ss{}nahmenkatalog anf\"{u}hren, der Ideen zur Behebung des behandelten Problems bzw. einen Weg empfiehlt, zu einer f\"{u}r alle Beteiligten passenden L\"{o}sung zu gelangen.

	\titleitem{Schluss}

	Im Schluss m\"{u}ssen nochmals alle genannten Argumente, Standpunkte und Forderungen zusammengefasst werden, um den Leser bzw. den Empf\"{a}nger mit einem letzten, impr\"{a}gnanten Appell zu einer (Re-)Aktion zu bewegen.

	\titleitem{Abschiedsformel}

	So wie eine Gru\ss{}formel, muss ein (offener) Brief ebenso eine formale Abschiedsformel anf\"{u}hren.

\end{enumerate}

\sub{Schema}

Ein offener Brief kann folgendes Schema befolgen:

{ \em

Peter Goldsborough\\
Untere Fellacher Str. 46\\
9500 Villach\\
\"{O}sterreich

B\"{u}rgermeister Helmut Manzenreiter\\
Rathausplatz 1\\
9500 Villach\\
\"{O}sterreich
\begin{flushright} 3. M\"{a}rz 2015 \end{flushright}
Sehr geehrter Herr B\"{u}rgermeister!

... Einleitung ...

... Hauptteil ...

... Schluss ...

Mit freundlichen Gr\"{u}\ss{}en,

Peter Goldsborough

} % em

\extrapar

\sub{Stil}

\begin{itemize}
	\item Adressatenorientiert
	\item Pointierte Formulierungen
	\item Pr\"{a}gnanter Ausdruck
	\item Rhetorische Mittel
	\item Formales, h\"{o}fliches, neutrales Sprachregister
	\item Pr\"{a}sens
	\item Abwechslungsreiche Wortwahl und Syntax
\end{itemize}

\end{document}