% Writing an Essay

\documentclass[11pt]{article}

\usepackage[german]{babel}

\usepackage[autostyle=true]{csquotes}

\usepackage[a4paper, margin=1in]{geometry}

\usepackage{libertine}

\setlength{\parindent}{0pt}

\addtolength{\parskip}{\baselineskip}

\newcommand{\extrapar}{\par\vspace{\baselineskip}}

\newcommand{\heading}[1]{\begin{center}\Huge \textbf{#1} \end{center}}

\newcommand{\sub}[1]{{\Large \textbf{#1}}\par}

\newcommand{\subsub}[1]{{\large \textbf{#1}}\par}

\newcommand{\zitat}[1]{\emph{\foreignquote{german}{#1}}}

\newcommand{\titleitem}[1]{\item \textbf{#1} \par}

\begin{document}
\thispagestyle{plain}

\heading{Essay}

\sub{Definition}

An essay is an either opinionated or argumentative text type that requires critical analysis of an article, event or any other topic or medium; logical, coherent and persuasive argumentation with a clearly structured line of thought; neutral register and style as well as a formal but elevated vocabulary. An essay can be either linear (\emph{opinion essay}) --- where only \textbf{one} opinion is presented, discussed and defended --- or dialectic (\emph{argumentative essay}) --- which deals with a topic \emph{pro and contra}, presenting both positive as well as negative sides but concluding with a clear standpoint regarding which of the two prevails. 

\sub{Structure}

The structure of an essay should be clear and pre-determined. However, it is not always the same and can vary depending on the assignment given. The general schema of an essay presented here can be re-structured appropriately if necessary:

\begin{enumerate}

	\titleitem{Title}

	The title of an essay does not need to be fancy, catchy or imaginative. Often, the title of an essay is simply the question asked by the assignment. It is equally possible to not write a title at all.

	\titleitem{Introduction}

	The introduction should begin the discussion and analysis of the subject presented and should, in a few sentences, summarize the contents of the subsequent paragraphs and arguments. Common methods of introducing the reader to the topic include a quote by a famous author, politician or other personality / expert; a reference to a historic event or occurence or to a recent occasion, problem or (social, economic, political) phenomenon. The purpose of the introduction is to stimulate the reader's interest in the topic and its discussion in the essay. Lastly, it is essential to conclude the introduction with a transition into the main body, e.g. by a rhetorical question. In an opinion essay, the introduction should already state the writer's opinion.

	\titleitem{Main Body}

	The main body often consists of three paragraphs. In the first paragraph one should present either the strongest or the weakest argument, depending on whether the following arguments will increase or decrease in strength. Therefore, option 1 is to present the strongest argument in the first paragraph and then decrease in strength, with the weakest argument in paragraph three; option 2 is to discuss one's weakest argument in the first paragraph and ascend in strength in the subsequent paragraphs, the strongest argument being in paragraph three, right before the conclusion. For an opinion essay, it is also possible to weaken counter-arguments to one's opinion.

	\extrapar

	In case of an argumentative / dialectic essay, the individual paragraphs can vary between full \emph{pro} and full \emph{con}, or each consist of one positive and one negative view of a subtopic. 

	\extrapar

	Each paragraph should start with a transition from the last along with a topic sentence that summarizes the essence of the views discussed in this paragraph. It should end with concluding words regarding the argument presented and provide a transition to the next paragraph.

\pagebreak

	\titleitem{Conclusion}

	The conclusion of an essay should summarize the opinions and arguments presented and formulate a clear, all-encompassing, concluding view that leaves no doubts about its logical validity, rational basis as well signficance in the domain of the topic. Moreover, the conclusion can include a call to action by the reader, a specifc person / group or society as a whole. It is also possible to provide an outlook to the future based on the conclusions drawn in the essay. For an opinion essay, it is absolutely crucial that the writer's opinion be clear and should be stated again in the conclusion to put further emphasis on it.

\end{enumerate}

\sub{Style}

The style of an essay is entirely formal. This includes:

\begin{itemize}

	\item Neutral register --- no personal pronouns (\emph{I, you, they}). Prefer impersonal language.

	\item Use of the passive voice. 

	\item No slang or contractions (\emph{I will} not \emph{I'll}).

	\item Elevated, while not unecessarily complex, vocabulary (reduced use of helping words, jargon allowed).

\end{itemize}

\sub{Essentials}

When writing an essay, the following things should be kept in mind:

\begin{itemize}

	\item Have a clear, pre-determined structure. The writing process itself is only a filling in of the mold / pattern created during preparation.

	\item Use appropriate linking devices to transition between paragraphs.

	\item Ensure that your reasoning and arguments are founded on logic and reason. Check for fallacious thinking or conclusions.

	\item Fulfill all additional tasks required for succesful completion of the assignment.

	\item Stay clear, concise and do not ramble or digress.

\end{itemize}

\end{document}