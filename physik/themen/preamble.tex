\documentclass[11pt]{article}

\usepackage[a4paper, margin=1in]{geometry}

\usepackage{amsmath}

\usepackage{amssymb}

\usepackage[american]{babel}

\usepackage[autostyle=true]{csquotes}

\usepackage{libertine}

\usepackage[libertine]{newtxmath}

\usepackage{tikz}

\usetikzlibrary{3d}

\usepackage{pgfplots}

\pgfplotsset{compat=1.10}

\usepackage{gensymb}

\usepackage{fancyhdr}

\usepackage{caption}

\usepackage{xfrac}

\usepackage{url}

\usepackage{cancel}

\usepackage[siunitx, european]{circuitikz}

\everymath{\displaystyle}

% Header / footer settings

\pagestyle{fancy}
\fancyhf{}
\renewcommand{\headrulewidth}{0.2mm}
\renewcommand{\footrulewidth}{0.2mm}
\fancyfoot[L]{Peter Goldsborough}
\fancyfoot[C]{\thepage}
\fancyfoot[R]{\today}

\fancypagestyle{plain}{%
	\fancyhf{}
	\renewcommand{\headrulewidth}{0mm}%
	\renewcommand{\footrulewidth}{0.2mm}%
	\fancyfoot[L]{Peter Goldsborough}
	\fancyfoot[C]{\thepage}
	\fancyfoot[R]{\today}
}


\setlength{\headheight}{15pt}

\setlength{\parindent}{0pt}

\addtolength{\parskip}{\baselineskip}


\newcommand{\heading}[1]{\fancyhead[C]{#1} \begin{center}\Huge \textbf{#1}\end{center}\par}

\newcommand{\sub}[1]{\vspace{\parskip}{\LARGE\textbf{#1}}}

\newcommand{\subsub}[1]{{\large \textbf{#1}}}

\newcommand{\subsubsub}[1]{\textbf{#1}}

\newcommand{\colvec}[1]{\begin{pmatrix}#1\end{pmatrix}}

\newcommand{\extrapar}{\par\vspace{\baselineskip}}

\newcommand{\bolditem}[1]{\item \textbf{#1}}

\newcommand{\titleitem}[1]{\bolditem{#1} \par}

\newcommand{\defas}{\, \dots \, \, }

\newcommand{\thus}[1][1]{\hspace{#1 cm} \Rightarrow \hspace{#1 cm}}

\newcommand{\mathtext}[2]{\hspace{#1 cm} \text{#2} \hspace{#1 cm}}

\newcommand{\AND}[1][1]{\mathtext{1}{and}}

\newcommand{\OR}[1][1]{\mathtext{1}{or}}

\newenvironment{plot}[1][h]
{
	\begin{figure}[#1]
		\centering
		\begin{tikzpicture}
}
{
		\end{tikzpicture}
	\end{figure}
}

\newenvironment{circuit}[1][h]
{
	\begin{figure}[#1!]
		\centering
		\begin{circuitikz}
}
{
	\end{circuitikz}
\end{figure}
}