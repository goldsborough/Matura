% Natural Phenomena

\documentclass[11pt]{article}

\usepackage[a4paper, margin=1in]{geometry}

\usepackage{amsmath}

\usepackage{amssymb}

\usepackage[german]{babel}

\usepackage[autostyle=true]{csquotes}

\usepackage{libertine}

\usepackage[libertine]{newtxmath}

\usepackage{tikz}

\usepackage{gensymb}

\usepackage{fancyhdr}

\usepackage{amsfonts}

\usepackage{pgfplots}

\pgfplotsset{compat=1.10}

\usepackage{multicol}

\usepackage{caption}

\usepackage{floatrow}

\everymath{\displaystyle}

% Header / footer settings

\pagestyle{fancy}
\fancyhf{}
\renewcommand{\headrulewidth}{0.2mm}
\fancyhead[C]{Funktionen}
\renewcommand{\footrulewidth}{0.2mm}
\fancyfoot[L]{Peter Goldsborough}
\fancyfoot[C]{\thepage}
\fancyfoot[R]{\today}

\fancypagestyle{plain}{%
	\fancyhf{}
	\renewcommand{\headrulewidth}{0mm}%
	\renewcommand{\footrulewidth}{0.2mm}%
	\fancyfoot[L]{Peter Goldsborough}
	\fancyfoot[C]{\thepage}
	\fancyfoot[R]{\today}
}


\setlength{\headheight}{15pt}

\setlength{\parindent}{0pt}

\addtolength{\parskip}{\baselineskip}


\newcommand{\overbar}[1]{\mkern 1.5mu\overline{\mkern-1.5mu#1\mkern-1.5mu}\mkern 1.5mu}

\newcommand{\heading}[1]{\begin{center}\Huge \textbf{#1}\end{center}\par}

\newcommand{\sub}[1]{\vspace{\parskip}{\LARGE\textbf{#1}}}

\newcommand{\subsub}[1]{{\Large \textbf{#1}}}

\newcommand{\subsubsub}[1]{\textbf{#1}}

\newcommand{\colvec}[1]{\begin{pmatrix}#1\end{pmatrix}}

\newcommand{\extrapar}{\par\vspace{\baselineskip}}

\newcommand{\zitat}[1]{\foreignquote{german}{#1}}

\newcommand{\bolditem}[1]{\item \textbf{#1}}

\newcommand{\titleitem}[1]{\bolditem{#1}\par}

\newcommand{\defas}{ \dots \,\,}

\begin{document}
\thispagestyle{plain}

\heading{Natural Phenomena}

\sub{Optical Phenomena}

\subsub{The Law of Reflection}

When a light ray hits a reflective surface (i.e. a mirror), the law of reflection mandates that the angle of the incident ray $\theta_i$ and the angle of the reflected ray $\theta_r$ are equal relative to a normal to the surface at the reflection point. The incident ray is the light ray stemming from the light source while the reflected ray is the light ray that is the product of reflection of the incident ray off of the reflective surface. 

\begin{plot}

	% Surface
	\draw (-4, 0) -- (4, 0) node [midway, below] {Reflective surface};

	% Normal
	\draw [dashed] (0, 4) -- (0, 0);

	% Incident ray
	\draw (135:4) node [above] {Incident Ray} -- (0, 0);

	% Angle of incidence
	\draw (0, 1.5) arc [radius=1.5cm, start angle=90, end angle=135];

	% Theta_i
	\draw (-0.5, 1) node {$\theta_i$};

	% Reflected ray
	\draw [->] (0, 0) -- (45:4) node [above] {Reflected Ray};

	% Angle of reflection
	\draw (0, 1.5) arc [radius=1.5cm, start angle=90, end angle=45];

	% Theta_r
	\draw (0.5, 1) node {$\theta_r$};

	% Law of Reflection
	\draw (0, -1) node {$\theta_i = \theta_r$};

\end{plot}

\subsub{The Law of Refraction}

The law of refraction pertains to the situation in which an incident light ray travelling from one transparent medium with a certain \emph{optical density}, in which its velocity is $c_1$, crosses the boundary to another transparent medium. Depending on the optical density of the second medium compared to the first, the incident light ray with experience a change in its velocity, which is then referred to as $c_2$. This will cause the light ray to be bent, such that there is a noticable difference between the angle of incidence $\theta_i$ and the angle of refraction $\theta_i$ relative to a normal to the boundary between the media at the point of incidence. The law of refraction says that the ratio of the sine of the incident $\sin \theta_i$ to that of the refracted angle $\sin \theta_r$ is equal to the ratio $c_1 : c_2$ of the velocity of the light ray in the two media. This law is often referred to as \emph{Snell's law}. The ratio $c_1 : c_2$ is referred to as the \emph{refractive index} $n_{1, 2}$ and is constant for any two given media and their optical densities: $$\frac{\sin \theta_i}{\sin \theta_r} = \frac{c_1}{c_2} = n_{1, 2}$$ The subscript numbers of the refractive index indicate which is the first medium (1) and which is the second (2). This direction can be swapped so that the refractive index $n_{2, 1}$ stands for the change in velocity from medium 2 to medium 1 (the opposite direction) by taking the reciprocal of any refractive index $n_{1, 2}$: $$n_{2, 1} = n_{1, 2}^{-1} = \frac{1}{n_{1, 2}}$$

\pagebreak

 \begin{plot}
	% Surface
	\draw (-4, 0) -- (4, 0);

	% First medium
	\draw (-4, 1) node {First medium};

	% Second medium
	\draw (-4, -1) node {Second medium};

	% Refractive index
	\draw (4.5, 0) node {$n_{1, 2}$};

	% Normal
	\draw [dashed] (0, 4) -- (0, -4);

	% Incident ray
	\draw (135:4) node [above] {Incident Ray} -- (0, 0);

	% Angle of incidence
	\draw (0, 1.5) arc [radius=1.5cm, start angle=90, end angle=135];

	% Theta_i
	\draw (-0.5, 1) node {$\theta_i$};

	% Reflected ray
	\draw [->] (0, 0) -- (302:4) node [below] {Reflected Ray};

	% Angle of reflection
	\draw (0, -1.5) arc [radius=1.5cm, start angle=270, end angle = 302];

	% Theta_r
	\draw (0.4, -1) node {$\theta_r$};

	% Law of Refraction
	\draw (0, -5)
	      node {$\frac{\sin \theta_i}{\sin \theta_r} = \frac{c_1}{c_2}$};
\end{plot}

 From this law, two general observations can be made:

 \begin{enumerate}
 	\item A light ray is bent toward the normal when entering an optically denser medium, such that $n_{1, 2} > 1$ and $c_1 > c_2$. This would be the case if a light ray would move from air into water.

 	\item A light ray is bent away from the normal when leaving an optically denser mediam, such that $n_{1, 2} < 1$ and $c_1 < c_2$. This occurs, for example, when light leaves glass and enters air.
 \end{enumerate}

 Often, not the refractive index between two media is given, but the \emph{absolute diffractive index} of a medium, e.g. $n_{glass} \approx 1.5$. This is the ratio of the speed of light in vacuum to the speed of light in that medium: $$n_{medium} = \frac{c}{c_{medium}}$$ The refractive index $n_{1, 2}$ between two media can then be found by dividing the absolute refractive index $n_2$ of the \emph{second} medium by the absolute refractive $n_1$ of the \emph{first} medium. When speaking of absolute diffractive indices, a greater value indicates a greater optical density (a reduced speed of light). $$n_1 = \frac{c}{c_1} \Rightarrow c_1 = \frac{c}{n_1} \hspace{1cm} \text{and} \hspace{1cm} n_2 = \frac{c}{c_2} \Rightarrow c_2 = \frac{c}{n_2}$$ $$\Downarrow$$ $$n_{1, 2} = \frac{c_1}{c_2} = \cfrac{\cfrac{c}{n_1}}{\cfrac{c}{n_2}} = \frac{n_2}{n_2}$$ $$\Downarrow$$ $$\frac{\sin \theta_i}{\sin \theta_r} = \frac{c_1}{c_2} = \frac{n_2}{n_1} = n_{1, 2}$$

\pagebreak

\subsub{Total Internal Reflection}

When travelling from a medium of higher absolute refractive index (optical density) to one of a lower refractive index, the angle of refraction is greater than the angle of incidence --- the light ray is bent away from the normal. However, \emph{refraction} can really only occur if the refracted angle is greater than 90 degrees, else it would not actually enter the second medium. The limiting angle of incidence above which no refracted ray can be formed is called the \emph{critical angle} $\theta_c$. For incident angles greater than $\theta_c$ the rays reflect back into the first medium. This is referred to as \emph{total internal reflection}. A formal definition for the critical angle reads:

\begin{displayquote}
	The critical angle $\theta_c$ is the angle of incidence for a ray crossing the boundary from a medium of higher optical density into one of lower optical density (or refractive index) at which the law of refraction predicts a refracted angle of $90\degree$. No refracted ray can form, thus the incident ray undergoes \emph{total internal reflection}.
\end{displayquote}

The critical angle can easily be found for any two given media, where medium 1 is the more optically dense, given that the angle of refraction at an angle of incidence equal to the critical angle is $90\degree$, whose sine is 1: $$\frac{\sin \theta_i}{\sin \theta_r} = n_{1, 2} \hspace{1cm} \text{or} \hspace{1cm} \frac{\sin \theta_i}{\sin \theta_r} = \frac{n_2}{n_1}$$ $$\Downarrow$$ $$\sin \theta_i = n_{1, 2} \hspace{1cm} \text{or} \hspace{1cm} \sin \theta_i = \frac{n_2}{n_1}$$ Thus it can be stated that the sine of the critical angle is equal to the refractive index of any two given media, where medium 1 is the more optically dense. For example, the critical angle for light travelling from glass ($n_g = 1.5$) into air would be: $$\sin \theta_c = \frac{n_2}{n_1} \Rightarrow \theta_c = \sin^{-1} \frac{1}{1.5} \approx 42\degree$$

 \begin{plot}
	% Surface
	\draw (-4, 0) -- (4, 0);

	% First medium
	\draw (-4, 1) node {First medium};

	% Second medium
	\draw (-4, -1) node {Second medium};

	% Refractive index
	\draw (4.5, 0) node {$n_{1, 2}$};

	% Normal
	\draw [dashed] (0, 4) -- (0, -4);

	% Incident ray
	\draw  (135:4) node [above] {Incident Ray} -- (0, 0);

	% Angle of incidence
	\draw (0, 1.5) arc [radius=1.5cm, start angle=90, end angle=135];

	% Theta_i
	\draw (-0.5, 1) node {$\theta_i$};

	% Reflected ray
	\draw [->, red] (0, 0) -- (3, 0) node [above, black] {Reflected Ray};

	% Angle of reflection
	\draw (0, -1.5) arc [radius=1.5cm, start angle=270, end angle = 360];

	% Theta_r
	\draw (0.6, -0.7) node {$\theta_c$};
\end{plot}

\pagebreak

One real-world application of the principle of total internal reflection are fiber optic cables. In such cables data is transmitted via light at very high speeds. There is an inner glass core with a refractive index $n_1$ and an outer glass cladding with a refractive index $n_2$. The glass core has a higher refractive index than the glass cladding, such that there is always total internal reflection (the angle of incidence is always greater than the critical angle):

\begin{plot}
	
	% Glass core + label
	\draw (0, 0) circle [radius=0.5cm]
	   -- (3, 0) node [right] {Glass core ($n_1$)};

	% Glass cladding
	\draw (0, 0) circle [radius=1.8cm];

	% Glass cladding label
	\draw (0, 1.5) -- (3, 1.5) node [right] {Glass cladding ($n_2$)};

	% Protective coating
	\draw (0, 0) circle [radius=2cm];

	% Protective coating label
	\draw (0, -1.9) -- (3, -1.9) [right] node {Protective coating};

	% Caption
	\draw (2, -3) node {A fiber optics cable};

\end{plot}

\subsub{Dispersion}

Dispersion is the separation or \emph{decomposition} of white light into its color components, the visible light spectrum: red, orange, yellow, green, blue and violet. Dispersion is method used for \emph{spectroscopy}, the analysis of light into its constituent wavelengths. It is commonly observed when white light is shown through a prism (a triangular body of glass). Such dispersion is possible because the different color components of the visible spectrum have different wavelengths and frequencies. Different frequencies result in a different degree of bending when moving between media of different optical densities. Their velocity changes by different amounts and thus the refractive indices of the individual color components differ from each other as well. The higher the frequency (and the shorter the wavelength) of a color component, the more it is refracted --- the more it bends towards the normal when moving from a medium of higher optical density to one of lower optical density. Thus, violet light is bent the most and red light the least, given that the wavelength of violet light, around 400 nm, is much shorter than that of red light, around 700 nm. The result of dispersion through a prism is a continuous (band) spectrum of light.

\begin{plot}

	% Prism
	\draw (0, 0) -- (6, 0) node [midway, above] {Prism} -- (3, 5) -- (0, 0);

	% First normal
	\draw [dashed] (0.5, 3.1) -- (1.5, 2.5) -- (2.5, 1.9);

	% White light
	\draw (1.5, 2.5) -- +(200:4) 
	      node [pos=0.7, rotate=20, above] {White light};

	\foreach \c/\x in {red/4.5, orange/4.55, yellow/4.6,
	                   green/4.65, blue/4.7, violet/4.75}
	{
		\draw [\c, line width=1.2pt] (1.5, 2.5) -- (\x, {(-5/3) * \x + 10});

		\draw [\c, line width=2pt] (\x, {(-5/3) * \x + 10})
		                        -- ({\x+4}, {0.4 * (25 - \x^2)});
	}

\end{plot}

The angle of incidence $\alpha_1$ on the left side of the prism (entry point) is greater than the angle of refraction $\beta_1$ at that point. The rays are bent towards the normal because they are entering a medium of higher refractive index and optical density (glass) from one of lower optical density (air). On the far (right) side of the prism (exit point), however, the opposite is the case. The incident angle $\alpha_2$ is less than the angle of refraction $\beta_2$. The light rays are now entering a medium of less optical density than the one they came from, thus their velocity increases upon exit and they are bent away from the normal. On the left side, $n_{1, 2}$ is therefore greater one, as $c_1 > c_2$, and on the right side $n_{1, 2}$ it is less than one, as $c_1 < c_2$.

It should be mentioned that dispersion of white light works best with triangular prisms because the angles of the normal to the surface at the point of refraction at either side of the prism causes the light to be bent away from each other twice. If a rectangle were to be used, dispersion could not effectively take place because, after being dispersed upon entering the rectangle (where different frequencies bend differently, away from each other), the individual color components of the white light would bend back towards each other when leaving the rectangle at the opposite side. The reason for this is that the normal would be at the same angle to the surface both times such that the different frequencies would bend towards the normal upon entering and away from the normal upon exit. There would be no net difference between the light before and after going through the rectangular prism.

\subsub{Rainbows}

A perfect example of a natural phenomenon involving the dispersion and decomposition of white light into its color components is a rainbow. A rainbow is formed when light from the sun, which must be behind the observer, hits tiny water or rain droplets dispended in the air. These water droplets, having a different optical density than air, act as prisms. Some of the light hitting a water droplet reflects off of it, but most refracts into the droplet bending towards the normal, given that water has a higher refractive index than air. As was discussed, the different color components of the visible spectrum of light have different frequencies and wavelengths, causing each to refract at different angles. Higher frequencies and shorter wavelengths cause more refraction of the color components. Therefore, violet and blue light bend the most while red and orange bend the least (towards the normal). In the case of the formation of a rainbow, the refracted light rays reach the other end of the rain droplet at an angle that is greater than the critical angle of water, around $48.6 \degree$. Thus, there is total internal reflection, causing the light rays not to refract out of the rain droplet, but to reflect and thus move towards another point within the droplet. At this point, the light rays do exit the droplet and refract and disperse again, with higher frequency light components bending away furthest from the normal. The dispersion of white light is what is seen as a rainbow.

\begin{plot}

	\coordinate (A) at (-2, {sqrt(3^2 - 2^2)});

	% Droplet
	\draw (0, 0) circle [radius=3cm];

	% Droplet label
	\draw (0, -3.5) node {Water droplet};

	% White light
	\draw (A) -- +(-3, 0)
	      node [above, pos=0.7] {White light};

	% Normal
	\draw [dashed] (-2, {sqrt(3^2 - 2^2)})+(135:1) -- +(315:1);

	% Colors
	\foreach \c/\x/\y in {red/2.75/-2.4, orange/2.8/-2, yellow/2.85/-1.4,
	                   green/2.9/-1, blue/2.95/-0.4, violet/2.98/0}
	{
		\draw [\c, line width=1pt] (A) 
		   -- (\x, {sqrt(3^2 - (\x)^2)}) 
		   -- (-\x, {-sqrt(3^2 - (\x)^2)});

		\draw [\c, line width=2.5pt] 
		      (-\x, {-sqrt(3^2 - (\x)^2)})
		   -- (-5, \y);
	}

\end{plot}

\pagebreak

\subsub{Mirages}

Another optical phenomenon that is found in nature are \emph{mirages}. Mirages are the result of the fact that the temperature of air influences its refractive index. The greater the temperature of the air, the lower is its optical density and thus its refractive index --- the faster the speed of the light rays. On the other hand, colder air has a greater refractive index, thus light is bent towards the normal. In effect, light is bent when it moves through air that varies in temperature. Depending on whether hot air is at the bottom or at the top (of the surface of the earth), either an \emph{inferior} or a \emph{superior} mirage will be created.

\subsubsub{Inferior Mirage}

An inferior mirage is created when warm or hot air is at the bottom and cold air at the top. It should be mentioned that these temperatures increase in their direction, i.e. the air gets warmer and warmer towards the surface and colder and colder away from it. Given that warm air has a lower refractive index and optical density, it will cause light rays to bend away from the normal when the incident ray comes from air of lower temperature. Therefore, when light from the sun reflects off of an object downwards at an observer into gradually increasing temperatures, it will continuously bend upwards until it will have made a full change in direction. After that point, the increasingly cold air in the upward direction will cause the light to bend towards the normal, given the greater optical density, thus continuing the circular motion. A mirage is created when the light consequently reaches the eye of the observer from a position below him. This will cause our eyes and brain to think that the light really came from a position below, following the angle of incidence on the eye in a straight line and into the earth. This is especially common with light coming from the sky (the blue component of white light scattered by air molecules) which will produce the optical illusion of water on the ground in this way.

\begin{plot}
	
	% Surface
	\draw (0, 0) node [left] {Surface} -- (8, 0);

	% Temperature direction
	\draw [->] (8.5, 3) -- +(0, -2.8)
	      node [midway, right] {Increasing $\degree C$};

	% Eye of the Observer, part 1
	\draw (1, 2.5) -- ++(202.5:1) -- +(337.5:1);

	% Eye of the Observer, part 2
	\draw (0.8, 1.8) arc [radius=0.8, start angle=-22.5, end angle=22.5];

	% Eye of the Observer, part 3
	\draw (0.75, 2.1) circle [radius=3pt];

	% Eye of the Observer, label
	\draw (0.5, 1.2) node {Observer};

	% Object
	\draw (6, 1.5) -- +(1, 0) -- +(1, 1) -- +(0, 1) -- +(0, 0);

	% Object label
	\draw (6.5, 1) node {Real object};

	% Light ray from sun
	\draw (4, 3) node [above] {Light from sun} -- (6, 2);

	% Bent light
	\draw (6, 2) .. controls (3.5, 0) .. (1, 2);

	% Apparent path of light
	\draw [dashed] (1, 2) -- +(-38:6.35) 
	      node [pos=0.8, left] {Apparent path of light \,\,\,};

	% Mirage
	\draw [dashed] (6, -2.3) -- +(1, 0) -- +(1, 1) -- +(0, 1) -- +(0, 0);

	% Mirage label
	\draw (8, -1.8) node {Mirage};

\end{plot}

\pagebreak

\subsubsub{Superior Mirage}

In case of a superior mirage, the increasingly cold air is at the bottom and the warmer at the top. When light reflects off of an object downwards towards an observer, it is thus bent towards the normal given the increasing optical density and refractive index of the air. The apparent path of light is in this case above the object, where the object from which the light originally reflected appears as the optical phenomenon that is a \emph{superior mirage}.

\begin{plot}
	
	% Surface
	\draw (0, 0) node [left] {Surface} -- (8, 0);

	% Temperature direction
	\draw [->] (8.5, 3) -- +(0, -2.8)
	      node [midway, right] {Decreasing $\degree C$};

	% Eye of the Observer, part 1
	\draw (1, 1) -- ++(202.5:1) -- +(337.5:1);

	% Eye of the Observer, part 2
	\draw (0.8, 0.3) arc [radius=0.8, start angle=-22.5, end angle=22.5];

	% Eye of the Observer, part 3
	\draw (0.75, 0.6) circle [radius=3pt];

	% Eye of the Observer, label
	\draw (0.5, 1.3) node {Observer};

	% Object
	\draw (6, 2.3) -- +(1, 0) -- +(1, 1) -- +(0, 1) -- +(0, 0);

	% Object label
	\draw (6.5, 1.8) node {Real object};

	% Light ray from sun
	\draw (2, 4) node [above] {Light from sun} -- (6, 2.8);

	% Bent light
	\draw (6, 2.8) .. controls (5, 3) and (3, 2.5) .. (1, 0.6);

	% Apparent path of light
	\draw [dashed] (1, 0.6) -- +(40:6.5) 
	      node [pos=0.5, left] {Apparent path of light \,\,\,};

	% Mirage
	\draw [dashed] (6, 4.2) -- +(1, 0) -- +(1, 1) -- +(0, 1) -- +(0, 0);

	% Mirage label
	\draw (7.8, 4.7) node {Mirage};

\end{plot}

\sub{Lightning}

Lightning is a perfect example of the effects of electric fields in nature. The pre-conditions for and the mechanism of lightning strikes can be divided into four separate stages or phases, investigated and explained below. Beforehand, however, the scenario and environment in which lightning occurs should be examined. The birthplace of lightning are clouds in the sky. Clouds are regions of condensed water vapour in the atmosphere, where evaporating water from the surface of the earth accumulates. Clouds are found at high altitudes, where the temperature is well below zero. As a result, there may be condensed water clusters and water vapour in the cloud, but also frozen ice particles. Given the right conditions, lightning will form via the steps discussed in the following paragraphs.

\subsub{Phase 1: Static Charge Buildup}

At the very beginning of any lightning strike, there must first be a \emph{static charge buildup} in the cloud. The adjective \emph{static} here refers to a gathering of electric charge in a non-conducting material or object, i.e. with no flow of current. Thus, before a lightning strike occurs, there is a charge separation --- also referred to as \emph{polarization} --- in the cloud, whereby the upper portion of the cloud becomes positively charged while the lower region takes on a negative charge. One reason for this is that evaporating water molecules collide with atoms already present in the cloud, thereby knocking away and freeing electrons from the outer energy shells of atoms, thereby causing a charge separation. In any case, as a result of the polarization of the cloud, a strong electric field is established between the positive and the negative end (from $+$ to $-$).

\begin{plot}

	\node [cloud, draw,cloud puffs=10,cloud puff arc=120, aspect=2, inner ysep=1em] {};

	% Charges and field
	\foreach \i in {-0.5, 0, 0.5}
	{
		\draw [red] (\i, 0.35) node {$+$};

		\draw [->] (\i, 0.15) -- (\i, -0.2);

		\draw [blue] (\i, -0.35) node {$-$};
	}

\end{plot}

\subsub{Phase 2: Ionization of the Surroundings}

After charge separation and polarization of the cloud, the strong electric field established between the now positively charged top and the negatively charged bottom of the cloud begins to ionize the surrounding air and gas molecules. As a result, the air around the cloud becomes positively charged. We can now refer to the surrounding air as \emph{plasma} --- ionized air or gas --- which has an important property: it is conductive, i.e. charge and current may flow through it. This is especially important in the regions below the cloud, nearest to the surface of the earth.

\begin{plot}

	\node [cloud, draw,cloud puffs=10,cloud puff arc=120, aspect=2, inner ysep=1em] {};

	% Charges and field
	\foreach \i in {-0.5, 0, 0.5}
	{
		\draw [red] (\i, 0.35) node {$+$};

		\draw [->] (\i, 0.15) -- (\i, -0.2);

		\draw [blue] (\i, -0.35) node {$-$};
	}

	% Plasma
	\foreach \x in {-1, -0.5, ..., 1}
	{
		\foreach \y in {-1, -1.5}
		{
			\draw [red] (\x, \y) node {$+$};
		}
	}

	% Plasma label
	\draw [<-] (1.5, -1.25) -- ++(0.5, 0) node [right] {Plasma};

\end{plot}

\subsub{Phase 3: Formation of Step-Leader and Streamer}

With the presence of plasma below the cloud, there is now a conductive path for the negatively charged particles to travel through. Thus, excess electrons from the lower region of the cloud begin to travel through the now conductive air in a zig-zag pattern as what is known as the \emph{step-leader}. As the electrons come closer to the surface of the earth on their path through the conductive plasma, they repell the electrons present in the ground. Consequently, the surface of the earth becomes \emph{positively charged}. Positive charges then begin to migrate upwards through buildings, trees or people and move in the direction of the approaching step-leader. This flow of positive charges is referred to as the \emph{streamer}.

\begin{plot}

	\node [cloud, draw,cloud puffs=10,cloud puff arc=120, aspect=2, inner ysep=1em] {};

	% Charges and field
	\foreach \i in {-0.5, 0, 0.5}
	{
		\draw [red] (\i, 0.35) node {$+$};

		\draw [->] (\i, 0.15) -- (\i, -0.2);

		\draw [blue] (\i, -0.35) node {$-$};
	}

	% Plasma
	\foreach \x in {-1, -0.5, 0.5, 1}
	{
		\foreach \y in {-1, -1.5}
		{
			\draw [red] (\x, \y) node {$+$};
		}
	}

	% Plasma label
	\draw [<-] (1.5, -1.25) -- ++(0.5, 0) node [right] {Plasma};

	% Step leader
	\draw [blue] (0, -0.7) \foreach \i in {1, 2}
	{
		-- ++(250:0.5)

		\ifnum\i<2
			 -- ++(0.2, 0)
		\fi
	};

	% Step leader label
	\draw [<-] (-0.4, -1.25) -- ++(-0.8, 0) node [left] {Step Leader};

	% Ground
	\draw (-2, -3) -- (2, -3);

	% Ground charges
	\foreach \x in {-2, -1.5, ..., 2}
	{
		\draw [red] (\x, -3.3) node {$+$};
	}

	% Streamer
	\draw [red] (-0.7, -3) \foreach \i in {1, 2}
	{
		-- ++(70:0.5)

		\ifnum\i<2
			 -- ++(0.2, 0)
		\fi
	};

	% Streamer label
	\draw [<-] (-1, -2.5) -- ++(-0.7, 0) node [left] {Streamer};

\end{plot}

\subsub{Phase 4: Lightning}

When the flow of negative charges in the step leader meets the flow of positive charges in the streamer, a complete conductive path is established between the region of high electron concentration in the cloud and the positively charged surface of the ground. There is now a difference in electric potential between the cloud and the ground, such that electric current can and will flow through the conductive path from the negative region of the cloud to the positive surface of the earth. Current always involves an increase in thermal energy, i.e. generation of heat. Thus, as current flows through the conductive path, it causes the air around it to expand. This is what we hear as thunder.

\begin{plot}

	\node [cloud, draw,cloud puffs=10,cloud puff arc=120, aspect=2, inner ysep=1em] {};

	% Charges and field
	\foreach \i in {-0.5, 0, 0.5}
	{
		\draw [red] (\i, 0.35) node {$+$};

		\draw [->] (\i, 0.15) -- (\i, -0.2);

		\draw [blue] (\i, -0.35) node {$-$};
	}

	% Plasma
	\foreach \x in {-1, -0.5, 0.5, 1}
	{
		\foreach \y in {-1, -1.5}
		{
			\draw [red] (\x, \y) node {$+$};
		}
	}

	% Plasma label
	\draw [<-] (1.5, -1.25) -- ++(0.5, 0) node [right] {Plasma};

	% Bolt color
	\definecolor{bolt}{RGB}{248,202,0}

	% Lightning
	\draw [bolt, thick] (0, -0.7) \foreach \i in {1, 2, 3, 4, 5}
	{
		-- ++(250:0.5)

		\ifnum\i<5
			 -- ++(0.2, 0)
		\fi
	};

	% Lightning label
	\draw [<-] (-0.4, -2.2) -- ++(-0.8, 0) node [left] {Lightning};

	% Ground
	\draw (-2, -3) -- (2, -3);

	% Ground charges
	\foreach \x in {-2, -1.5, ..., 2}
	{
		\draw [red] (\x, -3.3) node {$+$};
	}

\end{plot}

\pagebreak

\subsub{Protection from Lightning}

To protect houses and people from the highly dangerous effects of a lightning strike (fire or death), lightning rods are used. The concept behind them is fairly simple. Instead of letting the high current of a lightning bolt flow directly through a house, it is diverted by a lightning rod. Such a lightning rod is simply a wire at the highest point of the house (on the roof), which leads any lightning striking it to ground (earth).

\sub{Earth Magnetic Field}

The earth's magnetic field is very similar to that of a powerful bar magnet.The source of this magnetic field is not entirely understood, but one of the most prominent theories regarding the generation of the earth's magnetic field is called the \emph{dynamo theory}. This theory states that the magnetic field in and around the earth is caused by heat convection currents in the core of the earth. These heat convection currents have electrical current flowing through them, which in turn generates the earth's magnetic field according to the laws of electromagnetism. A very important fact to mention with respect to the earth's magnetic field is that the magnetic poles $N_m, S_m$ do not lie at the same locations $N_g, S_g$ as the earth's geographical poles. Rather, the center of the earth's magnetic field is tilted at an angle of approximately $11\degree$ from the earth's rotational (spin) axis.

\begin{plot}

	% Earth
	\draw (0, 0) circle [radius=1.5cm];

	% Axis of rotation
	\draw [magenta]
	      (270:2) node [black, below] {$S_g$} 
	   -- (90:2) node [black, above] {$N_g$};

	% Magnetic axis
	\draw [blue]
	      (281:2) node [black, below right] {$S_m$} 
	   -- (101:2) node [black, above left] {$N_m$};

	% Points
	\foreach \a in {0, 11}
	{
		\draw [fill=black] ({90 + \a}:1.5) circle [radius=1.2pt];

		\draw [fill=black] ({270 + \a}:1.5) circle [radius=1.2pt];
	}

	% Poles box at core
	\begin{scope}[rotate=11]

		% North pole box
		\draw [fill=red, red]
		      (-0.3, 0.5)
		 -- ++(0.6, 0) -- ++(0, -0.5) -- ++(-0.6, 0) -- ++(0, 0.5);

		% North pole label
		\draw (0, 0.25) node {N};

		% South pole box
		\draw [fill=cyan, cyan]
		      (-0.3, -0.5)
		 -- ++(0.6, 0) -- ++(0, 0.5) -- ++(-0.6, 0) -- ++(0, -0.5);

		% South pole label
		\draw (0, -0.25) node {S};

	\end{scope}

\end{plot}

Another interesting property of the earth's magnetic field is that there exists a certain characteristic area in space around it, in which charged particles interact with the earth's magnetic field. This region around the earth is referred to as the \emph{magnetosphere}. Especially solar wind, a stream of charged particles, i.e. plasma (ionized air or gas), from the corona (outer atmospheric layer) of the sun will interact with the earth's magnetic field in the magnetosphere.

\subsub{The Van Allen Radiation Belts}

The Van Allen radiation belts are two layers around the earth in which energetic charged particles (radiation) are trapped and held in place by the earth's magnetic field. These belts are found at an altitude of 1000 to 60000 kilometers above the earth's surface and are thus located in the inner region of the earth's magnetosphere. The charged particles found in these belts --- electrons and protons --- originate mostly from solar wind, while cosmic rays may transport other nuclei, such as alpha particles. The outer of the two Van Allen belts contains mostly negative particles, while highly energetic protons dominate in the inner belt. The force that traps particles in these regions of space is the Lorentz force --- the force exerted on moving charges in a magnetic field. Van Allen belt radiation can be very harmful to satellites, spacecrafts and other electronics due to their high-energy and ionizing property.

\sub{Polar Lights}

Polar lights, also referred to as \emph{Auroras}, are color phenomena in the sky most commonly experienced near the earth's magnetic poles. The source of these polar lights is the interaction between ionizing radiation from the sun, known as \emph{solar wind}, and certain molecules in the earth's atmosphere. This solar wind is a stream of plasma (ionized air or gas) released from the upper atmosphere of the sun --- the \emph{corona} --- and is responsible for exciting atoms in the earth's atmosphere. These atoms, in turn, emit energy in form of light photons and thereby create polar lights. For a complete discussion of the topic, it must first be examined what solar wind is, wherein its origin lies and how it influences the creation of polar lights.

\subsub{Solar Wind}

When Hydrogen atoms collide with high velocity and immense energy in the core of the sun, at a temperature of more than 15 million degrees Celsius, nuclear reactions take place under conditions of high pressure to form Helium. These reactions produce energy which radiates away from the sun's core towards its outermost layer --- the \emph{convection zone} --- where it is subsequently stored in (heat) \emph{convection cells}. These convection cells generate electrical currents, which in turn produce magnetic fields that may interact with the ionized gas in the sun's upper atmosphere, also known as its \emph{corona}. Usually, the gravitational force exerted by the sun
prevents this plasma from leaving the corona. However, occasional bursts of energy from the sun’s core may create strong enough magnetic fields in and around the convection cells to overpower the sun’s gravity and transport the plasma out of the corona and into space. These bursts of plasma leaving the sun are what is then referred to as \emph{solar wind}. 

When solar wind propagates through space towards the earth, it does not impact our planet directly or without any resistance. Rather, it interacts with the earth’s magnetosphere. The magnetosphere is the area of space where charged particles are controlled by the earth’s magnetic field. As the plasma reaches the magnetosphere's bow shock and interacts with the earth's magnetic field, a fraction of it is deflected towards the two poles and pulled into the earth’s atmosphere.

\subsub{Auroral Mechanism}

When solar wind enters the earth’s atmosphere (80km above sea level and higher) its charged particles collide with the oxygen and nitrogen atoms in the air. These collision excite the atoms and thereafter lead to the emission of a quantum of energy in the form of light (photons) when electrons inside the excited atoms transition to a lower energy state from their excited states. The colors of the light waves emitted depend on the atoms involved and the altitude of the collisions.

\begin{itemize}
	\item Oxygen emits \dots

		\begin{itemize}

			\bolditem{Red light} at altitudes of circa 240km above sea level and higher. This wavelength of light is very rare, as it takes up to two minutes for an oxygen atom to emit red light. During this relatively long period of time the chance of another atmospheric atom or a charged particle from the solar wind colliding and thus transferring new energy to the atom in the process of emission is very high. If an emission is indeed achived, however, this most often occurs at higher altitudes, as the atmospheric density and abundance of oxygen is lower with increasing distance from sea level. However, a reduced abundance of oxygen at high altitudes also means that there are fewer atoms available to emit light in the first place.

			\bolditem{Green light} at heights of \emph{up to} 240 km above sea level. It takes an oxygen atom approximately three quarters of a second to emit to emit green light. Therefore, the chance of a green light photon to be emitted is a lot higher than for red light. This the reason why green light is the most common color observed in polar lights.

		\end{itemize}

	\item Nitrogen emits \dots

		\begin{itemize}

			\bolditem{Blue light} at altitude levels of 100km and less. This color is emitted when an ionized nitrogen atoms regains an electron. Emission of light at this frequency and wavelength is very common during high levels of solar activity.

			\bolditem{Dark red light} at 100km sea level and higher. 

		\end{itemize}

	\item Other colors such as yellow or pink are usually mixtures of any of the above colors.

\end{itemize}

\end{document}