% Astrophysics

\documentclass[11pt]{article}

\usepackage[german]{babel}

\usepackage[autostyle=true]{csquotes}

\usepackage[a4paper, margin=1in]{geometry}

\usepackage{libertine}

\setlength{\parindent}{0pt}

\addtolength{\parskip}{\baselineskip}

\newcommand{\extrapar}{\par\vspace{\baselineskip}}

\newcommand{\heading}[1]{\begin{center}\Huge \textbf{#1} \end{center}}

\newcommand{\sub}[1]{{\Large \textbf{#1}}\par}

\newcommand{\subsub}[1]{{\large \textbf{#1}}\par}

\newcommand{\zitat}[1]{\emph{\foreignquote{german}{#1}}}

\newcommand{\titleitem}[1]{\item \textbf{#1} \par}

\begin{document}
\thispagestyle{plain}

\heading{Astrophysics}

\sub{Models of the Universe}

\subsub{Aristotelian System}

One of the first theories of the universe was developed by Aristotle and his pupils. Aristotle thought the universe to be finite in size, geocentric and entirely based on perfect mathematical forms. Thus, the planets moved in perfectly circular orbits around the earth, with the stars fixed to the outermost spheres.  

\begin{figure}[h!]
	\centering
	\begin{tikzpicture}

		% Earth
		\draw [fill=blue] (0, 0) circle [radius=3pt];

		% The planets
		\foreach \r/\x/\c in {0.5/0.5/red, 1/0.2/cyan, 1.5/-1/orange,
		                      2/-2/magenta, 2.5/2.5/teal}
		{
			\draw (0, 0) circle [radius=\r cm];

			\draw [fill=\c]
			      (\x, {(-1)^(\r*2)*sqrt(\r^2 - (\x)^2)}) circle [radius=3pt];
		}

		% The stars
		\draw (0, 0) circle [radius=3cm];

		\foreach \x in {-3, -2, ..., 3}
		{
			\draw [fill=yellow]
			      (\x, {sqrt(9 - (\x)^2)}) circle [radius=2pt];

			\draw [fill=yellow]
			      (-\x, {-sqrt(9 - (\x)^2)}) circle [radius=2pt];
		}

		% The legend
		\draw [fill=blue] (5, 0.5)
		      circle [radius=3pt] node [right] {\, \, Earth};

		\draw [fill=red] (5, 0)
		      circle [radius=3pt] node [right] {\, \, Planets};

		\draw [fill=yellow] (5, -0.5)
		      circle [radius=2pt] node [right] {\, \, Stars};

	\end{tikzpicture}
\end{figure}

\begin{itemize}
	\item Finite in size
	\item Geocentric -- earth at the center, all planets orbit around it
	\item Based on perfect mathematical forms (the orbits are circles)
	\item Stars are fixed to the outer sphere
\end{itemize}

\subsub{Ptolemaic System}

In the 2nd century AD the Alexandrian scientist Ptolemy proposed major changes to the Aristotelian model to incorporate the latest observations of his time. For one, it was found that the distance of planets relative to earth (as seen by their brightness) varies with time, which made perfect uniform circles impossible. Therefore, the first change introduced by Ptolemy was to displace the centers of the planets' spheres slightly from the center of the Earth. The center of these off-center spheres, here called \emph{deferents}, is called the \emph{eccentric}. However, Ptolemy also noticed that the speed (angular velocity / rate) of the planets was not uniform, but varies as the planets seem to change direction at certain times. This change in direction (often seen as backwards motion) is called \emph{retrograde motion}. To account for the retrogade motion, it was proposed that the planets move at uniform speed around a small circle --- called an \emph{epicycle} --- whose center moves at uniform speed around the deferent. Yet, Ptolemy observed that the center of the epicycle moved at different speeds when it was closer to earth than when it was further way. To account for this, Ptolemy created another point in space --- the \emph{equant} --- around which a planet moves at uniform speed. Now, he could say that a planet moves in a circular motion around the eccentric (the center of the deferent) and with uniform speed around the equant. The position of the equant is at the same distance as that between earth and the eccentric, but at the opposite side of earth (relative to the eccentric).

\begin{figure}[h!]
	\centering
	\begin{tikzpicture}

		% Earth
		\draw [fill=blue] (-0.6, 0) circle [radius=3pt];

		% Center of Deferent
		\draw [fill=black] (0, 0) circle [radius=2pt];

		% Equant
		\draw [fill=red] (0.6, 0) circle [radius=2pt];

		% Deferent
		\draw (0, 0) circle [radius=2cm];

		% Epicycle
		\draw (-1.5, {sqrt(4 - (1.5)^2)}) circle [radius=0.5cm];

		% Planet
		\draw [fill=orange] (-1, {sqrt(4 - (1.5)^2)}) circle [radius=3pt];

		% Legend

		\draw (-2.8, 1.5) node {Epicycle};

		\draw (-3, 0) node {Deferent};

		\draw [fill=blue] (3, 1)
		      circle [radius=3pt] node [right] {\, \, Earth};

		\draw [fill=orange] (3, 0.5)
		      circle [radius=3pt] node [right] {\, \, Planet};

		\draw [fill=red] (3, -0.05)
		      circle [radius=2pt] node [right] {\, \, Equant};

		\draw [fill=black] (3, -0.5)
		      circle [radius=2pt] node [right] {\, \, Center of Deferent};

	\end{tikzpicture}
\end{figure}

\begin{itemize}
	\item Geocentric and finite in size
	\item The centers of the circular orbits, the \emph{deferents}, are called the \emph{eccentrics} and are off-center (the center being earth)
	\item Planets move around small circles, called \emph{epicycles}, which in turn move on the deferent (the rotating sphere)
	\item The centers of the epicycles move circularly around the deferent, but with constant angular velocity around the \emph{equant}
\end{itemize}

{\small Resource: {\footnotesize \url{http://www.mathpages.com/home/kmath639/kmath639.htm}}}

\subsub{Copernican System}

Ptolemy's desperate attempts to fit the observations of the time into a system of perfect mathematical circular shapes turned into a very complex system. Copernicus, around 1500, thought that by moving from a geocentric system to a \emph{heliocentric} one, from having the earth at the center of the universe to the sun being at the center, the complexities could be dissovled. He therefore proposed that all planets orbit around the sun, while the moon orbits around the earth in a circular orbit. Again, all orbital shperes were based on perfect euclidian circles.

\begin{figure}[h!]
	\centering
	\begin{tikzpicture}

		% Sun
		\draw [fill=yellow] (0, 0) circle [radius=7pt];

		% The planets
		\foreach \r/\x/\c in {1/1/red, 1.5/0.2/cyan, 2/-1/blue,
		                      2.5/-2/magenta, 3/2.5/orange}
		{
			\draw (0, 0) circle [radius=\r cm];

			\draw [fill=\c]
			      (\x, {(-1)^(\r*2)*sqrt(\r^2 - (\x)^2)}) circle [radius=3pt];
		}

		% Moon
		\draw [fill=gray] (-0.825, {sqrt(0.35^2 - 0.175^2) + sqrt(3)})
		      circle [radius=2pt];

		% Moon orbit
		\draw (-1, {sqrt(3)}) circle [radius=0.35cm];

		% The legend
		\draw [fill=yellow] (5, 1)
		      circle [radius=7pt] node [right] {\, \, Sun};

		\draw [fill=red] (5, 0.45)
		      circle [radius=3pt] node [right] {\, \, Other Planets};

		\draw [fill=blue] (5, 0)
		      circle [radius=3pt] node [right] {\, \, Earth};

		\draw [fill=gray] (5, -0.5)
		      circle [radius=2pt] node [right] {\, \, Moon};

	\end{tikzpicture}
\end{figure}

\begin{itemize}
	\item \textbf{Heliocentric}
	\item Finite in size
	\item Based on perfect mathematical forms (circles)
\end{itemize}

\pagebreak

\subsub{Kepler's System}

Around 1600 the Imperial Mathematician of the Holy Roman Empire, Johannes Kepler, made import astrophysical observations and propositions. He accepted Copernicus' heliocentric model, but found that such a model was only plausible if the planets moved in elliptical rather than circular shapes. He therefore proposed three laws of planetary motion, commonly known as \emph{Kepler's Laws}:

\begin{enumerate}
	\titleitem{Law of Ellipses / Orbits}

	Planets move in elliptical orbits with the Sun as one focus

	\titleitem{Law of Areas}

	A straight line from the Sun to the planet sweeps out equal areas in equal times as it orbits the Sun with a higher velocity closer to the Sun (wider angle / more orbital path covered, shorter side) and a lower velocity further away from it (smaller angle / less orbital path covered, longer side).

	\titleitem{Law of Harmonies}

	The orbital periods scale with ellipsis size, such that the cube of the planet's mean distance $r$ from the Sun, divided by the square of the orbital period $T$, is the same for all planets: $\frac{r^3}{T^2}$
\end{enumerate}

\begin{figure}[h!]
	\centering
	\begin{tikzpicture} 
	[
		declare function = 
		{
			ell(\x) = sqrt(\b^2 * (1 - ((\x)^2 / \a^2)); 
			f(\x) = 0.056799641 * \x + 0.090879425;
			g(\x) = (ell(-3.5) / -1.1) * \x + 2.4 * (ell(-3.5) / -1.1);
		} 
	]

		\newcount\a
		\a4\relax

		\newcount\b
		\b2\relax

		% Elliptical orbit
		\draw (0, 0) circle [x radius=\a, y radius = \b];

		% Sun
		\draw [fill=yellow] (-2, 0) circle [radius=0.4cm];

		% Label Sun
		\draw (-2, -0.5) node [below] {Sun};

	    % Planet position 1 (slow)
	    \draw [fill=blue] (-1.6, 0) -- (3.95, {-ell(3.95)}) circle [radius=3pt];

	    % Planet position 2 (slow)
	    \draw [fill=blue] (-1.6, 0) -- (3.95, {ell(3.95)}) circle [radius=3pt];

	    % Fill area
	    \foreach \x in {-1.6, -1.4, ..., 3.95}
	    {
	    	\draw [blue] (\x, {f(\x)}) -- (\x, {-f(\x)});
	    }

	    % Slow label
	    \draw (4.8, 0) node {(Slower)};

	    % Planet position 3 (faster)
	    \draw [fill=red] (-2.4, 0) -- (-3.5, {ell(-3.5)}) circle [radius=3pt];

	    % Planet position 4 (faster)
	    \draw [fill=red] (-2.4, 0) -- (-3.5, {-ell(-3.5)}) circle [radius=3pt];

	    % Fill area, first part
	    \foreach \x in {-3.5, -3.3, ..., -2.4}
	    {
	    	\draw [red] (\x, {g(\x)}) -- (\x, {-g(\x)});
	    }

	    % Fill area, second part
	    \foreach \x in {-3.95, -3.8, -3.65}
	    {
	    	\draw [red] (\x, {ell(\x)}) -- (\x, {-ell(\x)});
	    }

	    % Faster label
	    \draw (-5, 0) node {(Faster)};

	\end{tikzpicture}
\end{figure}

\subsub{Galileo Galilei}

Galileo Galileo was an Italian scientist from the Renaissance period who made import contributions to astrophysics, as he was the first to use a telescope for systematic astronomical analysis. His most import findings include:

\begin{itemize}
	\item Craters and mountains on the moon
	\item The moons of Jupiter and the fact that they, as well as all other astronomical bodies, are not perfect spheres
	\item That the earth is not the only center for rotation as proposed by geocentric theories.
\end{itemize}

\pagebreak

\subsub{Isaac Newton}

Isaac Newton built on Galileo Galilei as well as Johannes Kepler's work and, for the first time, found a mathematical basis for their hypotheses. He did so using by defining and using his concept of gravity, an attractive force exerted by all masses on all other masses, and his corresponding laws of motion and gravitation. He found that the attractive force between two masses -- gravity -- could be found using the formula known as the \emph{universal law of gravitation}. It states that the gravitational force acting on any two objects is directly proportional to their mass and indirectly proportional to the distance $r$ between the two objects, squared: $$F = G \cdot \frac{m_1 \cdot m_2}{r^2}$$

$m_1, m_2$ \defas The two relevant masses on which and by which the attractive force is exerted

$r$ \defas The distance between the two masses

$G$ \defas The universal constant of gravitation = $6.67 \cdot 10^{-11} \, Nm^2 kg^{-1}$

$F$ \defas The attractive force between the two masses $m_1$ and $m_2$

\vspace{\parskip}

\begin{figure}[h!]
	\centering
	\begin{tikzpicture}[scale=1.5]

		% The distance
		\draw [<->] (-1.975, 1) -- (1.975, 1)
		      node [midway, above] {Distance $r$};

		% First mass
		\draw [fill=red] (-2, 0)
		      circle [radius=0.08cm] node [left] {$m_1\,\,$};

		% Force exerted from first mass
		\draw [->] (-1.9, 0) -- (-0.3, 0)
		      node [midway, above] {$\vec{F}$};

		% First helper line
		\draw [dashed] (-2, 0.06) -- (-2, 1.3);

		% Second mass
		\draw [fill=blue] (2, 0)
		      circle [radius=0.08cm] node [right] {$\,\,m_2$};

		% Force exerted from second mass
		\draw [->] (1.9, 0) -- (0.3, 0)
		      node [midway, above] {$\vec{F}$};

		% Second helper line
		\draw [dashed] (2, 0.06) -- (2, 1.3);

	\end{tikzpicture}
\end{figure}

\vspace{\parskip}

The constant of gravitational acceleration, denoted by the letter $g$, is used in many physical equations. It contains the constant as well as all quantities of the universal law of gravitation that are specific to earth, such that $$g = G \cdot \frac{\text{Mass of earth}}{(\text{Radius of earth})^2} \approx 9.80665 \, \frac{m}{s^2}$$ Therefore, Newton's second law of motion, stating that any force is equal to the mass of an object multiplied by its acceleration, can be used as a quicker form of the universal law of gravity when dealing with systems on earth: $$F = G \cdot \frac{m_1 \cdot m_2}{r^2} = m \cdot g$$

Newton's other conclusions about the universe include:

\begin{itemize}
	\item Even though the sun is at the center of the Solar System, the universe itself has no center
	\item The reason for this is that, according to Newton, the universe is infinite
	\item All phenomena of astronomy and planetary motion are based on mathematical laws of motion and gravitation
\end{itemize}

\pagebreak

\sub{Weight and Weightlessness}

Considering Isaac Newton's definition of gravity, the distinction between mass and weight must be made clear. Mass is a scalar quantity and a measure of how much matter something contains. It is measured in kilograms $[kg]$, using a balance comparing a known amount of matter to an unknown amount of matter. Mass always stays the same, even if you change location (e.g. another planet). 

Weight, on the other hand, is a force, thus a vector quantity that is measured in Newtons $[N]$, that describes the gravitational pull on an object. It is measured on a scale and changes with location, as different locations may exert a different gravitational pull. The weight of an object can be calculated using the universal law of gravitation. As such, it requires knowledge of the object's mass as well as that of the body exerting the gravitational pull on the object. Also, the distance between the two must be known. 

\begin{table}[h!]
	\centering
	\begin{tabular}{| c | c | c |}
		\hline
		& Mass & Weight
		\\ \hline
		Quantity & Scalar & Vector
		\\ \hline
		Unit & Kilograms $[kg]$ & Newtons $[N]$
		\\ \hline
		Measured with & Balance & Scale
		\\ \hline
		Changes with Location & No & Yes
		\\ \hline
	\end{tabular}
\end{table}

\textbf{Example: } \emph{Calculate the weight of an object with a mass of 80 kilograms on earth as well as on the moon.}

Planet earth has a mass of $5.972 \cdot 10^24 \,kg$ and a radius of $6.371 \cdot 10^6 \,m$. Thus, the weight of the object can be calculated with the following equation: $$G \cdot \frac{m_1 \cdot m_2}{r^2} =  6.67 \cdot 10^{-11} \cdot \frac{80 \cdot 5.972 \cdot 10^24}{(6.371 \cdot 10^6)^2} \approx 785.09 N$$ As was previously mentioned, when dealing with gravity on earth, this information is contained in the gravitational constant of acceleration $g$. Therefore, the above equation can be approximated and simplified with an implementation of Newton's second law of motion: $$F = m \cdot a = m \cdot g = 80 \cdot 9.80665 \approx 784.53 N$$ The gravity on the moon is approximately one sixth of that on earth, thus the weight of the object on the moon will be: $$\frac{1}{6} \cdot m \cdot g \approx 130.85 N$$ This number could have also been calculated using the universal law of gravitation with the moon's mass and radius, of course. 

\subsub{Free Fall}

Free fall is the situation in which the only force acting upon an object is the gravitational force --- its own \emph{weight}. In such a case, there is no air resistance or any other force to counteract or balance the gravitational force pulling the object towards earth (or any other planet). Under such conditions, all objects will fall with the same rate of acceleration $g$, regardless of their mass. The reason for this is that, according to Newton's second law of motion, the acceleration of an object is indirectly proportional to its mass and directly proportional to the force acting upon it. Therefore, if the mass of an object increases while the force stays constant, the acceleration of this object decreases. However, in this case, the relevant force is gravity. As was shown, the gravitational force is directly proportional to the two masses it acts on --- the earth's and the object's. The greater the mass of the object, the greater the gravitational force. Thus, if the mass of the object increases, its acceleration decreases, but increases again by the same factor because the force acting on it, to which the acceleration is directly proportional, changes by that same factor. The decrease in acceleration caused by the change in mass is canceled out by the increase in acceleration caused by the consequent change in (gravitational) force. Practically speaking, a greater mass means a greater degree of inertia -- the resistance of an object to change its state of motion. Therefore, if two masses were in free fall and acted on by the same constant force, the more massive one, with more inertia, would accelerate less. But this is not the case. It being more massive, it experiences a greater gravitational force and is pulled more towards earth than the lesser mass. Thus, in effect, the ratio between the gravitational force exerted on an object and its mass -- its acceleration -- is always the same value 

\begin{figure}[h!]
	\centering
	\begin{tikzpicture}

		% Bigger mass label
		\draw (-2, 1) node {$M = 1000 \, kg$};

		% Bigger mass circle
		\draw [fill=black] (-2, 0) circle [radius=0.2cm];

		% Bigger mass force arrow
		\draw [->] (-2, -0.2) -- (-2, -2);

		% Bigger mass force label
		\draw (-2, -2.5) node {$F_{grav} \approx 9800N$};

		% Bigger mass acceleration
		\draw (-4, -1) node {$a = \frac{F_{grav}}{M} \approx 9.8 \, m/s^2$};


		% Smaller mass label
		\draw (2, 1) node {$m = 10 \, kg$};

		% Smaller mass circle
		\draw [fill=black] (2, 0) circle [radius=0.1cm];

		% Smaller mass force arrow
		\draw [->] (2, -0.1) -- (2, -2);

		% Smaller mass force label
		\draw (2, -2.5) node {$F_{grav} \approx 98N$};

		% Smaller mass acceleration
		\draw (4, -1) node {$a = \frac{F_{grav}}{m} \approx 9.8 \, m/s^2$};
	\end{tikzpicture}
\end{figure}

With air resistance, the picture is a little different. Air resistance acts as a frictional force countering the force of gravity on an object. It is caused by collisions of a body with air molecules, which slow down its fall (a source of acceleration). As such, it is dependent mainly on the speed of the object and its surface area. An interesting phenomenon associated with air resistance and free fall is that objects reach a terminal (final) velocity after a certain period of time. This is explained by the fact that as an object accelerates, it reaches a higher and higher velocity, thus experiences a greater and greater air resistance. At a certain point, this air resistance provides enough force in the opposite direction of the force pulling the object downwards --- gravity --- to cancel and balance the forces acting on it. At this point, the object no longer accelerates (because no \emph{net} force is acting on it) and moves with uniform velocity from that point on. Interestingly, here the mass of the object does play a significant role concerning when this terminal velocity is reached and thus how high it is. The reason for this is that objects with a greater mass experience a greater gravitational pull --- more weight --- which requires higher air resistance and thus higher speeds to be equaled and balanced out. In comparison to a lighter object, whose weight is quickly cancelled by air resistance, a more massive one will therefore have more time to accelerate with the same rate of acceleration $g$ until it encounters an amount of air resistance that is equal to its weight. More time to accelerate ultimately results in a higher terminal velocity --- the heavier object falls more quickly.

\begin{figure}[h!]
	\centering
	\begin{tikzpicture}

		\foreach \t in {0, 2, 4, 6}
		{
			% Air resistance label
			\draw (\t, 1.5) node {$F_{air}$};

			% Force arrow
			\draw [<->] (\t, 1) -- (\t, -1);

			% Mass circle
			\draw [fill=black] (\t, {0.75 - \t * 0.125}) circle [radius=0.1cm];

			% Gravity label
			\draw (\t, -1.4) node {$F_{grav}$};

			% Time label
			\draw (\t, -2.1) node {$t = \t$};
		}

	\end{tikzpicture}
\end{figure}

Thus, in summary, it can be stated that mass does not influence an object's acceleration in free fall. Therefore, if no air resistance is present, light and heavy objects fall with the same velocity. However, if air resistance is indeed present, a more massive object will reach the ground with a higher velocity than a lighter one, because it has more time to accelerate until the frictional force of air resistance can balance out the gravitational force acting upon it. 

\subsub{Weightlessnes}

\sub{Basic Concepts of Circular Motion and Rotation}

\sub{The Centripetal Force}

\sub{Cosmic Velocities}

\subsub{The First Cosmic Velocity}

The first cosmic velocity is the horizontal velocity needed to bring an object into a circular orbit with radius $r$ around a central body with mass $M$ (at velocities lower than this the object would fall towards the body).

\sub{Satellites}

\sub{Evolution of Stars}

\subsub{Hertzsprung-Russel Diagrams}

\end{document}