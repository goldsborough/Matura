\documentclass[11pt]{article}

\usepackage[a4paper, margin=1in]{geometry}

\usepackage{amsmath}

\usepackage{amssymb}

\usepackage{eurosym}

\let\normaleuro\euro
\renewcommand{\euro}{{\,\footnotesize \normaleuro}}

\usepackage[german]{babel}

\usepackage[autostyle=true]{csquotes}

\usepackage{libertine}

\usepackage[libertine]{newtxmath}

\usepackage[dvipsnames, usenames]{xcolor}

\definecolor{nicegreen}{HTML}{009933}

\usepackage{tikz}

\usepackage{gensymb}

\usepackage{fancyhdr}

\usepackage{cancel}

\usepackage{calc}

\usepackage{etoolbox}

\usepackage{pgfplots}

\pgfplotsset{compat=1.6}

\everymath{\displaystyle}

% Header / footer settings

\pagestyle{fancy}
\fancyhf{}
\renewcommand{\headrulewidth}{0.2mm}
\fancyhead[C]{Wahrscheinlichkeit}
\renewcommand{\footrulewidth}{0.2mm}
\fancyfoot[L]{Peter Goldsborough}
\fancyfoot[C]{\thepage}
\fancyfoot[R]{\today}

\fancypagestyle{plain}{%
	\fancyhf{}
	\renewcommand{\headrulewidth}{0mm}%
	\renewcommand{\footrulewidth}{0.2mm}%
	\fancyfoot[L]{Peter Goldsborough}
	\fancyfoot[C]{\thepage}
	\fancyfoot[R]{\today}
}

\newcommand{\power}[2]
{
	\ifnum#2=0
		#1 1\relax
	\else
		\ifnum#2=1
			\relax
		\else

			\newcount\x
			\x \the#1\relax

			\foreach \ in {2,...,#2}
			{
				\global \multiply \x by \the#1\relax
			}

			\global #1 \the\x
		\fi
	\fi
}

\setlength{\headheight}{15pt}

\setlength{\parindent}{0pt}

\addtolength{\parskip}{\baselineskip}

\newcommand{\defas}{ \dots \,\,}

\newcommand{\heading}[1]{\begin{center}\Huge \textbf{#1}\end{center}\par}

\newcommand{\sub}[1]{\vspace{\parskip}{\LARGE\textbf{#1}}}

\newcommand{\subsub}[1]{{\large \textbf{#1}}}

\newcommand{\colvec}[1]{\begin{pmatrix}#1\end{pmatrix}}

\newcommand{\extrapar}{\par\vspace{\baselineskip}}

\newcommand{\zitat}[1]{\foreignquote{german}{#1}}

\newcommand{\bolditem}[1]{\item \textbf{#1}}

\newcommand{\titleitem}[1]{\bolditem{#1} \par}

\newcommand{\beispiel}[1]{\textbf{Beispiel}: \emph{#1}}