\documentclass[11pt]{article}

\usepackage[a4paper, margin=1in]{geometry}

\usepackage{amsmath}

\usepackage{amssymb}

\usepackage[german]{babel}

\usepackage[autostyle=true]{csquotes}

\usepackage{libertine}

\usepackage[libertine]{newtxmath}

\usepackage{tikz}

\usepackage{gensymb}

\usepackage{fancyhdr}

\usepackage{amsfonts}

\usepackage{pgfplots}

\usepgfplotslibrary{statistics}

\pgfplotsset{compat=1.10}

\usepackage{multicol}

\usepackage{caption}

\usepackage{floatrow}

\usepackage[labelfont=bf]{caption}

\everymath{\displaystyle}

% Header / footer settings

\pagestyle{fancy}
\fancyhf{}
\renewcommand{\headrulewidth}{0.2mm}
\fancyhead[C]{Statistik}
\renewcommand{\footrulewidth}{0.2mm}
\fancyfoot[L]{Peter Goldsborough}
\fancyfoot[C]{\thepage}
\fancyfoot[R]{\today}

\fancypagestyle{plain}{%
	\fancyhf{}
	\renewcommand{\headrulewidth}{0mm}%
	\renewcommand{\footrulewidth}{0.2mm}%
	\fancyfoot[L]{Peter Goldsborough}
	\fancyfoot[C]{\thepage}
	\fancyfoot[R]{\today}
}


\setlength{\headheight}{15pt}

\setlength{\parindent}{0pt}

\addtolength{\parskip}{\baselineskip}


\newcommand{\overbar}[1]{\mkern 1.5mu\overline{\mkern-1.5mu#1\mkern-1.5mu}\mkern 1.5mu}

\newcommand{\heading}[1]{\begin{center}\Huge \textbf{#1}\end{center}\par}

\newcommand{\sub}[1]{\vspace{\parskip}{\LARGE\textbf{#1}}}

\newcommand{\subsub}[1]{{\Large \textbf{#1}}}

\newcommand{\colvec}[1]{\begin{pmatrix}#1\end{pmatrix}}

\newcommand{\extrapar}{\vspace{\parskip}\par}

\newcommand{\zitat}[1]{\foreignquote{german}{#1}}

\newcommand{\bolditem}[1]{\item \textbf{#1}}

\newcommand{\titleitem}[1]{\bolditem{#1}\par}

\newcommand{\defas}{ \dots \,\,}

\newcommand{\ceil}[1]{\left\lceil #1 \right\rceil}

\newcommand{\floor}[1]{\left\lfloor #1 \right\rfloor}

\newenvironment{eq}
{\begin{flushleft} \setlength{\parskip}{0.2cm}}
{\end{flushleft}}

% Source: http://tex.stackexchange.com/questions/135393/how-to-draw-bar-pie-chart

\tikzstyle{chart}=[
    legend label/.style={font={\scriptsize},anchor=west,align=left},
    legend box/.style={rectangle, draw, minimum size=5pt},
    axis/.style={black,semithick,->},
    axis label/.style={anchor=east,font={\tiny}},
]

\tikzstyle{pie chart}=[
    chart,
    slice/.style={line cap=round, line join=round, very thick,draw=white},
    pie title/.style={font={\bf}},
    slice type/.style 2 args={
        ##1/.style={fill=##2},
        values of ##1/.style={}
    }
]

\pgfdeclarelayer{background}
\pgfdeclarelayer{foreground}
\pgfsetlayers{background,main,foreground}


\newcommand{\pie}[3][]{
    \begin{scope}[#1]
    \pgfmathsetmacro{\curA}{90}
    \pgfmathsetmacro{\r}{1}
    \def\c{(0,0)}
    \node[pie title] at (90:1.3) {#2};
    \foreach \v/\s in{#3}{
        \pgfmathsetmacro{\deltaA}{\v/100*360}
        \pgfmathsetmacro{\nextA}{\curA + \deltaA}
        \pgfmathsetmacro{\midA}{(\curA+\nextA)/2}

        \path[slice,\s] \c
            -- +(\curA:\r)
            arc (\curA:\nextA:\r)
            -- cycle;
        \pgfmathsetmacro{\d}{max((\deltaA * -(.5/50) + 1) , .5)}

        \begin{pgfonlayer}{foreground}
        \path \c -- node[pos=\d,pie values,values of \s]{$\v\%$} +(\midA:\r);
        \end{pgfonlayer}

        \global\let\curA\nextA
    }
    \end{scope}
}
