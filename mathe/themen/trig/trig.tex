% Trigonometrie

\documentclass[11pt]{article}

\usepackage[a4paper, margin=1in]{geometry}

\usepackage{amsmath}

\usepackage{amssymb}

\usepackage[german]{babel}

\usepackage[autostyle=true]{csquotes}

\usepackage{libertine}

\usepackage[libertine]{newtxmath}

\usepackage{tikz}

\usepackage{gensymb}

\usepackage{fancyhdr}

\usepackage{amsfonts}

\usepackage{pgfplots}

\pgfplotsset{compat=1.10}

\usepackage{multicol}

\usepackage{caption}

\usepackage{floatrow}

\everymath{\displaystyle}

% Header / footer settings

\pagestyle{fancy}
\fancyhf{}
\renewcommand{\headrulewidth}{0.2mm}
\fancyhead[C]{Funktionen}
\renewcommand{\footrulewidth}{0.2mm}
\fancyfoot[L]{Peter Goldsborough}
\fancyfoot[C]{\thepage}
\fancyfoot[R]{\today}

\fancypagestyle{plain}{%
	\fancyhf{}
	\renewcommand{\headrulewidth}{0mm}%
	\renewcommand{\footrulewidth}{0.2mm}%
	\fancyfoot[L]{Peter Goldsborough}
	\fancyfoot[C]{\thepage}
	\fancyfoot[R]{\today}
}


\setlength{\headheight}{15pt}

\setlength{\parindent}{0pt}

\addtolength{\parskip}{\baselineskip}


\newcommand{\overbar}[1]{\mkern 1.5mu\overline{\mkern-1.5mu#1\mkern-1.5mu}\mkern 1.5mu}

\newcommand{\heading}[1]{\begin{center}\Huge \textbf{#1}\end{center}\par}

\newcommand{\sub}[1]{\vspace{\parskip}{\LARGE\textbf{#1}}}

\newcommand{\subsub}[1]{{\Large \textbf{#1}}}

\newcommand{\subsubsub}[1]{\textbf{#1}}

\newcommand{\colvec}[1]{\begin{pmatrix}#1\end{pmatrix}}

\newcommand{\extrapar}{\par\vspace{\baselineskip}}

\newcommand{\zitat}[1]{\foreignquote{german}{#1}}

\newcommand{\bolditem}[1]{\item \textbf{#1}}

\newcommand{\titleitem}[1]{\bolditem{#1}\par}

\newcommand{\defas}{ \dots \,\,}

\begin{document}
\thispagestyle{plain}

\heading{Trigonometrie}

\sub{Definition}

Die Trigonometrie behandelt die Dreiecksvermessung von rechtwinkeligen sowie anderen Dreiecken im Einheitskreis sowie allgemein. Wichtige Funktionen sind hierbei Sinus, Cosinus sowie Tangens, mit welchen man den Sinus- und Cosinussatz bilden kann. Ebenso spielt der Satz des Pythagoras eine bedeutende Rolle.

\sub{Schreibweisen}

Allgemein ist f\"{u}r die Trigonometrie wichtig, dass man Punkte in einem Koordinatensystem als kartesische Koordinaten oder als Polarkoordinaten anschreiben kann. 

Ein kartesisches Koordinatentupel $P (x | y)$ besteht aus einer Variable $x$, welche einen Abstand auf der Abszisse ($x$-Achse) beschreibt, sowie einer Variable $y$, welche die Position des Punktes auf der Ordinate ($y$-Achse) angibt. Ein Beispiel w\"{a}re der Punkt $P(3 | 4)$.

In Polarschreibweise wird ein Punkt $P [r; \varphi]$ durch einen Winkel $\varphi$, der eine Richtung zwischen $0$ und $360$ Grad angibt, sowie einen Abstand vom Ursprung $r$ (Radius) in die Richtung des Winkels beschrieben. Der vorherige Punkt $P$ w\"{a}re als Polarkoordinatentupel so angegeben: $P [5; 53.13\degree]$

\sub{Sinus und Cosinus im Einheitskreis}

Sowohl die Sinusfunktion $sin(x)$ als auch die Cosinusfunktion $cos(x)$ beschreiben Seitenverh\"{a}ltnisse zwischen den Seiten $a, b, c$ eines rechtwinkeligen Dreiecks. Besonders im Einheitskreis sind diese Verh\"{a}ltnisse von Interesse. Ein Einheitskreis $k$ ist jener Kreis, dessen Mittelpunkt $M (0 | 0)$ im Ursprung liegt und dessen Radius $r$ eine L\"{a}nge von $1$ besitzt. Geometrisch wird ein Einheitskreis durch die Gleichung $x^2 + y^2 = 1$ beschrieben.

Am Einheitskreis kann jeder Punkt $P$ in Polarform mit $P [1;\varphi]$ und in Kartesischer Form mit $P [\cos \varphi; \sin \varphi]$ angegeben werden. Die Steigung $k$ der Hypothenuse jenes rechtwinkligen Dreiecks, welches zwischen der An- und Gegenkathete des Punktes aufgespannt wird und im Falle des Einheitskreises stets eine L\"{a}nge von $r = 1$ hat, wird durch die Tangensfunktion $\tan \varphi$ beschrieben. Daraus folgt:

$\sin \varphi$ \defas $y$-Koordinate des Punktes $P$

$\cos \varphi$ \defas $x$-Koordinate des Punktes $P$

$\tan \varphi$ \defas Steigung $k$ der Hypotenuse im Punkt $P \Rightarrow \tan \varphi = \frac{\sin \varphi}{\cos \varphi} = \frac{\Delta y}{\Delta x} = k$

In dem zwischen $x = \cos \varphi$ und $y = \sin \varphi$ aufgespannten rechtwinkligen Dreieck mit Hypotenuse $r = 1$, kann man die Seitenverh\"{a}ltnisse auch durch den Satz des Pythagoras $a^2 + b^2 = c^2$ beschreiben: $$\sin^2 \varphi + \cos^2 \varphi = 1$$

\pagebreak

\begin{figure}[t!]
\centering
	\begin{tikzpicture}

		% x Achse
		\draw [<->]
		      (-4, 0)
		   -- (-3, 0) node [above left] {$-1$}
		   -- ( 3, 0) node [above right] {$1$}
		   			  node [below, pos=0.7] {$x = \cos \varphi$}
		   -- ( 4, 0) node [above] {$x$};

		% y Achse
		\draw [<->]
		      (0, -4)
		   -- (0, -3) node [below right] {$-1$}
		   -- (0,  3) node [above right] {$1$}
		   -- (0,  4) node [right] {$y$};

		% Quadrant
		\draw ( 3,  3.5) node {I}
		      (-3,  3.5) node {II}
		      (-3, -3.5) node {III}
		      ( 3, -3.5) node {IV};

		% Kreis
		\draw (0, 0) circle [radius=3cm]
		      		 node [below left] {$0$};

		% Dreieck
		\draw [fill=black]
		      (0, 0)
		   -- (2.121, 2.121)
		      node [midway, above, rotate = 45] {$\mathbf{k = \tan \varphi}$}
			  circle [radius=1.2pt] node [above right] {$P (x | y)$}
		   -- (2.121, 0)    
		      node [midway, right] {$y = \sin \varphi$};


		% Winkelbogen
		\draw (1, 0) arc [start angle = 0, end angle = 45, radius = 1cm];

		% Angle node
		\draw (0.6, 0.25) node {$\varphi$};


	\end{tikzpicture}
	\caption*{Ein Einheitskreis mit Mittelpunkt $M (0 | 0)$ und $r = 1$, in welchem zwischen $x = \cos \varphi$ und $y = \sin \varphi$ ein rechtwinkliges Dreieck mit $\varphi = 45\degree$ aufgespannt ist.}
\end{figure}

Die  $y$-Koordinate des Punktes ist in der oberen H\"{a}lte des Einheitskreises, also Quadranten I und II, stets positiv und in der unteren H\"{a}lfte, also Quadranten III und IV, negativ. Man \"{u}berlege sich dazu den Verlauf der Sinusfunktion $\sin(x)$. Sie beginnt mit $x = y = 0$, findet nach $x = 90\degree$ ihr Maximum, wo $y = 1$, f\"{a}llt dann bis zu $(x = 180\degree | y = 0)$ und wechselt dann ihr Vorzeichen. Gegens\"{a}tzlich dazu wechselt das Vorzeichen der $x$-Kathete nach Quadranten I und III, was auch mit dem Verlauf der Cosinusfunktion $\cos(x)$ \"{u}bereinstimmt. Aus diesen Beobachtungen folgt:

\vspace{1cm}

\begin{table}[h!]
	\large
	\begin{tabular}{c | c | c | c}
		Quadrant & Winkel & $\sin \varphi$ & $\cos \varphi$
		\\ \hline
		I & $0 < \varphi < 90$ & + & +
		\\ 
		II & $90 < \varphi < 180$ & + & -
		\\
		III & $180 < \varphi < 270$ & - & -
		\\
		IV & $270 < \varphi < 360$ & - & +
		\\
	\end{tabular}
\end{table}

\pagebreak

\sub{Sinus und Cosinus im rechtwinkligen Dreieck}

Wie vorhin angemerkt beschreiben Sinus und Cosinus Verh\"{a}ltnisse zwischen den Seiten $a, b$ und $c$ in einem rechtwinkligen Dreieck. In diesem bezeichnet man die dem Winkel $\varphi$ gegen\"{u}berliegende Seite als \emph{Gegenkathete}, die anliegende als \emph{Ankathete} und die l\"{a}ngste Seite als \emph{Hypotenuse}. Gegen- und Ankathete fallen unter den Sammelbegriff \emph{Kathete}. F\"{u}r einen Winkel $0 < \varphi < 90$ gilt in einem rechtwinkligen Dreieck somit:

\begin{center}
$\sin \varphi = \frac{\text{Gegenkathete}}{\text{Hypotenuse}}$
\hspace{1cm}
$\cos \varphi = \frac{\text{Ankathete}}{\text{Hypotenuse}}$
\hspace{1cm}
$\tan \varphi = \frac{\text{Gegenkathete}}{\text{Ankathete}}$
\end{center}

Nimmt man die Tatsache in Betracht, dass im Einheitskreis die Hypotenuse $r$ stets die L\"{a}nge $1$ besitzt, wird klar, das $\sin \varphi$ gleich der $y$-Koordinate ist, da dann gilt $\sin \varphi = \frac{\text{Gegenkathete}}{\text{Hypotenuse}} = \frac{y}{1} = y$. Selbes gilt f\"{u}r $x$ und $\cos \varphi$.


\begin{figure}[h!]
	\begin{tikzpicture}

		% Triangle
		\draw (0, 0)
		   -- (4, 0) node [midway, below] {Ankathete}
		   -- (4, 2) node [midway, right] {Gegenkathete}
		   -- (0, 0) node [midway, above, rotate = 26] {Hypotenuse};

		% Right angle arc
		\draw (4, 0.75) arc [radius = 0.75cm, start angle = 90, end angle = 180];

		% Right angle dot
		\draw [fill=black] (3.7, 0.3) circle [radius=1.2pt];

		% Phi arc
		\draw (1.2, 0) arc [radius=1.2cm, start angle = 0, end angle = 26];

		% Phi dot
		\draw (0.95, 0.25) node {$\varphi$};

	\end{tikzpicture}
	\caption*{Ein rechtwinkliges Dreieck mit Gegen- und Ankathete sowie Hypotenuse und Winkel $\varphi$}
\end{figure}

\sub{Sinussatz}

Der Sinussatz ist eine Reihe von Verh\"{a}ltnissen zwischen den Seiten und Winkeln eines Dreiecks, welche einem helfen k\"{o}nnen, ein allgemeines Dreieck aufzul\"{o}sen: $$\frac{a}{\sin \alpha} = \frac{b}{\sin \beta} = \frac{c}{\sin \gamma}$$

Man bemerke, dass die Proportionalit\"{a}t zwischen Winkelspanne und L\"{a}nge der gegen\"{u}berliegenden Seite daher folgt, dass die L\"{a}nge jeder Seite im Winkelbogen aufgespannt werden muss. Ein gr\"{o}\ss{}erer Winkel bedeutet ein breiterer Winkelbogen, in welchem die Seite aufgespannt wird. Dieses Verh\"{a}ltnis besteht zwischen jedem Winkel und seiner gegen\"{u}berliegenden Seite. Er wird am besten dann angewendet, wenn man von einem allgemeinen Dreieck kennt:
\begin{itemize}
	\item eine Seite und zwei Winkel
	\item zwei Seiten und einen Winkel, welcher von den Seiten nicht eingeschlossen wird
\end{itemize}

\begin{figure}[h!]
	\begin{tikzpicture}

		% c
		\draw [red] (0, 0) -- (4, 0) node [midway, below] {c};

		% a
		\draw [blue] (4, 0) -- (3, 2) node [midway, right] {a};

		% b
		\draw [teal] (3, 2) -- (0, 0) node [midway, above  left] {b};

		% gamma arc
		\draw [red] (2.5, 5/3) arc [radius=0.5cm, start angle=220, end angle=310];

		% gamma node
		\draw [red] (2.95, 1.7) node {$\gamma$};

		% alpha arc
		\draw [blue] (1, 0) arc [radius=1cm, start angle = 0, end angle = 33];

		% alpha node
		\draw [blue] (0.7, 0.25) node {$\alpha$};

		% beta arc
		\draw [teal] (3.3, 0) arc [radius=0.7cm, start angle = 180, end angle = 117];

		% beta node
		\draw [teal] (3.6, 0.25) node {$\beta$};

	\end{tikzpicture}
\end{figure}

\pagebreak

\subsub{Herleitung des Sinussatzes}

Die Herleitung des Sinussatzes ist vergleichsweise einfach. Man nehme ein allgemeines Dreieick und teile es durch die H\"{o}he $h_c$ in zwei rechtwinklige Dreiecke:

\begin{figure}[h!]
	\begin{tikzpicture}

		% c
		\draw [fill=black]
		      (0, 0) circle [radius=1.2pt] node [left] {A}
		   -- (4, 0) circle [radius=1.2pt] node [right] {B}
		      node [midway, below] {c};

		% a
		\draw [fill=black]
		     (4, 0)
		  -- (3, 2) circle [radius=1.2pt] node [above] {C}
		      node [midway, right] {a};

		% b
		\draw (3, 2) -- (0, 0) node [midway, above  left] {b};

		% gamma arc
		\draw (2.5, 5/3) arc [radius=0.5cm, start angle=220, end angle=310];

		% gamma node
		\draw (2.88, 1.7) node {$\gamma$};

		% alpha arc
		\draw (1, 0) arc [radius=1cm, start angle = 0, end angle = 33];

		% alpha node
		\draw (0.7, 0.25) node {$\alpha$};

		% beta arc
		\draw (3.3, 0) arc [radius=0.7cm, start angle = 180, end angle = 117];

		% beta node
		\draw (3.6, 0.25) node {$\beta$};

		% h_c
		\draw (3, 2) -- ++(0, -2) node [midway, left] {$h_c$};

	\end{tikzpicture}
\end{figure}

Nun lassen sich folgende Verh\"{a}ltnisse zwischen den Seiten $a$ und $b$, den Winkeln $\alpha$ und $\beta$ sowie der H\"{o}he $h_c$ finden: $$\sin \alpha = \frac{h_c}{b} \hspace{2cm} \sin \beta = \frac{h_c}{a}$$ Aus diesen l\"{a}sst sich folglich die H\"{o}he $h_c$ aus einer der beiden Gleichungen ausdr\"{u}cken um somit $h_c$ durch den entstehenden Ausdruck in der zweiten Gleichung zu substitutieren: $$\sin \alpha \cdot b = h_c \hspace{1cm} \Rightarrow \hspace{1cm} \sin \beta = \frac{\sin \alpha \cdot b}{a}$$ Diese Gleichung kann nun weiter umgeformt werden indem man beide Seiten durch die Seite $b$ teilt, sodass letztlich gilt: $$\frac{\sin \beta}{b} = \frac{\sin \alpha}{a}$$ F\"{u}r den Winkel $\gamma$ und die Seite $b$ m\"{u}sste man das Dreieick durch eine andere H\"{o}he teilen und das selbe Verfahren anwenden.

\sub{Cosinussatz}

Der Cosinussatz beschreibt eine Reihe von Verh\"{a}ltnissen, bei welchen eine Seite bzw. ihr gegen\"{u}berliegender Winkel von den beiden anderen Seiten eines allgemeinen Dreiecks abh\"{a}ngen: $$a^2 = b^2 + c^2 - 2bc \cdot \cos \alpha$$ $$b^2 = a^2 + c^2 - 2ac \cdot \cos \beta$$ $$c^2 = a^2 + b^2 - 2ab \cdot \cos \gamma$$

Den Cosinussatz verwendet man am besten dann, wenn man kennt:
\begin{itemize}
	\item drei Seiten
	\item zwei Seiten und einen Winkel, der von den Seiten eingeschlossen wird
\end{itemize}

\begin{figure}[h!]
	\begin{tikzpicture}

		% c
		\draw [red] (0, 0) -- (4, 0) node [midway, above] {c};

		% a
		\draw (4, 0) -- (3, 2) node [midway, right] {a};

		% b
		\draw [red] (3, 2) -- (0, 0) node [midway, above  left] {b};

		% gamma arc
		\draw (2.5, 5/3) arc [radius=0.5cm, start angle=220, end angle=310];

		% gamma node
		\draw (2.95, 1.7) node {$\gamma$};

		% alpha arc
		\draw [blue] (1, 0) arc [radius=1cm, start angle = 0, end angle = 33];

		% alpha node
		\draw [blue] (0.7, 0.25) node {$\alpha$};

		% beta arc
		\draw (3.3, 0) arc [radius=0.7cm, start angle = 180, end angle = 117];

		% beta node
		\draw (3.6, 0.25) node {$\beta$};

	\end{tikzpicture}
\end{figure}

\pagebreak

\subsub{Herleitung des Cosinussatzes}

Die Herleitung des Cosinussatzes ist etwas schwieriger als jene des Sinussatzes. Wieder teilt man ein allgemeines Dreieick durch die H\"{o}he $h_c$ in zwei rechtwinkelige Dreiecke:

\begin{figure}[h!]
	\begin{tikzpicture}

		% c
		\draw [fill=black]
		      (0, 0) circle [radius=1.2pt] node [left] {A}
		   -- (3, 0) node [midway, below] {$c_1$}
		   -- (4, 0) circle [radius=1.2pt] node [right] {B}
		      node [midway, below] {$c_2$};

		% a
		\draw [fill=black]
		     (4, 0)
		  -- (3, 2) circle [radius=1.2pt] node [above] {C}
		      node [midway, right] {a};

		% b
		\draw (3, 2) -- (0, 0) node [midway, above  left] {b};

		% gamma arc
		\draw (2.5, 5/3) arc [radius=0.5cm, start angle=220, end angle=310];

		% gamma node
		\draw (2.88, 1.7) node {$\gamma$};

		% alpha arc
		\draw (1, 0) arc [radius=1cm, start angle = 0, end angle = 33];

		% alpha node
		\draw (0.7, 0.25) node {$\alpha$};

		% beta arc
		\draw (3.3, 0) arc [radius=0.7cm, start angle = 180, end angle = 117];

		% beta node
		\draw (3.6, 0.25) node {$\beta$};

		% h_c
		\draw (3, 2) -- ++(0, -2) node [midway, left] {$h_c$};

	\end{tikzpicture}
\end{figure}

Nun bemerke man, dass die Seite $b$ die Hypotenuse des linken rechtwinkeligen Dreiecks mit den Katheten $c_1$ und $h_c$ ist, w\"{a}hrend die Seite $a$ die Hypotenuse im rechten rechtwinkeligen Dreieck ist, in welchem $c_2$ und $h_c$ die Katheten darstellen. Daher l\"{a}sst sich in beiden Dreiecken das Quadrat der Hypotenuse als die Summe der Quadrate der Kathetenl\"{a}ngen beschreiben: $$b^2 = c_1^2 + h_c^2 \hspace{2cm} a^2 = c_2^2 + h_c^2$$ Aus einer dieser beiden Gleichungen l\"{a}sst sich das H\"{o}henquadrat ausdr\"{u}cken, welches folglich f\"{u}r $h_c^2$ in der zweiten Gleichung eingesetzt werden kann: $$h_c^2 = a^2 - c_2^2 \hspace{1cm} \Rightarrow \hspace{1cm} b^2 = a^2 - c_2^2 + c_1^2$$ Die Summe aus den Teill\"{a}ngen $c_1$ und $c_2$ ist nat\"{u}rlich die L\"{a}nge $c$. Daher muss $c_2$ gleich der Differenz aus $c$ und $c_1$ sein. Somit verschwindet eine Teill\"{a}nge aus der Gleichung: $$b^2 = a^2 - (c - c_1)^2 + c_1^2$$ Der Term in der Klammer ist eine binomische Formel, welche aufgel\"{o}st werden muss. Hierbei ist es wegen dem vorangestellten Minuszeichen wichtig, den ausgerechneten Term in der Klammer zu lassen: $$b^2 = a^2 - (c^2 - 2cc_1 + c_1^2) + c_1^2 \hspace{1cm} \Rightarrow \hspace{1cm} b^2 = a^2 - c^2 + 2cc_1 - c_1^2 + c_1^2$$ Die L\"{a}nge $c_1$ f\"{a}llt nun teilweise weg: $$b^2 = a^2 - c^2 + 2cc_1 - \cancel{c_1^2} + \cancel{c_1^2} \hspace{1cm} \Rightarrow \hspace{1cm} b^2 = a^2 - c^2 + 2cc_1$$ Die verbliebene L\"{a}nge $c_1$ kann nun durch den Cosinus des Winkels $\alpha$ sowie der Seite $b$ substituiert werden: $$\cos \alpha = \frac{c_1}{b} \hspace{1cm} \Rightarrow \hspace{1cm} b^2 = a^2 - c^2 + 2bc \cdot \cos \alpha$$ Danach m\"{u}ssen noch $a^2$ und $b^2$ getauscht werden, indem man von beiden Seiten der Gleichung $a^2$ sowie $b^2$ subtrahiert: $$-a^2 = -b^2 - c^2 + 2bc \cdot \cos \alpha$$ Letztlich wird noch mit $-1$ multipliziert und der Cosinussatz ist vollst\"{a}ndig hergeleitet: $$a^2 = b^2 + c^2 - 2bc \cdot \cos \alpha$$

\pagebreak

\sub{Anmerkungen}

Die Summe der Winkel bei einer Gerade ist $180\degree$ (Halbkreis):

\begin{figure}[h!]
	\begin{tikzpicture}
		\draw (0, 0) -- (4, 0);

		\draw (2, 0) -- (0.5, 1.2);

		\draw (2, 0) -- (2, 1.5);

		\draw (2, 0) -- (3, 1.5);

		\draw (1, 0) arc [radius=1cm, start angle = 180, end angle = 135];

		\draw (1.4, 0.25) node {$\alpha$};

		\draw (1.3, 0.7) arc [radius=1cm, start angle = 135, end angle = 90];

		\draw (1.8, 0.7) node {$\beta$};

		\draw (2, 1) arc [radius=1cm, start angle = 90, end angle = 65];

		\draw (2.2, 0.7) node {$\gamma$};

		\draw (3, 0) arc [radius=1cm, start angle = 0, end angle = 65];

		\draw (2.5, 0.3) node {$\delta$};

		% legend
		\draw (2, -0.5) node {$\alpha + \beta + \gamma + \delta = 180\degree$};

	\end{tikzpicture}
\end{figure}

Der H\"{o}henwinkel ist jener, den eine Gerade mit der Horizontalen unter sich einschlie\ss{}t (man blickt von unten in die \emph{H\"{o}he}). Der Tiefenwinkel ist gegens\"{a}tzlich dazu jener Winkel, den die Gerade mit der Horizontalen \"{u}ber sich einschlie\ss{}t (man blickt von oben in die \emph{Tiefe}):

\begin{figure}[h!]
	\begin{tikzpicture}

		% Gerade
		\draw (0, 0) -- (4, 2);

		% Untere Horizontale
		\draw [dashed, blue] (0, 0) -- (4, 0);

		% Hoehenwinkel arc
		\draw [blue] (1.5, 0) arc [radius=1cm, start angle = 0, end angle = 40];

		% Hoehenwinkel node
		\draw [blue] (1, 0.25) node {$H$};

		% Obere Horizontale
		\draw [red, dashed] (4, 2) -- (0, 2);

		% Tiefenwinkel arc
		\draw [red] (2.5, 2) arc [radius=1cm, start angle = 180, end angle = 220];

		% Tiefenwinkel node
		\draw [red] (3, 1.775) node {$T$};

	\end{tikzpicture}
\end{figure}

\end{document}