% Integral

\documentclass[11pt]{article}

\usepackage[a4paper, margin=1in]{geometry}

\usepackage{amsmath}

\usepackage{amssymb}

\usepackage[german]{babel}

\usepackage[autostyle=true]{csquotes}

\usepackage{libertine}

\usepackage[libertine]{newtxmath}

\usepackage{tikz}

\usepackage{gensymb}

\usepackage{fancyhdr}

\usepackage{amsfonts}

\usepackage{pgfplots}

\pgfplotsset{compat=1.10}

\usepackage{multicol}

\usepackage{caption}

\usepackage{floatrow}

\everymath{\displaystyle}

% Header / footer settings

\pagestyle{fancy}
\fancyhf{}
\renewcommand{\headrulewidth}{0.2mm}
\fancyhead[C]{Funktionen}
\renewcommand{\footrulewidth}{0.2mm}
\fancyfoot[L]{Peter Goldsborough}
\fancyfoot[C]{\thepage}
\fancyfoot[R]{\today}

\fancypagestyle{plain}{%
	\fancyhf{}
	\renewcommand{\headrulewidth}{0mm}%
	\renewcommand{\footrulewidth}{0.2mm}%
	\fancyfoot[L]{Peter Goldsborough}
	\fancyfoot[C]{\thepage}
	\fancyfoot[R]{\today}
}


\setlength{\headheight}{15pt}

\setlength{\parindent}{0pt}

\addtolength{\parskip}{\baselineskip}


\newcommand{\overbar}[1]{\mkern 1.5mu\overline{\mkern-1.5mu#1\mkern-1.5mu}\mkern 1.5mu}

\newcommand{\heading}[1]{\begin{center}\Huge \textbf{#1}\end{center}\par}

\newcommand{\sub}[1]{\vspace{\parskip}{\LARGE\textbf{#1}}}

\newcommand{\subsub}[1]{{\Large \textbf{#1}}}

\newcommand{\subsubsub}[1]{\textbf{#1}}

\newcommand{\colvec}[1]{\begin{pmatrix}#1\end{pmatrix}}

\newcommand{\extrapar}{\par\vspace{\baselineskip}}

\newcommand{\zitat}[1]{\foreignquote{german}{#1}}

\newcommand{\bolditem}[1]{\item \textbf{#1}}

\newcommand{\titleitem}[1]{\bolditem{#1}\par}

\newcommand{\defas}{ \dots \,\,}

\begin{document}
\thispagestyle{plain}

\heading{Integral}

Die Integralrechnung besch\"{a}ftigt sich mit der Berechnung von Fl\"{a}cheninhalten unter Kurven. Diese k\"{o}nnen entweder als Summe von Rechtecksfl\"{a}chen angen\"{a}hert oder mittels dem Integrationsverfahren genau bestimmt werden. Im Bereich der Analysis ist das Integrieren einer Funktion die inverse Operation zum Differenzieren.

\sub{Produktsummen}

Das Integral einer Kurve kann in einem Ann\"{a}herungsverfahren durch Summen von Fl\"{a}cheninhalten einfacher Figuren wie Rechtecken, Dreiecken oder Trapezen approximiert werden, da die Berechnung ihrer Fl\"{a}chen vergleichsweise einfach ist. Verwendet man Rechtecke f\"{u}r die Approximierung, so ist das Integral eine Summe vieler kleiner Rechtecke. Die Breite $\Delta x$ dieser Rechtecke ist hierbei ein festgelegter, konstanter Abstand auf der $x$-Achse, welcher die Funktion in die Intervalle $[0; \Delta x], [\Delta x; 2 \cdot \Delta x], [2 \cdot \Delta x; 3 \cdot \Delta x], ...$ teilt. F\"{u}r ein beliebiges Intervall $[a; b]$ einer Funktion, das in $N$ Teilintervalle geteilt werden soll, wird $\Delta x$ mittels der Formel $\Delta x = \frac{b - a}{N}$ berechnet. Die L\"{a}nge jeden Rechtecks ist der jeweilige Funktionswert $f(x)$ an den einzelnen Intervallsgrenzen. Das Integral ist somit eine Summe vieler Produkte: eine \emph{Produktsumme}. 

Es gibt vier verschiedene M\"{o}glichkeiten, Rechtecke unter einer Kurve aufzuspannen und so eine Produktsumme zu bilden: als Untersumme, Obersumme, Linkssumme oder Rechtssumme.

\subsub{Untersumme}

\begin{figure}[h!]
	\centering
	\begin{tikzpicture}

		% x axis
		\draw [->]
		      (-0.5, 0) 
		   -- (+4.5, 0) node [pos=0.06, below] {0}
		   			    node [above] {$x$}
		   				node [pos=0.2, below] {$\Delta x$}
		   				node [pos=0.4, below] {$\Delta x$}
		   				node [pos=0.6, below] {$\Delta x$};

		% y axis
		\draw [->]
		      (0, -0.5)
		   -- (0, +4.5) node [right] {$f(x)$};

		% Graph
		\draw [domain=0:4] plot(\x, {0.25 * (\x - 4)^2});

		% First rectangle
		\draw [red]
		      (1, 0)
		   -- (1, 2.25)
		   -- (0, 2.25);

		% First node
		\draw [fill=black]
		 	  (1, 2.25) circle [radius=1.2pt]
		   				node [above right] {$min_1$};

		% Second rectangle
		\draw [red]
		      (2, 0)
		   -- (2, 1)
		   -- (1, 1);

		% Second node
		\draw [fill=black]
		 	  (2, 1) circle [radius=1.2pt]
			         node [above right] {$min_2$};

		% Third rectangle
		\draw [red]
		      (3, 0)
		   -- (3, 0.25)
		   -- (2, 0.25);

		% Third node
		\draw [fill=black]
		 	  (3, 0.25) circle [radius=1.2pt]
			         node [above right] {$min_3$};

	\end{tikzpicture}
	%
	\hspace{2cm}
	%
	\begin{tikzpicture}

	% x axis
		\draw [->]
		      (-0.5, 0) 
		   -- (+4.5, 0) node [pos=0.06, below] {0}
		   			    node [above] {$x$}
		   				node [pos=0.2, below] {$\Delta x$}
		   				node [pos=0.4, below] {$\Delta x$}
		   				node [pos=0.6, below] {$\Delta x$}
		   				node [pos=0.8, below] {$\Delta x$};

		% y axis
		\draw [->]
		      (0, -0.5)
		   -- (0, +4.5) node [right] {$f(x)$};

		% Graph
		\draw [domain=0:4] plot(\x, {-0.2 * \x^2 + 4});

		% First rectangle
		\draw [red]
		      (1, 0)
		   -- (1, 3.8)
		   -- (0, 3.8);

		% First node
		\draw [fill=black]
		 	  (1, 3.8) circle [radius=1.2pt]
		   		     node [above right] {$min_1$};

		% Second rectangle
		\draw [red]
		      (2, 0)
		   -- (2, 3.2)
		   -- (1, 3.2);

		% Second node
		\draw [fill=black]
		 	  (2, 3.2) circle [radius=1.2pt]
			         node [above right] {$min_2$};

		% Third rectangle
		\draw [red]
		      (3, 0)
		   -- (3, 2.2)
		   -- (2, 2.2);

		% Third node
		\draw [fill=black]
		 	  (3, 2.2) circle [radius=1.2pt]
			           node [above right] {$min_3$};

		% Fourth rectangle
		\draw [red]
		      (4, 0)
		   -- (4, 0.8)
		   -- (3, 0.8);

		% Fourth node
		\draw [fill=black]
		 	  (4, 0.8) circle [radius=1.2pt]
			           node [above right] {$min_4$};

	\end{tikzpicture}
\end{figure}

F\"{u}r die Untersumme $A_U$ wird der kleinste Funktionswert $min_i$ jedes Intervalls als L\"{a}nge genommen: $$A_U = \sum_{i=1}^{N} f(min_i) \cdot \Delta x$$ F\"{a}llt die Funktion, liegt der kleinste Funktionswert jedes Intervalls meist am rechten Ende. Steigt die Funktion, bestimmt bei der Untersumme meist der Funktionswert der linken Intervallsgrenze die L\"{a}nge des Rechtecks. Die Breite $\Delta x$ ist bei allen Produktsummen ein konstanter Faktor, mit ihm kann also auch nachtr\"{a}glich multipliziert werden: $$A_U = \Delta x \cdot \sum_{i=1}^{N} f(min_i)$$

\pagebreak

\subsub{Obersumme}

\begin{figure}[h!]
	\centering
	\begin{tikzpicture}

		% x axis
		\draw [->]
		      (-0.5, 0) 
		   -- (+4.5, 0) node [pos=0.06, below] {0}
		   			    node [above] {$x$}
		   				node [pos=0.2, below] {$\Delta x$}
		   				node [pos=0.4, below] {$\Delta x$}
		   				node [pos=0.6, below] {$\Delta x$};

		% y axis
		\draw [->]
		      (0, -0.5)
		   -- (0, +4.5) node [right] {$f(x)$};

		% Graph
		\draw [domain=0:4] plot(\x, {0.25 * (\x - 4)^2});

		% First rectangle
		\draw [red]
		      (1, 0)
		   -- (1, 4)
		   -- (0, 4);

		% First node
		\draw [fill=black]
		 	  (0, 4) circle [radius=1.2pt]
		   			 node [left] {$max_1$};

		% Second rectangle
		\draw [red]
		      (2, 0)
		   -- (2, 2.25)
		   -- (1, 2.25);

		% Second node
		\draw [fill=black]
		 	  (1, 2.25) circle [radius=1.2pt]
			            node [left] {$max_2$};

		% Third rectangle
		\draw [red]
		      (3, 0)
		   -- (3, 1)
		   -- (2, 1);

		% Third node
		\draw [fill=black]
		 	  (2, 1) circle [radius=1.2pt]
			         node [left] {$max_3$};

	\end{tikzpicture}
	%
	\hspace{1.5cm}
	%
	\begin{tikzpicture}

	% x axis
		\draw [->]
		      (-0.5, 0) 
		   -- (+4.5, 0) node [pos=0.06, below] {0}
		   			    node [above] {$x$}
		   				node [pos=0.2, below] {$\Delta x$}
		   				node [pos=0.4, below] {$\Delta x$}
		   				node [pos=0.6, below] {$\Delta x$}
		   				node [pos=0.8, below] {$\Delta x$};

		% y axis
		\draw [->]
		      (0, -0.5)
		   -- (0, +4.5) node [right] {$f(x)$};

		% Graph
		\draw [domain=0:4] plot(\x, {-0.2 * \x^2 + 4});

		% First rectangle
		\draw [red]
		      (1, 0)
		   -- (1, 4)
		   -- (0, 4);

		% First node
		\draw [fill=black]
		 	  (0, 4) circle [radius=1.2pt]
		   		     node [left] {$max_1$};

		% Second rectangle
		\draw [red]
		      (2, 0)
		   -- (2, 3.8)
		   -- (1, 3.8);

		% Second node
		\draw [fill=black]
		 	  (1, 3.8) circle [radius=1.2pt]
			         node [above right] {$max_2$};

		% Third rectangle
		\draw [red]
		      (3, 0)
		   -- (3, 3.2)
		   -- (2, 3.2);

		% Third node
		\draw [fill=black]
		 	  (2, 3.2) circle [radius=1.2pt]
			           node [above right] {$max_3$};

		% Fourth rectangle
		\draw [red]
		      (4, 0)
		   -- (4, 2.2)
		   -- (3, 2.2);

		% Fourth node
		\draw [fill=black]
		 	  (3, 2.2) circle [radius=1.2pt]
			           node [above right] {$max_4$};

	\end{tikzpicture}
\end{figure}

Gegens\"{a}tzlich zur Untersumme $A_U$ bestimmt bei der Obersumme $A_O$ nicht der kleinste Funktionswert jeden Intervalls die L\"{a}nge des Rechtecks, sondern der gr\"{o}\ss{}te Funktionswert: $$A = \Delta x \cdot \sum_{i=1}^{N} f(max_i)$$ Man bemerke, dass die Rechtecke der Obersumme jenen der Untersumme stark \"{a}hneln. So ergibt es sich, dass die Obersumme berechnet werden kann, indem man die Rechtecke der Untersumme um ein $\Delta x$ verschiebt, das letzte Untersummenrechteck verwirft und das erste Rechteck mit dem maximalen Intervallswert neu berechnet ($min_N$ ist die Minimalstelle des letzten von $N$ Intervallen):

\begin{figure}[h!]
	\centering
	\begin{tikzpicture}

		% x axis
		\draw [->]
		      (-0.5, 0) 
		   -- (+4.5, 0) node [pos=0.06, below] {0}
		   			    node [above] {$x$}
		   				node [pos=0.4, below] {$\Delta x$}
		   				node [pos=0.6, below] {$\Delta x$}
		   				node [pos=0.8, below] {$\Delta x$};

		% y axis
		\draw [->]
		      (0, -0.5)
		   -- (0, +4.5) node [right] {$f(x)$};

		% Graph
		\draw [domain=0:4] plot(\x, {0.25 * \x^2});

		% First rectangle
		\draw [red]
		      (1, 0)
		   -- (1, 0.25)
		   -- (2, 0.25);

		% First node
		\draw [fill=black]
		 	  (1, 0.25) circle [radius=1.2pt]
		   			    node [above left] {$min_1$};

		% Second rectangle
		\draw [red]
		      (2, 0)
		   -- (2, 1)
		   -- (3, 1);

		% Second node
		\draw [fill=black]
		 	  (2, 1) circle [radius=1.2pt]
			         node [left] {$min_2$};

		% Third rectangle
		\draw [red]
		      (3, 0)
		   -- (3, 2.25)
		   -- (4, 2.25)
		   -- (4, 0);

		% Third node
		\draw [fill=black]
		 	  (3, 2.25) circle [radius=1.2pt]
			         node [left] {$min_3$};

	\end{tikzpicture}
	%
	\begin{tikzpicture}[font=\Large]
		% y scaling
		\draw (0, -0.5) (0, 4.5);

		% x scaling
		\draw (-1, 0) (1, 0);

		\draw (0, 2) node {$\Rightarrow$};
	\end{tikzpicture}
	%
	\begin{tikzpicture}

		% x axis
		\draw [->]
		      (-0.5, 0) 
		   -- (+4.5, 0) node [pos=0.06, below] {0}
		   			    node [above] {$x$}
		   			    node [pos=0.2, below] {$\Delta x$}
		   				node [pos=0.4, below] {$\Delta x$}
		   				node [pos=0.6, below] {$\Delta x$}
		   				node [pos=0.8, below] {$\Delta x$};

		% y axis
		\draw [->]
		      (0, -0.5)
		   -- (0, +4.5) node [right] {$f(x)$};

		% Graph
		\draw [domain=0:4] plot(\x, {0.25 * \x^2});

		% First rectangle
		\draw [red]
		      (0, 0.25)
		   -- (1, 0.25);

		% First node
		\draw [fill=black]
		 	  (1, 0.25) circle [radius=1.2pt]
		   			 node [above left] {$min_1$};

		% Second rectangle
		\draw [red]
		      (1, 0)
		   -- (1, 1)
		   -- (2, 1);

		% Second node
		\draw [fill=black]
		 	  (2, 1) circle [radius=1.2pt]
		   			 node [above left] {$min_2$};

		% Third rectangle
		\draw [red]
		      (2, 0)
		   -- (2, 2.25)
		   -- (3, 2.25)
		   -- (3, 0);

		% Third node
		\draw [fill=black]
		 	  (3, 2.25) circle [radius=1.2pt]
			            node [above left] {$min_3$};

		% Fourth rectangle
		\draw [blue]
		      (3, 2.25)
		   -- (3, 4)
		   -- (4, 4)
		   -- (4, 0);

		% Fourth node
		\draw [fill=black]
		 	  (4, 4) circle [radius=1.2pt]
			         node [above left] {$max_4$};

	\end{tikzpicture}
\end{figure}

$$A_O = A_U - \Delta x \cdot f(min_N) + \Delta x \cdot f(max_1)$$ $$\Downarrow$$ $$A_O = A_U + \Delta x \cdot (f(max_1) - f(min_N))$$

Da die Obersumme h\"{o}her ist als die wirkliche Fl\"{a}che unter der Kurve und die Untersumme niedriger, kann der Fl\"{a}cheninhalt der Kurve mit Relationszeichen zwischen Unter- und Obersumme eingeschr\"{a}nkt werden: $$A_U < A < A_O$$

\pagebreak

\subsub{Linkssumme}

Die Linkssumme unterscheidet nicht zwischen minimalen oder maximalen Funktionswerten, sondern setzt die L\"{a}nge jeden Rechtecks stets an das linke Ende jeden Intervalls:

\begin{figure}[h!]
	\centering
	\begin{tikzpicture}[scale=1.5]

		% x axis
		\draw [->]
		      (-0.5, 0) 
		   -- (+3.0, 0) node [midway, below] {$A_L$}
		   				node [above] {$x$};

		% y axis
		\draw [->]
		      (0, -0.5)
		   -- (0, +3.0) node [right] {$f(x)$};

		% Graph
		\draw [domain=0:2.5] plot(\x, {\x^3 - 4.5*\x^2 + 6 * \x});

		% Rectangles
		\foreach \x in {0.5,1,...,2}
		{
			\draw [red]
				  (\x, {(\x-0.5)^3 - 4.5*(\x-0.5)^2 + 6 * (\x-0.5)})
			   -- (\x, {\x^3 - 4.5*\x^2 + 6 * \x})
			   -- (\x + 0.5, {\x^3 - 4.5*\x^2 + 6 * \x})
			   -- (\x + 0.5, 0);


			\draw [fill=black] (\x, {\x^3 - 4.5*\x^2 + 6 * \x})
						       circle [radius=0.9pt];
		}

	\end{tikzpicture}
\end{figure}

$$A_L = \Delta x \cdot \sum_{i=0}^{N - 1} f(i \cdot \Delta x)$$

\subsub{Rechtssumme}

Zur Berechnung der Rechtssumme wird die L\"{a}nge jeden Rechtecks immer am Funktionswert der rechten Intervallsgrenze gemessen:

\begin{figure}[h!]
	\centering
	\begin{tikzpicture}[scale=1.5]

		% x axis
		\draw [->]
		      (-0.5, 0) 
		   -- (+3.0, 0) node [midway, below] {$A_R$}
		   				node [above] {$x$};

		% y axis
		\draw [->]
		      (0, -0.5)
		   -- (0, +3.0) node [right] {$f(x)$};

		% Graph
		\draw [domain=0:2.5] plot(\x, {\x^3 - 4.5*\x^2 + 6 * \x});

		% Rectangles
		\foreach \x in {0.5,1,...,2.5}
		{
			\draw [red]
				  (\x - 0.5, {(\x-0.5)^3 - 4.5*(\x-0.5)^2 + 6 * (\x-0.5)})
			   -- (\x - 0.5, {\x^3 - 4.5*\x^2 + 6 * \x})
			   -- (\x, {\x^3 - 4.5*\x^2 + 6 * \x})
			   -- (\x, 0);

			   \draw [fill=black] (\x, {\x^3 - 4.5*\x^2 + 6 * \x})
						          circle [radius=0.9pt];
		}

	\end{tikzpicture}
\end{figure}

$$A_R = \Delta x \cdot \sum_{i=1}^{N} f(i \cdot \Delta x)$$

\pagebreak

\sub{Das Bestimmte Integral}

Ebenso wie der Differentialquotient, also die momentane \"{A}nderung einer Funktion an einer bestimmten Stelle $x$, ein Grenzwert von Differenzenquotienten ist, in dem das Intervall [$x$ ; $x + \Delta x$] des Differenzenquotienten infinitisemal klein ist und $\Delta x$ somit gegen null strebt, ist auch ein Integral das Resultat eines Grenzwerts. W\"{a}chst n\"{a}mlich die Anzahl $N$ an Rechtecken, mit denen man die Fl\"{a}che unter einer Kurve als Produktsumme ann\"{a}hert, wird $\Delta x$ immer kleiner und diese Ann\"{a}herung somit immer genauer. Bei einer sehr gro\ss{}en Anzahl an Rechtecken $N$ konvergieren also Minimum und Maximum, Untersumme und Obersumme jeden Intervalls in einem Wert. Das bestimmte Integral, welches die Fl\"{a}che unter einer Funktion in einem bestimmten Intervall $[a ; b]$ exakt berechnet, l\"{a}sst sich also mit Hilfe eines Limes und den vorhin beschriebenen Produktsummen definieren: $$\int_{a}^{b} f(x) \,dx = \lim_{N \rightarrow \infty} \sum_{i=1}^{N} f(x_i) \cdot \Delta x = \lim_{\Delta x \rightarrow 0} \sum_{i=1}^{N} f(x_i) \cdot \Delta x$$ Bei einem bestimmten Integral $\int_{a}^{b} f(x) \, dx$ beschreiben $a$ und $b$ somit die untere sowie obere Intervallsgrenze. $f(x)$ ist die im Intervall $[a ; b]$ zu integrierende Funktion --- der Integrand. Der Ausdruck $dx$ bestimmt die Variable, nach der man integriert --- in diesem Fall $x$. Der Ausdruck $dx$ sollte immer angegeben werden und ist besonders f\"{u}r das Integral multivariabler Funktionen wichtig: $$\int_{a}^{b} f(x) = \int_{a}^{b} (2x^3 - 4z + y) \rightarrow \text{ Integration nach welcher Variable?}$$ 

Es sei angemerkt, dass Fl\"{a}chenst\"{u}cke unter der der Abszisse ein negatives Vorzeichen besitzen. Soll also die absolute Fl\"{a}che, die eine Funktion mit der Absizsse einschlie\ss{}t, berechnet werden, muss der Betrag aller negativen Fl\"{a}chensegmente verwendet werden. Ist jedoch nach dem Wert des bestimmten Integral in diesem Intervall gefragt, nicht nach der Fl\"{a}che, so m\"{u}ssen alle Vorzeichen --- positiv und negativ --- beibehalten werden:

\begin{figure}[h!]
	\centering
	\begin{tikzpicture}

		\begin{axis}
		[
			xlabel=$x$,
			ylabel=$f(x)$,
			samples=1000,
			axis lines = middle
		]

			\addplot [domain=-0.2:2.2] {(\x - 1)^3 -\x + 1};

		\end{axis}

		\draw (2, 3.75) node {\Large $A_1$};

		\draw (5, 1.75) node {\Large $A_2$};

	\end{tikzpicture}
\end{figure}

Absolute mit der $x$-Achse eingeschlossene Fl\"{a}che $A$ der Funktion: $$A = |A_1| + |A_2| = \left|\,\int_{0}^{1}f(x)\,dx\,\right| + \left|\,\int_{1}^{2}f(x)\,dx\,\right|$$ Bestimmtes Integral: $\int_{0}^{2} f(x)\,dx = A_1 + A_2$

\pagebreak

\sub{Stammfunktionen}

Ein wichtiger Begriff im Zusammenhang mit der Integralrechnung ist die sogennante \zitat{Stammfunktion} einer Funktion, welche eine genau Berechnung des bestimmten Integral erm\"{o}glicht. Da die Stammfunktion $F(x)$ das Resultat der Integration --- dem Gegenteil der Differenzierung --- einer Funktion $f(x)$ ist, ist die erste Ableitung $F'(x)$ der Stammfunktion wiederum die urspr\"{u}ngliche Funktion $f(x)$. Eine Stammfunktion $F(x)$, welche equivalent zum \emph{unbestimmten} Integral ist, wird als Integral ohne Integrationsgrenzen angeschrieben: $$F(x) = \int f(x)\,dx$$ Da Konstanten bei der Differenzierung einer Funktion wegfallen, besitzt jede Funkion $f(x)$ nicht nur eine, sondern unendlich viele Stammfunktionen, welche sich nur durch eine reelle Integrationskonstante $C$ voneinander unterscheiden. Sind $F(x)$ und $G(x)$ zwei Stammfunktionen einer Funktion $f(x)$, gilt somit: $$F'(x) = G'(x) = f(x)$$ Eine besondere Eigenschaft von Stammfunktionen ist, dass sie die Berechnung des bestimmten Integrals einer Funktion in einem beliebigen Intervall $[a ; b]$ erm\"{o}glichen. Der vorhin vorgestellte Grenzwert aus Produktsummen als Definition des Integrals liefert noch keine eindeutige Rechenmethode zur genauen Bestimmung seines Wertes. Kennt man eine Stammfunktion $F(x)$ einer Funktion $f(x)$, so ist das bestimmte Integral dieser Funktion in einem beliebigen Intervall $[a ; b]$ die Differenz der Funktionswerte der Stammfunktion an diesen Stellen. Die Integration einer Funktion in den Grenzen von $a$ nach $b$ wird hierbei durch einen vertikalen Strich angedeutet. $$\int_a^b f(x) \, dx = F(x) \, \Big|_a^b F(b) - F(a)$$

\sub{Integrationsregeln}

Aus der Definition des Integrals, der Stammfunktion sowie dem Verh\"{a}ltnis zwischen Differential- und Integralrechnung lassen sich einige Regeln der Integration festlegen. Allen voran steht die \emph{Potenzregel}, welche die Bestimmung der Stammfunktion $F(x)$ einer Funktion $f(x)$ erm\"{o}glicht. M\"{o}chte man einen Term des Schemas $x^n$ integrieren, bzw. f\"{u}r diesen eine Stammfunktion finden, so erh\"{o}ht man die Potenz der unabh\"{a}ngigen Variable um $1$ und dividiert den Term dadurch: $$\int (x^n) \, dx = \frac{x^{n + 1}}{n + 1}$$ Da konstante Werte wie die Zahl $7$ oder $\pi$ Koeffizienten $k$ eines Terms der Form $k \cdot x^0$ sind, fallen sie nicht wie bei der Differenzialrechnung weg, sondern werden Koeffizienten der unabh\"{a}ngigen Variable $x$: $$\int 7 \, dx = \int (7 \cdot x^0) \, dx = 7 \cdot \frac{x^1}{1} = 7x$$ 

\textbf{Beispiel}: \emph{Integriere die Funktion $f(x) = 3x^4 + 6x^2 -7x + 1$.}

\begin{center}
	$\int 3x^4 \, dx = \frac{3}{5}x^5$ \hspace{1cm} $\int 6x^2 \, dx = \frac{6}{3}x^3 = 2x^3$ \hspace{1cm} $\int -7x \, dx = -\frac{7}{2}x^2$ \hspace{1cm} $\int 1 \, dx = \frac{1}{1}x = x$
\end{center}

\textbf{Ergebnis}: $\int f(x) \, dx = F(x) = \frac{3}{5}x^5 + 2x^3 -\frac{7}{2}x^2 + x$

\begin{itemize}
	\titleitem{Summenregel} 

	Das bestimmte Integral der Summe aus zwei Funktionen $f(x)$ und $g(x)$ in einem beliebigen Intervall $[a ; b]$ ist gleich der Summe der einzelnen Integrale: $$\int_a^b \big[f(x) + g(x)\big] \,dx = \int_a^b f(x) \, dx + \int_a^b g(x) \, dx$$ Auch allgemein gilt dass die Stammfunktion einer Summe aus zwei Funktionen gleich der Summe ihrer Stammfunktionen ist: $$\int \big[f(x) + g(x)\big]\,dx = F(x) + G(x)$$

	\titleitem{Faktorregel}

	Konstante Faktoren $k \in \mathbb{R}$ einer Funktion $f(x)$ bleiben beim Integrieren sowie beim Bilden einer Stammfunktion erhalten: $$\int_a^b \big[k \cdot f(x)\big] \, dx = k \cdot \int_a^b f(x) \, dx$$ $$\int \big[k \cdot f(x)\big] \, dx = k \cdot F(x)$$

	\titleitem{Kombinationsregel}

	Die Summe bestimmter Integrale einer Funktion in zwei benachbarten Intervallen $[a ; b]$ $[b ; c]$ ist gleich dem bestimmten Integral im gesamten Intervall $[a ; c]$: $$\int_a^b f(x) \, dx + \int_b^c f(x) \, dx = \int_a^c f(x)\, dx$$ Dies kann auch auf beliebig viele benachbarte Intervalle mit den jeweiligen Intervallsgrenzen $a_i$ und $b_i$ des $i$-ten Intervalls erweitert werden. Sei $N$ die Anzahl der benachbarten Teilintervalle: $$\sum_{i=1}^{N} \int_{a_i}^{b_i} f(x) \, dx = \int_{a_1}^{b_N} f(x) \, dx$$

	\titleitem{Nullregel}

	Ist die Differenz $b - a$ zwischen den Integrationsgrenzen $a$ und $b$ gleich null, so ist das Integral der Funktion in diesen Grenzen gleich null: $$b - a = 0 \rightarrow \int_{a}^{b} f(x) \, dx = 0$$

	\titleitem{Vertauschungsregel}

	Kehrt man die Integrationsgrenzen $a$ und $b$ eines bestimmten Integrals einer Funktion um, so \"{a}ndert sich das Vorzeichen des Integrals in diesen Grenzen: $$\int_a^b f(x) \, dx = -\int_b^a f(x) \, dx$$

	Dies folgt daraus, dass das bestimmte Integral einer Funktion in einem Intervall $[a ; b]$ die Differenz zwischen den Funktionswerten der Stammfunktion an den Stellen $a$ und $b$ ist. Kehrt man die Integrationsgrenzen um, werden Minuend und Subtrahend dieser Differenz vertauscht: $$\int_a^b f(x) \, dx = F(b) - F(a)$$ $$\int_b^a f(x) \, dx = F(a) - F(b)$$

\end{itemize}

\pagebreak

\sub{Integrale elementarer Funktionen}

Nicht alle Integrale sind leicht zu bestimmen, vor allem jene elementarer Funktionen wie der Sinus-, Cosinus-, Wurzel-, Logarithmus- oder Euler'schen Funktion. Die Ableitung dieser Funktionen zu finden ist meist leichter.

\vspace{\parskip}

\begin{table}[h!]
	\centering
	\large
	\begin{tabular}{| >{\centering\arraybackslash}p{1.5cm}
	 				  c
	 				  >{\centering\arraybackslash}p{1.5cm} |}
		\hline
		&& \\
		$f(x)$ & $\longleftrightarrow$ & $F(x)$
		\\ && \\ \hline && \\
		$e^x$ & $\longleftrightarrow$ & $e^x$
		\\ && \\
		$x^{-1}$ & $\longleftrightarrow$ & $\ln x$
		\\ && \\
		$\sqrt{x}$ & $\longleftrightarrow$ & $\frac{2}{3}\sqrt{x^3}$
		\\ && \\
		$sin(x)$ & $\longleftrightarrow$ & $-\cos(x)$
		\\ && \\
		$\cos(x)$ & $\longleftrightarrow$ & $\sin(x)$
		\\ && \\
		$\frac{1}{\cos^2 x}$ & $\longleftrightarrow$ & $\tan x$
		\\ && \\
		$a^x$ & $\longleftrightarrow$ & $\frac{a^x}{\ln a}$
		\\ && \\
		$e^{ax}$ & $\longleftrightarrow$ & $\frac{e^{ax}}{a}$
		\\ && \\
		$x^n$ & $\longleftrightarrow$ & $\frac{x^{n + 1}}{n + 1}$
		\\ && \\
		\hline
	\end{tabular}
\end{table}

\vspace{\parskip}

Die Integration der Wurzelfunktion $\sqrt{x}$ sollte erl\"{a}utert werden. Jede Wurzel der Form $\sqrt[m]{x^n}$ kann als Potenz der Form $x^{\frac{n}{m}}$ angeschrieben werden. Daher w\"{a}re die Quadratwurzel von $x$ gleich $x$ hoch $\frac{1}{2}$. Wendet man nun die Potenzregel an, erh\"{o}ht man die Potenz des Terms um 1 und dividiert schlie\ss{}lich dadurch: $$\int \sqrt{x} \,dx = \int x^{\frac{1}{2}} \,dx = \cfrac{x^{\frac{3}{2}}}{\cfrac{1}{2} + \cfrac{2}{2}} = \cfrac{\sqrt{x^3}}{\cfrac{3}{2}} = \frac{2}{3} \sqrt{x^3}$$

\pagebreak

\sub{Berechnung von eingeschlossenen Fl\"{a}chen}

Gilt es, die Fl\"{a}che zwischen zwei beliebigen Funktionen berechnen, so muss man die Differenz ihrer Integrale bestimmen. Hierbei sei angemerkt, dass die berechnete, eingeschlossene Fl\"{a}che dann positiv ist, wenn der kleinere Fl\"{a}cheninhalt (der \emph{unteren} Funktion) vom gr\"{o}\ss{}eren Fl\"{a}cheninhalt (der \emph{oberen} Funktion) abgezogen wird.Andernfalls muss der Betrag des negativen Fl\"{a}chenst\"{u}cks verwendet werden --- die absolute Fl\"{a}che bleibt gleich. Ebenso sei angemerkt, dass stets von Schnittpunkt zu Schnittpunkt integriert werden sollte, da sich die Lagebeziehungen der Funktionen ver\"{a}ndern k\"{o}nnen und somit das Vorzeichen der einzelnen Teilintegrale. F\"{u}r die Berechnung der von zwei Funktionen $f(x)$ und $g(x)$ umschlossenen Fl\"{a}che $A$ gilt somit: $$A = \left| \, \int_a^b \big[f(x) - g(x)\big] \, dx \, \right|$$

\textbf{Beispiel}: \emph{Berechne die Fl\"{a}che, die von den Funktionen $f(x) = x^3 - 2x^2 + 2$ und $g(x) = x + 1$ eingeschlossen wird.}

\extrapar

\begin{figure}[h!]
	\centering
	\begin{tikzpicture}
		\begin{axis}
		[
			x=3cm,
			xlabel = $x$,
			ylabel = $f(x)$,
			axis lines = middle,
			legend pos = outer north east,
			samples = 100,
			domain=-0.9:2.4,
			mark size = 1.2pt
		]
			% f(x)
			\addplot [red] {x^3 - 2*x^2 + 2};

			\addlegendentry{$f(x) = x^3 - 2x^2 + 2$};

			% g(x)
			\addplot [black] {x + 1};

			\addlegendentry{$g(x) = x + 1$};

			% Intersection point 1
			\addplot [mark=*] coordinates {(-0.8, 0.2)};

			% Intersection point 2
			\addplot [mark=*] coordinates {(0.55, 1.55)};

			% Intersection point 3
			\addplot [mark=*] coordinates {(2.25, 3.25)};

		\end{axis}
	\end{tikzpicture}
\end{figure}

Der Grafik kann man entnehmen, dass die beiden Funktionen drei Schnittpunkte besitzen und daher zwei Fl\"{a}chenst\"{u}cke einschlie\ss{}en. Das erste Fl\"{a}chenst\"{u}ck $A_1$ wird von den Funktionen im Intervall $[-0.8 ; 0.55]$ umschlossen. Das zweite Fl\"{a}chenst\"{u}ck $A_2$ liegt im Intervall $[0.55 ; 2.25]$. Diese Intervallsgrenzen legt man als Integrationsgrenzen fest und berechnet anschlie\ss{}end das bestimmte Integral der Differenz der beiden Funktionen in diesen Grenzen. Addiert man die beiden positiven Teilfl\"{a}chen $A_1$ und $A_2$ erh\"{a}lt man die gesamte Fl\"{a}che $A$. Im ersten Intervall sollte das Integral von $g(x) - f(x)$ bestimmt werden, um somit eine positive Fl\"{a}che zu erhalten. Konvers ist es im zweiten Intervall ratsam, $f(x)$ von $g(x)$ zu subtrahieren. 

In beiden F\"{a}llen kann ebenso der Betrag des Integrals verwendet werden, dann spielen die Anordnung von Minuend und Subtrahent keine Rolle. Ebenso kann man sich durch Verwendung des Betrags Rechenarbeit sparen, da es gen\"{u}gt, eine Stammfunktion $H(x) = f(x) - g(x)$ zu bestimmten, welche dann mittels dem Betrag f\"{u}r beide Teilintegrale g\"{u}ltig ist.

Berechnung der Stammfunktion $H(x)$ von $f(x) - g(x)$: 

$$H(x) = \int \big[ f(x) - g(x) \big] \, dx = \int \big[ (x^3 - 2x^2 + 2) - (x + 1)\big] \, dx = \frac{1}{4}x^4 + \frac{2}{3}x^3 - \frac{1}{2}x^2 + x$$

\pagebreak

Anschlie\ss{}end kann das erste Teilintegral $A_1$ in den Grenzen $[-0.8 ; 0.55]$ bestimmt werden: $$A_1 = \int_{-0.8}^{0.55} \big[ f(x) - g(x)\big] \, dx = H(x) \, \Big|_{-0.8}^{0.55} = H(0.55) - H(-0.8) \approx -1$$ Sowie das zweite Teilintegral $A_2$ in den Grenzen $[0.55 ; 2.25]$: $$A_2 = \int_{0.55}^{2.25} \big[ f(x) - g(x)\big] \, dx = H(x) \, \Big|_{0.55}^{2.25} = H(2.25) - H(0.55) \approx 1.8$$ Die gesamte, von den beiden Funktionen umschlossene Fl\"{a}che $A$ ist somit die Summe der beiden absoluten Teilfl\"{a}chen: $$A = | A_1 | + | A_2 | \approx 2.8$$

\sub{Volumsberechnungen}

Ein weiteres Anwendungsgebiet der Integralrechnung ist die Berechenung von Volumen. Ebenso wie das Integral dazu verwendet werden kann, eine Fl\"{a}che als Summe vieler kleiner Rechtecke zu bilden, kann das Volumen eines K\"{o}rpers als Summe der Volumina vieler kleiner Quader berechnet werden. Das Volumen eines Prismas oder Zylinders ist allgemein $G \cdot h$, wo $G$ die Grundfl\"{a}che und $h$ die H\"{o}he des Prismas bzw. des Zylinders ist. 

Bei der Fl\"{a}chenberechnung wurde das bestimmte Integral gebildet, indem die Breite der einzelnen Rechtecke der Produktsummen gegen Null strebte. Bei der Volumsberechnung hingegen strebt die H\"{o}he der einzelnen Quader gegen Null. Der andere Faktor dieser Produktsumme, die Querschnittsfl\"{a}che (bzw. Grundl\"{a}che jeden Quaders) $G$, muss als Funktion der $z$-Koordinate definiert werden. Dies folgt daraus, dass sich die Breite und L\"{a}nge der Querschnittsfl\"{a}che vieler K\"{o}rper (z.B. quadratische Pyramiden) mit der H\"{o}he ver\"{a}ndern. Das bestimmte Integral zur Berechnung des Volumens eines K\"{o}rpers in der H\"{o}he von $z = a$ bis $z = b$ sei somit definiert als: $$V = \lim_{\Delta z \rightarrow 0} \sum_{i=1}^{N} A(z_i) \cdot \Delta z = \int_a^b A(z) \, dz$$

Die folgende Abbildung zeigt eine quadratische Pyramide. Die Querschnittsfl\"{a}che $A(z)$, hier in Rot, wird mit ansteigendem $z$ kleiner. Die jeweilige H\"{o}he $h$ jeden Quaders strebt gegen null, ist also infinitisemal klein.

\begin{figure}[h!]

	\centering

	\begin{tikzpicture}
	[
		scale=1.5,
		declare function = {A(\z) = 2 - \z/2;}
	]

		\tikzstyle{every node}=[font=\scriptsize]

		\begin{axis}
		[	
			xlabel = $x$,
			ylabel = $y$,
			zlabel = $z$,
			ticks=none,
			axis lines = middle,
			samples = 25
		]

		% Pyramid, first part
		\addplot3 [mark=none]
		          coordinates {
		          				(-2, -2, 0)
		          				(+2, -2, 0)
		          				( 0,  0, 4)
		          				(-2, -2, 0)
		          				(-2, +2, 0)
		          				( 0,  0, 4)
		          				(+2, +2, 0)
		          				(+2, -2, 0)
		          			  };

		% Pyramid, second part
        \addplot3 [mark=none]
		          coordinates {
		          				(-2, +2, 0)
		          				(+2, +2, 0)
		          			  };

		\foreach \z in {0, 1, 2, 3}
		{
			\addplot3 [mark=none, fill=red, opacity=0.5]
			          coordinates {
			          				(-A(\z), -A(\z), \z)
			          				(-A(\z), +A(\z), \z)
			          				(+A(\z), +A(\z), \z)
			          				(+A(\z), -A(\z), \z)
			          				(-A(\z), -A(\z), \z)
			          			  };
		}

		% To extend domain
		\addplot3 [mark=none] coordinates {(-2.5, 0, 0) (2.5, 0, 0)};

		\addplot3 [mark=none] coordinates {(0, -2.5, 0) (0, 2.5, 0)};

		\addplot3 [mark=none] coordinates {(0, 0, 4.5)};

		\end{axis}

	\end{tikzpicture}

\end{figure}

\pagebreak

\textbf{Beispiel}: \emph{Berechne das Volumen eines Zirkuszelts, dessen quadratische Querschnittsfl\"{a}che in der H\"{o}he z eine Seitenl\"{a}nge nach $a(z) = 8 \cdot (4 - \sqrt{z})$ besitzt. Abmessungen in m.}

Zuerst muss eine Funktion $A(z)$ f\"{u}r die Querschnittsfl\"{a}che $A$ in der H\"{o}he $z$ gefunden werden. Da die Querschnittsfl\"{a}che quadratisch ist, ist sie das Quadrat ihrer Seitenl\"{a}nge $a$. Die Seitenl\"{a}nge des Zirkuszelts ist in Abh\"{a}ngigkeit der H\"{o}he laut Angabe definiert als $a(z) = 8 \cdot (4 - \sqrt{z})$. Somit gilt f\"{u}r die Querschnittsfl\"{a}che $A(z)$: $$A(z) = a(z)^2 = \left[ 8 \cdot (4 - \sqrt{z}) \right]^2 = 64 \cdot (16 -8\sqrt{z} + z)$$ Diese Querschnittsl\"{a}che kann nun nach $dz$ integriert werden, um so das Volumen des Zirkuszelts zu bestimmen. Daf\"{u}r m\"{u}ssen jedoch zuerst die Integrationsgrenzen gefunden werden. Die untere Grenze befindet sich bei $z = 0$, da das Zirkuszelt am Boden liegt. Die obere Grenze kann als $z = 16$ gefunden werden, da $a(16) = 0$. Das bestimmte Integral liefert nun das Volumen $V$: $$V = \int_{0}^{16} A(z) \, dz = \int_{0}^{16} \left[ 64 \cdot (16 -8\sqrt{z} + z) \right] \, dz = 64 \cdot \left( 16z -\frac{16}{3}\sqrt{z^3} + \frac{1}{2}z^2 \right) \, \bigg|_{0}^{16} \approx 2731$$

\textbf{Ergebnis}: Das Volumen des Zirkuszelts betr\"{a}gt $2731 \, m^3$.

\sub{Rotationsk\"{o}rper}

Eine besondere Form der Volumsberechnung besch\"{a}ftigt sich mit dem Finden von Volumina von Rotationsk\"{o}rpern. Rotationsk\"{o}rper entstehen, indem man eine Funktion, beispielsweise $y = \sqrt{x}$, um eine der beiden Achsen --- $x$ oder $y$ --- rotieren l\"{a}sst. Die Rotationsachse wird hierbei die \emph{Erzeugende} des Rotationsk\"{o}rpers genannt. Bei Rotationsk\"{o}rpern liegen den einzelnen Produktsummen keine Quader, sondern Zylinder vor. Das Volumen eines Zylinders ist in Abh\"{a}ngigkeit des Radius definiert als: $$V(r) = G \cdot h = r^2 \pi \cdot h$$ Die unabh\"{a}ngige Variable der Funktion $V(r)$ ist hierbei der Radius $r$ des Zylinders. Bei Rotationsk\"{o}rpern ist der Radius einzelner Zylinder jedoch selbst ebenso abh\"{a}ngig, da er sich mit steigendem $x$- bzw. $y$-Argument --- je nach Rotationsachse --- ver\"{a}ndert. Die H\"{o}he $h$ der Zylinder strebt beim bestimmten Integral wiederum gegen null. Diese H\"{o}he ist bei Rotationsk\"{o}rpern um die $x$-Achse gleich $\Delta x$ und bei Rotationsk\"{o}rpern um die $y$-Achse gleich $\Delta y$.

Rotiert die Funktion bzw. der K\"{o}rper um die $x$-Achse, wird das Volumen $V_x$ berechnet. Der Radius der einzelnen Zylinder ist hierbei equivalent zum Funktionswert $f(x)$ bzw. $y$ an jeder Stelle $x$ der Funktion. Im Intervall $[x_1 ; x_2 ]$ wird das Volumen eines um die $x$-Achse rotierenden Funktionsgraphen somit durch folgendes bestimmtes Integral berechnet: $$V_x = \lim_{\Delta x \rightarrow 0} \, \sum_{i=1}^{N} r_i^2 \pi \cdot \Delta x = \int_{x_1}^{x_2} (y^2 \pi) \, dx = \pi \cdot \int_{x_1}^{x_2} y^2 \, dx$$ 

Wird um die $y$-Achse rotiert, muss die Funktion zuerst nach der abh\"{a}ngigen Variable umgeformt werden. Bei der oben angegeben Wurzelfunktion w\"{a}re die nach $x$ umgeformte Funktion: $x = y^2$. Das Volumen eines um die $y$-Achse rotierenden K\"{o}rpers wird mit $V_y$ angegeben. In den Integrationsgrenzen von $y_1$ bis $y_2$ ist das bestimmte Integral eines um die $y$-Achse rotierenden Funktionsgraphen definiert durch: $$V_x = \lim_{\Delta y \rightarrow 0} \, \sum_{i=1}^{N} r_i^2 \pi \cdot \Delta y = \int_{y_1}^{y_2} (x^2 \pi) \, dy = \pi \cdot \int_{y_1}^{y_2} x^2 \, dy$$ 

\pagebreak

\textbf{Beispiel}: \emph{Die Kurve $f(x) = \sqrt{x}$ rotiert im Intervall $[0;4]$ um die $x$-Achse. Berechne das Volumen des daraus entstehenden Rotationsk\"{o}rpers.}

Da der Funktionsgraph um die $x$-Achse rotieren soll, ist der Radius der einzelnen Zylinder der Produktsumme bzw. des bestimmten Integrals gleich dem Funktionswert $f(x)$ an jeder Stelle $x \in [0 ; 4]$. Daher wird f\"{u}r die Querschnittsfl\"{a}chenformel (= Fl\"{a}chenformel eines Kreises) $A = r^2 \pi$ der Radius durch die Funktion $f(x)$ ersetzt, in Abh\"{a}ngigkeit des Fortschritts auf der $x$-Achse. Das Volumen des Rotationsk\"{o}rpers im Intervall $[0 ; 4]$ ist folglich das bestimmte Integral der Querschnittsfl\"{a}che in Abh\"{a}ngigkeit von $x$. Integriert wird nach $dx$, da die H\"{o}he der einzelnen Zylinder gegen Null strebt. Wird der Ausdruck $f(x)$ letztlich noch als $y$ angeschrieben, l\"{a}sst sich das Volumen so berechnen: $$V_x = \pi \cdot \int_0^4 y^2 \, dx = \pi \cdot \int_0^4 (\sqrt{x})^2 \, dx = \pi \cdot \int_0^4 x \, dx = \pi \cdot \frac{x^2}{2} \, \, \bigg|_0^4 = 8\pi$$

\textbf{Ergebnis}: Das Volumen des Rotationsk\"{o}rpers betr\"{a}gt $8\pi$ Volumseinheiten. 

\textbf{Beispiel}: \emph{Der Graph der Funktion $f(x) = 4 - 2x$ rotiert im Intervall $[0 ; 4]$ um die $y$-Achse. Berechne das Volumen des daraus entstehenden Rotationsk\"{o}rpers.}

Hier muss die Funktion $f(x)$ zuerst nach $x$ umgeformt werden, da der K\"{o}rper um die $y$-Achse rotiert. Das bedeutet, dass der Radius der Zylinder gleich der $x$-Variable ist und sich in Abh\"{a}ngigkeit der H\"{o}he bzw. des $y$-Arguments ver\"{a}ndert. Die nach $x$-umgeformte Funktion lautet: $$y = 4 - 2x \Rightarrow x = 2 - \frac{y}{2}$$ Anschlie\ss{}end kann auch gleich das Quadrat der Funktion berechnet werden, da dieses bei der Berechnung des Volumens ben\"{o}tigt wird: $$x^2 = \left(2 - \frac{y}{2}\right)^2 = -\frac{y^2}{4} - 2y + 4$$ Somit kann das Volumen des um die $y$-Achse rotierenden K\"{o}rpers berechnet werden: $$V_y = \pi \cdot \int_0^4 x^2 \, dy = \pi \cdot \int_0^4 \left( -\frac{y^2}{4} - 2y + 4 \right) \, dy = \pi \cdot \left( -\frac{y^3}{12} - y^2 + 4y \right) \,\, \bigg |_0^4 = \frac{324}{5} \pi $$

\textbf{Ergebnis}: Der Rotationsk\"{o}rper hat ein Volumen von $\frac{324}{5}\pi$ Volumseinheiten.

\pagebreak

\sub{Analysis}

In Verbindung mit der Analysis sei zur Integralrechnung gesagt, dass sie ein Ableitungsverfahren r\"{u}ckg\"{a}ngig macht. Man nehme eine Weg-Zeit Funktion $s(t)$. Diese hat als unabh\"{a}ngige Variable auf der Abszisse die verstrichene Zeit $t$ sowie als abh\"{a}ngige Variable auf der Ordinate den zur\"{u}ckgelegten Weg $s(t)$. Bildet man die erste Ableitung $s'(t)$, so ist dies die Geschwindigkeitsfunktion $v(t)$, welche die Ver\"{a}nderung des Weges relativ zur verstrichenen Zeit bzw. die momentane \"{A}nderungsrate der Weg-Zeit Funktion zu jedem Zeitpunkt $t$ beschreibt. Die unabh\"{a}ngige Variable der Funktion $v(t)$ ist immer noch die Zeit $t$, die abh\"{a}ngige Variable nun jedoch nicht mehr der zur\"{u}ckgelegte Weg $s$, sondern die Geschwindigkeit $s/t$. Die zweite Ableitung der Weg-Zeit Funktion nennt sich Beschleunigungsfunktion und wird mit $a(t)$ angegeben. Die Beschleunigungsfunktion beschreibt die Ver\"{a}nderung der Geschwindigkeit relativ zur Zeit, hat als abh\"{a}ngige Variable daher $s/t^2$. Daher wird klar, dass das Ableiten einer Funktion equivalent zur Division ihrer abh\"{a}ngigen Variable (hier $s$, $s/t$, $s/t^2$) durch ihre unabg\"{a}ngige Variable (hier $t$) ist: $$s'(t) = v(t) \rightarrow s/t$$

Ein Integrationsverfahren ist das genaue Gegenst\"{u}ck zum Ableiten einer Funktion. Integriert man eine Funktion $f(x)$, so kann man dies entweder als das Aufheben der letzten Ableitung oder als das Bilden einer Stammfunktion sehen. Leitet man eine Funktion $s(t)$ ab, wird ihre abh\"{a}ngige durch ihre unabh\"{a}ngige Variable dividiert. Integriert man eine Funktion hingegen, so wird ihre abh\"{a}ngige Variable mit ihrer unabh\"{a}ngigen \emph{multipliziert}: $$\int v(t) = s(t) \rightarrow s/t \cdot t = s$$ Einige bekannte und wichtige Zusammenh\"{a}nge, die sich so modellieren lassen, sind hier aufgelistet:

\begin{itemize}
	\item Weg $\longleftrightarrow$ Geschwindigkeit $\longleftrightarrow$ Beschleunigung

	\item Arbeit $\longleftrightarrow$ Leistung

	\item Ladung $\longleftrightarrow$ Stromst\"{a}rke

	\item Wassermenge $\longleftrightarrow$ Flussrate
\end{itemize}

\begin{figure}[h!]
	\centering
	\begin{tikzpicture}
		% t axis line
		\draw [->] (0, 0) -- (6, 0) node [above] {$t\, [s]$};

		% t axis legend
		\draw (3, -1) node {Zeit};

		% t axis ticks
		\foreach \t/\s in {0/4, 1/2, 2/2, 3/3, 4/3, 5/0}
		{
			\draw (\t, 0) node [below] {$\t$} -- (\t, \s);
		}

		% s axis line (rest is drawn above)
		\draw [->] (0, 4) -- (0, 4.5) node [right] {$v(t) \, [m/s]$};

		% s axis legend
		\draw (-1, 2.25) node [rotate=90] {Geschwindigkeit};

		% s axis ticks
		\foreach \s in {1, 2, 3, 4}
		{
			\draw (0,\s) node {$-$} node [left] {$\s$};
		}

 		% path
 		\draw (0, 4) -- (1, 2) -- (2, 2) -- (3, 3) -- (4, 3) -- (5, 0);

 		\draw (3, -1.9) node {$\int v(t) \rightarrow$ Zur\"{u}ckgelegter Weg in Metern $[m]$};
	\end{tikzpicture}
	%
	%
	%
	\begin{tikzpicture}
		% t axis line
		\draw [->] (0, 0) -- (6, 0) node [above] {$t\, [s]$};

		% t axis legend
		\draw (3, -1) node {Zeit};

		% t axis ticks
		\foreach \t/\s in {0/1, 1/1, 2/2, 3/2, 4/3, 5/3}
		{
			\draw (\t, 0) node [below] {$\t$} -- (\t, \s);
		}

		% P axis line (rest is drawn above)
		\draw [->] (0, 1) -- (0, 4.5) node [right] {$P(t) \, [J/s = W]$};

		% P axis legend
		\draw (-1, 2.25) node [rotate=90] {Leistung};

		% P axis ticks
		\foreach \s in {1, 2, 3, 4}
		{
			\draw (0,\s) node {$-$} node [left] {$\s$};
		}

 		% path
 		\draw (0, 1) -- (1, 1) -- (2, 2) -- (3, 2) -- (4, 3) -- (5, 3);

 		\draw (3, -1.9) node {$\int v(t) \rightarrow$ Vollbrachte Arbeit in Joule $[J]$};
	\end{tikzpicture}
\end{figure}

\end{document}