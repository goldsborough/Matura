% Differentialrechnung

\documentclass[11pt]{article}

\usepackage[a4paper, margin=1in]{geometry}

\usepackage{amsmath}

\usepackage{amssymb}

\usepackage[german]{babel}

\usepackage[autostyle=true]{csquotes}

\usepackage{libertine}

\usepackage[libertine]{newtxmath}

\usepackage{tikz}

\usepackage{gensymb}

\usepackage{fancyhdr}

\usepackage{amsfonts}

\usepackage{pgfplots}

\pgfplotsset{compat=1.10}

\usepackage{multicol}

\usepackage{caption}

\usepackage{floatrow}

\everymath{\displaystyle}

% Header / footer settings

\pagestyle{fancy}
\fancyhf{}
\renewcommand{\headrulewidth}{0.2mm}
\fancyhead[C]{Funktionen}
\renewcommand{\footrulewidth}{0.2mm}
\fancyfoot[L]{Peter Goldsborough}
\fancyfoot[C]{\thepage}
\fancyfoot[R]{\today}

\fancypagestyle{plain}{%
	\fancyhf{}
	\renewcommand{\headrulewidth}{0mm}%
	\renewcommand{\footrulewidth}{0.2mm}%
	\fancyfoot[L]{Peter Goldsborough}
	\fancyfoot[C]{\thepage}
	\fancyfoot[R]{\today}
}


\setlength{\headheight}{15pt}

\setlength{\parindent}{0pt}

\addtolength{\parskip}{\baselineskip}


\newcommand{\overbar}[1]{\mkern 1.5mu\overline{\mkern-1.5mu#1\mkern-1.5mu}\mkern 1.5mu}

\newcommand{\heading}[1]{\begin{center}\Huge \textbf{#1}\end{center}\par}

\newcommand{\sub}[1]{\vspace{\parskip}{\LARGE\textbf{#1}}}

\newcommand{\subsub}[1]{{\Large \textbf{#1}}}

\newcommand{\subsubsub}[1]{\textbf{#1}}

\newcommand{\colvec}[1]{\begin{pmatrix}#1\end{pmatrix}}

\newcommand{\extrapar}{\par\vspace{\baselineskip}}

\newcommand{\zitat}[1]{\foreignquote{german}{#1}}

\newcommand{\bolditem}[1]{\item \textbf{#1}}

\newcommand{\titleitem}[1]{\bolditem{#1}\par}

\newcommand{\defas}{ \dots \,\,}

\begin{document}
\thispagestyle{plain}

\heading{Differentialrechnung}

\sub{Definition}

Die Differentialrechnung besch\"{a}ftigt sich mit der Ver\"{a}nderung von unabh\"{a}ngigen und abh\"{a}ngigen Variablen einer Funktion in Abh\"{a}ngigkeit von einander. Wichtige Termini sind hierbei der Differenzenquotient, der die mittlere \"{A}nderungsrate bzw. durchschnittliche Steigung einer Funktion ausdr\"{u}ckt; der Differentialquotient, der die momentane \"{A}nderungsrate bzw. momentane Steigung beschreibt; sowie die Kurvendiskussion, bei welcher sich mit Hilfe von Differenzen- und Differentialquotient die Eigenschaften einer Funktion beschreiben lassen. 

Generell l\"{a}sst sich sagen, dass sich die Differentialrechnung mit \textbf{relativen} \"{A}nderungsma\ss{}en besch\"{a}ftigt (z.B. eine relative Zunahme des Weges um 2 Meter \emph{pro} Sekunde --- Variablen in Relation zu einander) und nicht mit \textbf{absoluten} \"{A}nderungsma\ss{}en (z.B. eine absolute Zunahme des Weges um 5 Meter nach 10 Sekunden --- keine Relation zwischen den Variablen).

\sub{Differenzenquotient}

Der Differenzenquotient $k = \frac{\Delta y}{\Delta x}$ einer Funktion $f(x)$ ist ein \textbf{relatives \"{A}nderungsma\ss{}}, welches f\"{u}r zwei beliebiege unabh\"{a}ngige Variablen $x_{0}$ und $x_{1}$ aus der Definitionsmenge $D_{f}$ und den beiden entsprechenden abh\"{a}ngigen Variablen $f(x_{0})$ bzw. $y_{0}$ und $f(x_{1})$ bzw. $y_{1}$ aus der Wertemenge $W_{f}$, die \"{A}nderung dieser beiden Variablen in Relation zu einander beschreibt. Der daraus resultierende Wert $k$ dr\"{u}ckt aus, um wieviele Einheiten sich $f(x)$ im Intervall $[x_{0} ; x_{1}]$ ver\"{a}ndert, wenn $x$ um eine Einheit w\"{a}chst oder f\"{a}llt. Der Differenzenquotient ist somit die \textbf{mittlere \"{A}nderungsrate} bzw. die \textbf{durschnittliche Steigung} in diesem Intervall. F\"{u}r den Differenzenquotient einer Funktion $f(x)$ im Intervall $[x_{0} ; x_{1}]$ gilt somit: $$k = \frac{\Delta f(x)}{\Delta x} = \frac{f(x_{1}) - f(x_{0})}{x_{1} - x_{0}}$$

Geometrisch gesehen l\"{a}sst sich der Differenzenquotient durch die \textbf{Sekantensteigung} modellieren. Die Sekantensteigung einer Funktion $f(x)$ ist die Hypotenuse des Steigungsdreiecks zwischen den Punkten $P_{0}(x_{0}|y_{0})$ und $P_{1}(x_{1}|y_{1})$.

\begin{figure}[h!]
\centering
	\begin{tikzpicture} 

		% x axis
		\draw [<->] 
		      (-1, 0) 
		   -- (5, 0) node [pos=0.167, below right] {0}
		   	         node [pos=0.333] {$|$}
		   	         node [pos=0.333, below right] {$x_{0}$}
		   			 node [pos=0.664] {$|$}
		   			 node [pos=0.664, below right] {$x_{1}$}
		   		     node [above] {x};

		% y axis
		\draw [<->]
		      (0, -1)
		   -- (0, 5) node [pos=0.333] {---}
		   		     node [pos=0.333, above left] {$y_{0}$}
		   			 node [pos=0.664] {---}
		   			 node [pos=0.664, above left] {$y_{1}$}
		   		     node [right] {y};

		% function
		\draw [samples=100, domain=-1:4]
		      plot (\x, {((\x * \x)/4) + 1});

		% steigungsdreieck
		\draw [ ]
			  (1, 1.25) circle [radius=1.2pt, fill=black]
		             	node [below] {$P_{0}$}
		   -- (3, 1.25) node [midway, below] {$\Delta x$}
		   -- (3, 3.25) circle [radius=1.2pt, fill=black]
		             	node [right] {$P_{1}$}
		             	node [midway, right] {$\Delta y$};

		% sekante
		\draw [red]
		      (-1, -0.75)
		   -- (4.75, 5) node [midway, above left]
		   	            {$k = \frac{\Delta y}{\Delta x}$};

	\end{tikzpicture}
\end{figure}

\pagebreak

\sub{Differentialquotient}

W\"{a}hrend der Differenzenquotient die durschnittliche Steigung in einem bestimmten Intervall $[x_{0};x_{1}]$ beschreibt, dr\"{u}ckt der Differentialquotient die \textbf{momentane Steigung} bzw. die \textbf{momentane \"{A}nderungsrate} einer Funktion an einer bestimmten Stelle $x$ aus. Der Differentialquotient an einer Stelle $x_{0}$ ist theoretisch gesehen ein Differenzenquotient in einem Intervall $[x_{0};x_{1}]$, in dem $x_{1}$ gegen $x_{0}$ und somit $\Delta x$ gegen $0$ strebt: $$\lim_{\Delta x\to0}\frac{\Delta f(x)}{\Delta x} = \frac{f(x + \Delta x) - f(x)}{(x + \Delta x) - x}$$

Geometrisch gesehen ist ein Differentialquotient einer Funktion $f(x)$ an einer Stelle $x_{0}$ die Steigung der Tangente an den Punkt $P(x_{0}|f(x_{0}))$. Diese Tangente entsteht durch eine Folge von Ann\"{a}herungen von Sekanten in einem immer kleiner werdenenden Intervall $[x_{0}, x_{1}]$. Somit kann der Differentialquotient bzw. die Tangentensteigung als Grenzwert von Sekantensteigungen gesehen werden.

\begin{figure}[h!]
\centering
	\begin{tikzpicture} 

		% x axis
		\draw [<->] 
		      (-1, 0) 
		   -- (5, 0) node [pos=0.167, below right] {0}
		   	         node [pos=0.333] {$|$}
		   	         node [pos=0.333, below right] {$x_{0}$}
		   		     node [above] {x};

		% y axis
		\draw [<->]
		      (0, -1)
		   -- (0, 5) node [pos=0.333] {---}
		   		     node [pos=0.333, above left] {$y_{0}$}
		   		     node [right] {y};

		% function
		\draw [samples=100, domain=-1:4.45]
		      plot (\x, {((\x * \x)/5) + 1});

		% sekanten points
		\draw (1, 1.25) circle [radius=1.2pt, fill=black]
						node [below] {$P_{0}$};	

		% sekanten
		\draw [blue] 
		      (-1, -0.717)
		   -- (4, 4.25) node [above left] {$P_{1}$}
		   -- (4.813, 5);

		\draw [magenta]
		      (-1, -0.3)
		   -- (3, 2.8) node [right] {$P_{2}$}
		   -- (5, 4.35);

		\draw [cyan]
		      (-1, 0.15)
		   -- (2, 1.8) node [below right] {$P_{3}$}
		   -- (5, 3.45);

	\end{tikzpicture}
\end{figure}

\sub{Ableiten}

\ldots

\end{document}