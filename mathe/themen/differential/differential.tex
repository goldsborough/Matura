% Differentialrechnung

\documentclass[11pt]{article}

\usepackage[a4paper, margin=1in]{geometry}

\usepackage{amsmath}

\usepackage{amssymb}

\usepackage[german]{babel}

\usepackage[autostyle=true]{csquotes}

\usepackage{libertine}

\usepackage[libertine]{newtxmath}

\usepackage{tikz}

\usepackage{gensymb}

\usepackage{fancyhdr}

\usepackage{amsfonts}

\usepackage{pgfplots}

\pgfplotsset{compat=1.10}

\usepackage{multicol}

\usepackage{caption}

\usepackage{floatrow}

\everymath{\displaystyle}

% Header / footer settings

\pagestyle{fancy}
\fancyhf{}
\renewcommand{\headrulewidth}{0.2mm}
\fancyhead[C]{Funktionen}
\renewcommand{\footrulewidth}{0.2mm}
\fancyfoot[L]{Peter Goldsborough}
\fancyfoot[C]{\thepage}
\fancyfoot[R]{\today}

\fancypagestyle{plain}{%
	\fancyhf{}
	\renewcommand{\headrulewidth}{0mm}%
	\renewcommand{\footrulewidth}{0.2mm}%
	\fancyfoot[L]{Peter Goldsborough}
	\fancyfoot[C]{\thepage}
	\fancyfoot[R]{\today}
}


\setlength{\headheight}{15pt}

\setlength{\parindent}{0pt}

\addtolength{\parskip}{\baselineskip}


\newcommand{\overbar}[1]{\mkern 1.5mu\overline{\mkern-1.5mu#1\mkern-1.5mu}\mkern 1.5mu}

\newcommand{\heading}[1]{\begin{center}\Huge \textbf{#1}\end{center}\par}

\newcommand{\sub}[1]{\vspace{\parskip}{\LARGE\textbf{#1}}}

\newcommand{\subsub}[1]{{\Large \textbf{#1}}}

\newcommand{\subsubsub}[1]{\textbf{#1}}

\newcommand{\colvec}[1]{\begin{pmatrix}#1\end{pmatrix}}

\newcommand{\extrapar}{\par\vspace{\baselineskip}}

\newcommand{\zitat}[1]{\foreignquote{german}{#1}}

\newcommand{\bolditem}[1]{\item \textbf{#1}}

\newcommand{\titleitem}[1]{\bolditem{#1}\par}

\newcommand{\defas}{ \dots \,\,}

\begin{document}
\thispagestyle{plain}

\heading{Differentialrechnung}

Die Differentialrechnung besch\"{a}ftigt sich mit der Ver\"{a}nderung von unabh\"{a}ngigen und abh\"{a}ngigen Variablen einer Funktion in Abh\"{a}ngigkeit von einander. Wichtige Termini sind hierbei der Differenzenquotient, der die mittlere \"{A}nderungsrate bzw. durchschnittliche Steigung einer Funktion ausdr\"{u}ckt; der Differentialquotient, der die momentane \"{A}nderungsrate bzw. momentane Steigung beschreibt; sowie die Kurvendiskussion, bei welcher sich mit Hilfe von Differenzen- und Differentialquotient die Eigenschaften einer Funktion beschreiben lassen.

\sub{Differenzenquotient}

Der Differenzenquotient $k = \frac{\Delta y}{\Delta x}$ einer Funktion $f(x)$ ist ein \textbf{absolutes \"{A}nderungsma\ss{}}, welches f\"{u}r zwei beliebiege unabh\"{a}ngige Variablen $x_{0}$ und $x_{1}$ aus der Definitionsmenge $\mathbb{D}_{f}$ und den beiden entsprechenden abh\"{a}ngigen Variablen $f(x_{0})$ bzw. $y_{0}$ und $f(x_{1})$ bzw. $y_{1}$ aus der Wertemenge $\mathbb{W}_{f}$, die \"{A}nderung dieser beiden Variablen in Relation zu einander beschreibt. Der daraus resultierende Wert $k$ dr\"{u}ckt aus, um wieviele Einheiten sich $f(x)$ im Intervall $[x_{0} ; x_{1}]$ durchschnittlich ver\"{a}ndert, wenn $x$ um eine Einheit w\"{a}chst. Der Differenzenquotient ist somit die \textbf{mittlere \"{A}nderungsrate} bzw. die \textbf{durschnittliche Steigung} in diesem Intervall. F\"{u}r den Differenzenquotient einer Funktion $f(x)$ im Intervall $[x_{0} ; x_{1}]$ gilt somit: $$k = \frac{\Delta f(x)}{\Delta x} = \frac{f(x_{1}) - f(x_{0})}{x_{1} - x_{0}}$$ Geometrisch gesehen l\"{a}sst sich der Differenzenquotient durch die \textbf{Sekantensteigung} ausdr\"{u}cken. Die Sekantensteigung ist die Steigung jener Gerade, die durch die Punkte $P_{0}(x_{0}|y_{0})$ und $P_{1}(x_{1}|y_{1})$ verl\"{a}uft.

\vspace{1cm}

\begin{figure}[h!]
\centering
	\begin{tikzpicture}[scale=1.5]

		% x axis
		\draw [<->] 
		      (-1, 0) 
		   -- (5, 0) node [pos=0.167, below right] {0}
		   	         node [pos=0.333] {$|$}
		   	         node [pos=0.333, below right] {$x_{0}$}
		   			 node [pos=0.664] {$|$}
		   			 node [pos=0.664, below right] {$x_{1}$}
		   		     node [above] {x};

		% y axis
		\draw [<->]
		      (0, -1)
		   -- (0, 5) node [pos=0.333] {---}
		   		     node [pos=0.333, above left] {$y_{0}$}
		   			 node [pos=0.664] {---}
		   			 node [pos=0.664, above left] {$y_{1}$}
		   		     node [right] {y};

		% function
		\draw [samples=100, domain=-1:4]
		      plot (\x, {((\x * \x)/4) + 1});

		% steigungsdreieck
		\draw [ ]
			  (1, 1.25) circle [radius=1.2pt, fill=black]
		             	node [below] {$P_{0}$}
		   -- (3, 1.25) node [midway, below] {$\Delta x$}
		   -- (3, 3.25) circle [radius=1.2pt, fill=black]
		             	node [right] {$P_{1}$}
		             	node [midway, right] {$\Delta y$};

		% sekante
		\draw [red]
		      (-1, -0.75)
		   -- (4.75, 5) node [midway, above left]
		   	            {$k = \frac{\Delta y}{\Delta x}$};

	\end{tikzpicture}
\end{figure}

\pagebreak

\sub{Differentialquotient}

W\"{a}hrend der Differenzenquotient die durschnittliche Steigung in einem bestimmten Intervall $[x_{0};x_{1}]$ beschreibt, dr\"{u}ckt der Differentialquotient die \textbf{momentane Steigung} bzw. die \textbf{momentane \"{A}nderungsrate} einer Funktion an einer bestimmten Stelle $x$ bzw. in einem bestimmten Punkt aus. Der Differentialquotient an einer Stelle $x_{0}$ ist theoretisch gesehen ein Differenzenquotient in einem Intervall $[x_{0};x_{1}]$, in dem $x_{1}$ gegen $x_{0}$ und somit $\Delta x$ gegen $0$ strebt: $$\lim_{\Delta x\to0}\frac{\Delta f(x)}{\Delta x} = \frac{f(x + \Delta x) - f(x)}{(x + \Delta x) - x}$$ Geometrisch gesehen ist ein Differentialquotient einer Funktion $f(x)$ an einer Stelle $x_{0}$ die Steigung der Tangente an den Punkt $P(x_{0}|f(x_{0}))$. Diese Tangente entsteht durch eine Folge von Ann\"{a}herungen von Sekanten in einem immer kleiner werdenenden Intervall $[x_{0}; x_{1}]$. Somit kann der Differentialquotient bzw. die Tangentensteigung als Grenzwert von Sekantensteigungen gesehen werden.

\begin{figure}[h!]
\centering
	\begin{tikzpicture} 

		% x axis
		\draw [<->] 
		      (-1, 0) 
		   -- (5, 0) node [pos=0.167, below right] {0}
		   	         node [pos=0.333] {$|$}
		   	         node [pos=0.333, below right] {$x_{0}$}
		   		     node [above] {x};

		% y axis
		\draw [<->]
		      (0, -1)
		   -- (0, 5) node [pos=0.333] {---}
		   		     node [pos=0.333, above left] {$y_{0}$}
		   		     node [right] {y};

		% function
		\draw [samples=100, domain=-1:4.45]
		      plot (\x, {((\x * \x)/5) + 1});

		% sekanten points
		\draw (1, 1.25) circle [radius=1.2pt, fill=black]
						node [below] {$P_{0}$};	

		% sekanten
		\draw [blue] 
		      (-1, -0.717)
		   -- (4, 4.25) node [above left] {$P_{1}$}
		   -- (4.813, 5);

		\draw [magenta]
		      (-1, -0.3)
		   -- (3, 2.8) node [right] {$P_{2}$}
		   -- (5, 4.35);

		\draw [cyan]
		      (-1, 0.15)
		   -- (2, 1.8) node [below right] {$P_{3}$}
		   -- (5, 3.45);

	\end{tikzpicture}
\end{figure}

\sub{Ableitung}

Die Ableitung einer Funktion $f(x)$ ordnet jedem unabh\"{a}ngigen $x$-Wert auf der Abszisse bzw. aus der Definitionsmenge $\mathbb{D}_f$ den Differentialquotienten an dieser Stelle als Funktionswert der Ableitungsfunktion $f'(x)$ zu. 

\vspace{1cm}

\begin{figure}[hb!]
\centering
	\begin{tikzpicture}
		\begin{axis}
		[
			xlabel = $x$,
			ylabel = $f(x)$,
			legend pos = outer north east,
			axis lines = middle,
			restrict y to domain = -10:10,
			samples = 100
		]

			\addplot [magenta] {x^3 - 3*x^2 - x + 3};

			\addlegendentry{$f(x) = x^3 - 3x^2 - x + 3$};

			\addplot [blue] {3*x^2 - 6*x - 1};

			\addlegendentry{$f'(x) = 3x^2 - 6x - 1$};

		\end{axis}
	\end{tikzpicture}
\end{figure}

\pagebreak

Um eine Funktion abzuleiten, gibt es mehrere Verfahren bzw. Regeln: die Potenzregel, die Kettenregel, die Quotientenregel sowie die Reziprokregel. Ebenso gibt es bestimmte Eigenheiten einiger Funktionstypen im Bezug auf ihre Ableitungen, welche genannt werden sollten.

\subsub{Potenzregel}

Die Potenzregel wird f\"{u}r Terme der Form $ax^b$ angewendet: $$[ax^b]' = b \cdot ax^{b-1}$$

\emph{Beispiel}: $f(x) = 3x^4 - 8x^3 - 5x^2 + 13x - 21$

\begin{tabular}{l l}
	I. Anwendung der Potenzregel beim ersten Term: & $[3x^4]' = 4 \cdot 3x^{4-1}  = 12x^3$
	\extrapar \\
	II. Anwendung der Potenzregel beim zweiten Term: & $[-8x^3]' = 4 \cdot (-8x^{3-1}) = 24x^2$
	\extrapar \\
	III. Anwendung der Potenzregel beim dritten Term & $[5x^2]' = 2 \cdot 5x^{2-1} = 10x$
	\extrapar \\
	IV. Anwendung der Potenzregel beim vierten Term: & $[13x]' = [13x^1]' = 1 \cdot 13x^{1 - 1} = 13$
	\extrapar \\
	V. Konstanter Term f\"{a}llt weg: & $[21]' = [21x^0]' = 0 \cdot 21x^{0-1} = 0$
\end{tabular}

\emph{Ergebnis}: $f'(x) = 12x^3 - 24x^2 - 10x + 13$

\subsub{Produktregel}

Die Produktregel wird zur Ableitung eines Produktes zweiter Funktion bzw. Funktionsgliedern $u(x)$ und $v(x)$ angewendet, in welchen beide Male die unabh\"{a}ngige Variable $x$ enthalten ist. Die Produktregel ist eine effiziente Methode die Ausmultiplikation der Funktionen zu vermeiden (wodurch die Ableitungsfunktion jedoch auch berechnet werden k\"{o}nnte): $$ [u(x) \cdot v(x)]' = u' \cdot v + v' \cdot u$$

\emph{Beispiel}: $f(x) = (x^2 - 5) \cdot (x^3 + 4x + 1)$

\begin{tabular}{l l}
	I. Bestimmung der ersten Termgruppe: & $u(x) = x^2 - 5$
	\extrapar \\
	II. Bestimmung der zweiten Termgruppe: & $v(x) = x^3 + 4x + 12$
	\extrapar \\
	III. Ableiten von $u(x)$: & $u'(x) = 2x$
	\extrapar \\
	IV. Ableiten von $v(x)$: & $v'(x) = 3x^2 + 4$
	\extrapar \\
	V. Anwendung der Produktregel: & $u' \cdot v + v' \cdot u = 2x \cdot (x^3 + 4x + 1) + (3x^2 + 4) \cdot (x^2 - 5)$
	\extrapar \\
	VI. Ausmultiplizieren: & $(2x^4 + 8x^2 + 2x) + (3x^4 - 15x^2 + 4x^2 - 20)$
\end{tabular}

\emph{Ergebnis}: $f'(x) = 5x^4 - 3x^2 + 2x - 20$

\pagebreak

\subsub{Quotientenregel}

Die Quotientenregel wird dann verwendet, wenn die Funktion Br\"{u}che enth\"{a}lt, in welchen die unab\"{a}ngige Variable $x$ sowohl im Nenner $v(x)$ als auch im Z\"{a}hler $u(x)$ vorkommt: $$\left[\frac{u(x)}{v(x)}\right]' = \frac{u' \cdot v - v' \cdot u}{v^2}$$
\emph{Beispiel}: $f(x) = \frac{5x^3 + 9x - 7}{4-x^2}$

\begin{tabular}{l l}
	I. Bestimmung der ersten Termgruppe: & $u(x) = 5x^3 + 9x - 7$
	\extrapar \\
	II. Bestimmung der zweiten Termgruppe: & $v(x) = 4 - x^2$
	\extrapar \\
	III. Ableiten von $u(x)$: & $u'(x) = 15x^2 + 9$
	\extrapar \\
	IV. Ableiten von $v(x)$: & $v'(x) = 2x$
	\extrapar \\
	V. Anwendung der Quotientenregel: & $\frac{u' \cdot v - v' \cdot u}{v^2} = \frac{(15x^2 + 9) \cdot (4-x^2) - (-2x) \cdot (5x^3 + 9x - 7)}{(4-x^2)^2}$
	\extrapar \\
	VI. Ausmultiplizieren: & $\frac{(60x^2 - 15x^4 + 36 - 9x^2) - (10x^4 + 18x^2 - 14x)}{x^4 - 8x^2 + 16}$
\end{tabular}

\emph{Ergebnis}: $f'(x) = \frac{-5x^4 + 69x^2 - 14x + 36}{x^4 - 8x^2 + 16}$

\subsub{Reziprokregel}

Die Reziprokregel ist eine spezielle Form der Quotientenregel, die nur zutrifft, wenn der Z\"{a}hler $u(x)$ gleich 1 ist, sodass f\"{u}r die Funktion $f(x)$ gilt, dass sie eine \emph{Reziprokfunktion} der Form $\frac{1}{v(x)}$ ist: $$\left[\frac{1}{v(x)}\right]' = -\frac{v}{v^2}$$
\emph{Beispiel}: $f(x) = \frac{1}{1 - x^2}$

\begin{tabular}{l l}
	I. Ermittlung der Termgruppe: & $v(x) = 1 - x^2$
	\extrapar \\
	II. Ableiten von $v(x)$: & $v'(x) = -2x$
	\extrapar \\
	III. Anwendung der Reziprokregel: & $-\frac{v'}{v^2} = -\left[\frac{-2x}{(1-x^2)^2}\right] = \frac{2x}{(1-x^2)^2}$
\end{tabular}

\emph{Ergebnis}: $f'(x) = \frac{2x}{x^4 - 2x^2 + 1}$

R\"{u}ckf\"{u}hrung auf die Quotientenregel: \hspace{1cm} $\left[\frac{u(x)}{v(x)}\right]' = \left[\frac{1}{v(x)}\right]' = \frac{[1]' \cdot v - v' \cdot 1}{v^2} = \frac{0 \cdot v - v'}{v^2} = \frac{-v'}{v^2}$

\pagebreak

\subsub{Kettenregel}

Die Kettenregel sollte dann angewendet werden, wenn eine Funktion $u(x)$ eine weitere Funktion $v(x)$ enth\"{a}lt, wobei beide Funktionen Terme mit der unabh\"{a}ngigen Variable $x$ besitzen. Hierbei muss man die \"{a}u\ss{}ere Ableitung $[u(v)]'$ mit der inneren $v'$ multiplizieren: $$[u(v)]' = v' \cdot [u(v)]'$$
\emph{Beispiel}: $f(x) = (3x - 4)^2$

\begin{tabular}{l l}
	I. Bestimmung des inneren Gliedes: & $v(x) = 3x - 4$
	\extrapar \\
	II. Bestimmung des \"{a}u\ss{}eren Gliedes: & $u(v(x)) = [v(x)]^2$
	\extrapar \\
	III. Ableiten des inneren Gliedes: & $v'(x) = 3$
	\extrapar \\
	IV. Ableiten des \"{a}u\ss{}eren Gliedes: & $[u(v(x))]' = 2 \cdot (3x - 4) = 6x - 8$
	\extrapar \\
	V. Anwendung der Kettenregel: & $v' \cdot [u(v)]' = 3 \cdot (6x - 8)$
\end{tabular}

\emph{Ergebnis}: $f'(x) = 18x - 24$

Eine aus der Potenz- und Kettenregel folgende Formel zur Berechnung von Ableitungsfunktionen von Wurzelfunktionen lautet: $$\left[\sqrt{f(x)}\,\,\right] = \frac{f'(x)}{2\sqrt{f(x)}}$$
\sub{Kurvendiskussion}

Die Kurvendiskussion besch\"{a}ftigt sich mit der Analyse der Null-, Extrem- sowie Wendepunkte, dem Verlauf, der Monotonie sowie der Kr\"{u}mmung einer Funktion. Ebenso ist oft nach der Tangente an die Wendepunkte, den sogenannten \emph{Wendetangenten}, gefragt. Bez\"{u}glich der Analyse der Nullstellen einer Funktion ist anzumerken, dass die \emph{Vielfachheit} der Nullstelle eine Rolle spielt. Eine \emph{$n$-fache Nullstelle} ist eine Stelle, an welcher sowohl in der Funktion $f(x)$ als auch in $n-1$ weiteren Ableitungen dieser Funktion der Funktionswert gleich null ist. 

\textbf{Beispiel}: \emph{Gib f\"{u}r die Funktion $f(x) = x^4 - 3x^2 + 2x$ die Wende- und Extrempunkte sowie die Nullstellen, samt Vielfachheit, an. Beschreibe ebenso die Monotonie der Funktion. Fertige eine Skizze an.}

\textbf{Schritt I: Ableitungen der Funktion $f(x)$ anschreiben}

$f(x) = x^4 - 3x^2 + 2x$

$f'(x) = 4x^3 - 6x + 2$

$f''(x) = 12x^2 - 6$

\textbf{Schritt II: Nullstellen finden}

Da es sich bei $f(x)$ um eine Polynomfunktion 4. Grades handelt, kann man sie nicht direkt in eine der beiden L\"{o}sungsformeln einsetzen, welche nur f\"{u}r Polynomfunktionen 2. Grades anwendbar sind. Daher muss man zuerst zwei Nullstellen durch Probieren sowie Abspalten finden. 

\pagebreak

Man sieht bei $f(x)$, dass der konstante Term nicht vorhanden bzw. gleich 0 ist, was bedeutet, dass die Funktion bei $x=0$ keinen Abstand vom Ursprung hat. Somit erf\"{a}hrt man schon, dass die erste Nullstelle $N_{1}$ im Ursprung liegt: $N_{1} = 0$

Danach f\"{u}hrt man eine Polynomdivision von $f(x)$ durch $x-N_{1}$. Da $N_{1}$ gleich 0 ist, ist die Polynomdivision auf eine Division durch $x$ reduziert: $$\frac{f(x)}{x} = \frac{x^4 - 3x^2 + 2x}{x} = x^3 - 3x + 2$$ Die n\"{a}chste Nullstelle findet man auf die selbe Weise durch Abspalten von $x^3 - 3x + 2 = 0$. Diesmal ist der konstante Term $k$ gleich $2$. Die neue Nullstelle ist jener $x$-Wert, f\"{u}r welchen bei Einsetzen in diese Gleichung gilt $0=0$. Diesen $x$-Wert findet man nur durch Probieren. Mit Sicherheit liegt der Wert zwischen $-k$ und $k$ und ist meistens ein Teiler von $k$. Daher versucht man in diesem Fall zuerst $\pm 1$ und $\pm 2$ einzusetzen. Man erf\"{a}hrt somit, dass die zweite Nullstelle $N_{2}$ bei $x=1$ liegt. Mit diesem Wert f\"{u}hrt man nun wieder eine Polynomdivision durch:

\begin{eq}
$(x^3 - 3x + 2) \div (x - 2) = x^2 + x - 2$

$\underline{-x^3 + x^2}$

$x^2 - 3x + 2$

$\underline{-x^2 + x}$

$-2x + 2$

$\underline{2x - 2}$

$0$ Rest

\end{eq}

Die neue Funktion lautet somit $x^2 + x - 2$. Da es sich hierbei um eine Funktion zweiten Grades handelt, kann man sie in die Gro\ss{}e L\"{o}sungsformel einsetzen, welche f\"{u}r Funktionen des Schemas $ax^2 + bx + c$ anwendbar ist: $$x_{1,2} = \frac{-b \pm \sqrt{b^2 - 4ac}}{2a}$$ Da der Koeffizient $a$ hierbei gleich $0$ ist, k\"{o}nnte man ebenso in die Kleine L\"{o}sungsformel f\"{u}r Funktionen des Schemas $x^2 + px + q$ einsetzen: $$x_{1,2} = -\frac{p}{2} \pm \sqrt{\left(\frac{p}{2}\right)^2 - q}$$ In diesem Fall sind aber alle Koeffizienten, also $a$ bei $ax^2$ sowie $b$ bei $bx$ gleich null. Daher ist es noch effizienter, einfach einen $x$-Wert herauszuheben:

\begin{eq}
$x^2 + x - 2 = 0$

$x(x + 1) = 2$

$x_{1} = -2$

$x + 1 = 2\,|\,-1$

$x_{2} = 1$
\end{eq}

Die dritte Nullstelle $N_{3}$ liegt somit bei $2$ und die vierte Nullstelle $N_{4}$ liegt bei $1$. Da schon $N_{2}$ bei $1$ lag, handelt sich bei $N_{2}$ bzw. $N_{4}$ um eine \emph{zweifache Nulstelle}. 

Nun sind also alle Nullstellen gefunden: $N_{1}(0|0),\, N_{2,4}(1|0),\, N_{3}(-2|0)$

\pagebreak

\textbf{Schritt III: Extrempunkte finden}

Als n\"{a}chstes gilt es, die Hoch- und Tiefpunkte, also die Extrempunkte, der Funktion zu finden. Zun\"{a}chst sei wieder angemerkt, dass der Unterschied zwischen einer Stelle und einem Punkt jener ist, dass eine Stelle nur den $x$-Wert auf der Abszisse nennt, w\"{a}hrend ein Punkt ein Zahlentupel bestehend aus der Stelle und ihrem Funktionswert ist.

An einer Extremstelle ist die Steigung gleich $0$, da sich die Monotonie der Funktion genau im Extrempunkt \"{a}ndert (die Tangente an einen Extrempunkt ist waagrecht). Daher besch\"{a}ftigen wir uns mit der ersten Ableitung $f'(x)$ der Funktion $f(x)$ und bestimmen die Extrempunkte, in dem wir die Ableitung $f'(x)$ gleich 0 setzen: $$4x^3 - 6x + 2 = 0$$

Da es sich hierbei wieder um eine Funktion dritten Grades handelt, gilt es eine Extremstelle durch probieren zu finden. Setzt man folglich Teiler des konstanten Terms $2$, also $\pm 1$ und $\pm 2$, ein, erh\"{a}lt man die erste Extremstelle $E_{1}$ bei $x = 1$. Mit diesem Wert f\"{u}hrt man erneut eine Polynomdivision durch:

\begin{eq}
$(4x^3 - 6x + 2) \div (x - 1) = 4x^2 + 4x - 2$

$\underline{-4x^3 + 4x^2}$

$4x^2 - 6x + 2$

$\underline{-4x^2 + 4x}$

$-2x + 2$

$\underline{2x - 2}$

$0$ Rest
\end{eq}

Die neue Funktion $4x^2 + 4x - 2$ dividiert man zun\"{a}chst durch $2$: $$\frac{4x^2 + 4x - 2}{2} = 2x^2 + 2x - 1$$

Hierbei liegt wieder eine Funktion 2. Grades vor, welche in eine der beiden L\"{o}sungsformeln eingesetzt werden kann. F\"{u}r die Kleine L\"{o}sungsformel m\"{u}sste der Koeffizient des quadratischen Terms ($2x^2$) gleich $1$ sein. Da dies hier nicht der Fall ist und eine weitere Division durch $2$ das Problem nicht wirklich vereinfacht, setzt man in die Gro\ss{}e L\"{o}sungsformel ein:

\begin{eq}
$x_{1,2} = \frac{-b \pm \sqrt{b^2 - 4ac}}{2a}$

$x_{1,2} = \frac{-2 \pm \sqrt{(-2)^2 - 4 \cdot 2 \cdot (-1)}}{2 \cdot 2} = \frac{-2 \pm \sqrt{12}}{4}$

$x_{1} = \frac{-2 - \sqrt{2}}{4} = -1.37$

$x_{2} = \frac{-2 + \sqrt{2}}{4} = 0.37$
\end{eq}

\pagebreak

Die beiden restlichen Extremstellen liegen also bei $x = -1.37$ und $x = 0.37$.  Durch Einsetzen der $x$-Werte der drei Extremstellen in die Funktionsgleichung $f(x)$ erh\"{a}lt man die Funktionswerte dieser Stellen: $$E_{1}(1 | 0),\, E_{2}(-1.37|-4.85), \, E_{3}(0.37|0.35)$$

Letztlich kann es noch von Interesse sein, ob es sich bei einem bestimmten Extrempunkt um einen Hoch- oder um einen Tiefpunkt handelt. Dies erf\"{a}hrt man durch die Kr\"{u}mmung der Funktion an der Extremstelle. Ist die Kr\"{u}mmung negativ, liegt ein Hochpunkt vor; ist die Kr\"{u}mmung positiv, handelt es sich um einen Tiefpunkt. Ist die Kr\"{u}mmung jedoch null, liegt an dieser Stelle ebenso eine Wendestelle. Solch ein Punkt, welcher sowohl Extrem- also auch Wendepunkt ist, wird \emph{Sattel-} oder \emph{Terassenpunkt} genannt:

\begin{tabular}{l l l l l}
$E_{1}(1 | 0)$ & $\rightarrow$ & $f''(1) > 0$ & $\rightarrow$ & Tiefpunkt
\\
&&&&
\\
$E_{2}(-1.37 | -4.85)$ & $\rightarrow$ & $f''(-1.37) > 0$ & $\rightarrow$& Tiefpunkt
\\
&&&&
\\
$E_{3}(0.37 | 0.35)$ & $\rightarrow$ & $f''(0.37) < 0$ & $\rightarrow$& Hochpunkt
\end{tabular}

\textbf{Schritt IV: Wendepunkte finden}

An einem Wendepunkt einer beliebigen Funktion \"{a}ndert sich stets das Kr\"{u}mmungsverhalten, also von positiv zu negativ oder umgekehrt. Daher ist an einem Wendepunkt die Kr\"{u}mmung gleich null. Somit gilt es, jene Stelle zu bestimmen, an welcher die zweite Ableitung $f''(x)$, welche die Kr\"{u}mmung der Funktion $f(x)$ beschreibt, gleich $0$ ist: $$12x^2 - 6 = 0$$ Diese Funktion hat nur einen Term, welcher die unab\"{a}ngige Variable $x$ enth\"{a}lt: $12x^2$. Daher ist es sinnlos, diese Funktion in eine der beiden L\"{o}sungsformeln einzusetzen. Man formt einfach um:

\begin{eq}
$12x^2 - 6 = 0 \,|\,+ 6$

$12x^2 = 6 \,|\, \div 12$

$x^2 = 0.5 \,|\, \sqrt{}$

$x_{1, 2} = \pm \sqrt{0.5}$

$x_{1} = \sqrt{0.5} = \frac{\sqrt{2}}{2}$

$x_{2} = - \sqrt{0.5} = -\frac{\sqrt{2}}{2}$
\end{eq}

Durch Einsetzten der gefundenden Wende\emph{stellen} in die urspr\"{u}ngliche Funktionsgleichung $f(x)$, erh\"{a}lt man die beiden vollst\"{a}ndigen Wende\emph{punkte} der Funktion:

\begin{tabular}{l l l l l}
	$x_{1} = \frac{\sqrt{2}}{2}$ & $\rightarrow$ & $f\left(\frac{\sqrt{2}}{2}\right) = 0.16$ & $\rightarrow$ & $W_{1}\left(\frac{\sqrt{2}}{2} \,\bigg|\, 0.16\right)$
	\\
	&&&&
	\\
	$x_{1} = -\frac{\sqrt{2}}{2}$ & $\rightarrow$ & $f\left(-\frac{\sqrt{2}}{2}\right) = -2.67$ & $\rightarrow$ & $W_{1}\left(-\frac{\sqrt{2}}{2} \,\bigg|\, -2.67 \right)$
\end{tabular}

\pagebreak

\textbf{Schritt V: Wendetangenten finden}

Die Wendetangenten $t_{W_{1,2}}$ sind jene Geraden, welche die Funktion genau in den Wendepunkten ber\"{u}hren. Man findet sie durch einfaches Einsetzten der Wendekoordinaten in die allgmeine lineare Funktionsgleichung $y = kx + d$. Die Steigung k erh\"{a}lt man durch die erste Ableitung, in welche man die $x$-Koordinate des Wendepunktes einsetzt. Den Abstand vom Ursprung $d$ errechnet man sich folglich durch L\"{o}sen der linearen Funktionsgleichung. Den ersten Wendepunkt $W_{1}(\sqrt{0.5}|0.16)$ findet man somit durch die folgenden Schritte:

\begin{tabular}{l l}
	I. Aufstellen der linearen Funktionsgleichung: & $y = kx + d$
	\\
	II. Berechnung der Steigung an der Wendestelle: & $f'(\frac{\sqrt{2}}{2}) = -0.83$
	\\
	III. Einsetzen in die Funktionsgleichung: & $0.16 = \frac{\sqrt{2}}{2} \cdot -0.83 + d$
	\\
	IV. L\"{o}sen nach d: & $d = 0.75$
	\\
	V. Fertige Wendetangente $t_{W_{1}}$: & $y = -0.83x + 0.75$
\end{tabular}

Das selbe Verfahren wendet man auch f\"{u}r den zweiten Wendepunkt $W_{2}$ an, um die zweite Wendetangente $t_{W_{2}}$ zu finden.

\textbf{Schritt VI. Skizze anfertigen}

\begin{figure}[h!]
	\begin{tikzpicture}
		\begin{axis}
		[
			xlabel = $x$,
			ylabel = $y$,
			axis lines = middle,
			legend pos = outer north east,
			restrict y to domain = -7:7,
			samples = 1000
		]

			\addplot [blue] {x^4 - 3*x^2 + 2*x};

			\addlegendentry{$f(x) = x^4 - 3x^2 + 2x$};


			\addplot [magenta] {4*x^3 - 6*x + 2};

			\addlegendentry{$f'(x) = 4x^3 - 6x + 2$};


			\addplot[cyan] {12*x^2 - 6};

			\addlegendentry{$f''(x) = 12x^2 - 6$};

		\end{axis}
	\end{tikzpicture}
\end{figure}

\textbf{Schritt VII. Monotonie beschreiben}

Die Monotonie einer Funktion bezieht sich auf ihr Steigungsverhalten, welches sich an den Extrempunkten ver\"{a}ndert. Ebenso erh\"{a}lt man durch die Kr\"{u}mmung der Funktion in einem bestimmten Intervall Informationen \"{u}ber ihre Monotonie. Ist die Kr\"{u}mmung negativ, wird die Steigung zunehmend negativer. Ist die Kr\"{u}mmung positiv, wird auch die Steigung in diesem Intervall zunehmend positiver. Da die Steigung an einer Extremstelle gleich 0 ist, darf diese Stelle nie zu einem Monotonieintervall dazugez\"{a}hlt werden. Das selbe trifft auch auf plus oder minus unendlich ($\infty$) zu. Somit gilt f\"{u}r die oben skizzierte Funktion $f(x)$:

\begin{tabular}{l l l l}
	Streng monoton wachsend f\"{u}r alle: & $x \in \,] -1.37\,; 0.37\,[$ & $\cup$ & $]\, 1\,; \infty \, [$
	\\
	&&&
	\\
	Streng monoton fallend f\"{u}r alle: & $x \in \,] -\infty\,; -1.37\,[$ & $\cup$ & $] \, 0.37\,; 1 \, [$
\end{tabular}

\pagebreak

\sub{Finden von Polynomfunktionen}

Mit Hilfe der Differentialrechnung ist es m\"{o}glich, eine Polynomfunktion beliebigen Grades mit nur wenigen Angaben zu finden. Pro Grad einer Polynomfunktion enth\"{a}lt sie eine weitere Variable, entweder als Koeffizient oder als konstanter Term bzw. Abstand vom Ursprung. Daher ben\"{o}tigt man zum Finden einer Polynomfunktion $n$-ten Grades mindestens $n + 1$ Angaben zur Funktion oder einer ihrer Ableitungen.

\textbf{Beispiel}: \emph{Der Graph einer Polynomfunktion 3. Grades besitzt den Tiefpunkt $T(3|-2)$ und den Wendepunkt $W(2|2)$. Wie lautet der Funktionsterm?}

\textbf{Schritt I: Aufstellen von Funktionsgleichung und Ableitungen}

Zun\"{a}chst ist es immer von Vorteil, die allgemeine Funktionsgleichung sowie jene der Ableitungen aufzustellen. F\"{u}r eine Funktion 3. Grades lauten diese:

$f(x) = ax^3 + bx^2 + cx + d$

$f'(x) = 3ax^2 + 2bx + c$

$f''(x) = 6ax + 2b$

\textbf{Schritt II: Interpretation der Angaben}

Die Funktion besitzt \zitat{besitzt den Tiefpunkt $T(3|-2)$}. Daraus l\"{a}sst sich zum Ersten schlie\ss{}en, dass der Funktionswert der Funktion an der Stelle $x = 3$ gleich $-2$ ist. Dies kann man sofort in die Funktionsgleichung $f(x)$ einsetzen: $$f(3) = -2$$ $$27a + 9b + 3c + d = -2$$ Zum Zweiten handelt es sich hierbei um einen Extrempunkt. Daher muss die Steigung an der Stelle $x = 3$ gleich $0$ sein: $$f'(3) = 0$$ $$27a + 6b + c = 0$$ Ebenso findet man auf der Funktion \zitat{den Wendepunkt $W(2|2)$}. Wieder ergibt sich daraus ein Funktionswert: $$f(2) = 2$$ $$8a + 4b + 2c + d = 2$$ An einer Wendestelle ver\"{a}ndert sich die Kr\"{u}mmung der Funktion, daher ist die zweite Ableitung an dieser Stelle gleich 0: $$f''(2) = 0$$ $$12a + 2b = 0$$

\pagebreak

\textbf{Schritt III: Schneiden der gefundenen Gleichungen}

Nun gilt es, die gefundenen Gleichungen zu schneiden und somit alle Variablen der Funktion zu finden.

I. $27a + 9b + 3c + d = -2$

II. $27a + 6b + c = 0$

III. $8a + 4b + 2c + d = 0$

IV. $12a + 2b = 0$


I $\cap\,-\text{II }\rightarrow$ V

V $\cap\,-\text{III }\rightarrow$ VI

VI $\cap \text{ II} \div -2 \rightarrow$ VII

VII $\rightarrow a = 2$

$a$ in VI $\rightarrow b = -12$

$a, b$ in III $\rightarrow c = 18$

$a, b, c$ in II $\rightarrow d = -2$

Die fertige Funktion lautet demnach: $f(x) = 2x^3 - 12x^2 + 18x - 2$ 

\end{document}