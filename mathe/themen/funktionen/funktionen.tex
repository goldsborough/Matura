% Funktionen

\documentclass[11pt]{article}

\usepackage[a4paper, margin=1in]{geometry}

\usepackage{amsmath}

\usepackage{amssymb}

\usepackage[german]{babel}

\usepackage[autostyle=true]{csquotes}

\usepackage{libertine}

\usepackage[libertine]{newtxmath}

\usepackage{tikz}

\usepackage{gensymb}

\usepackage{fancyhdr}

\usepackage{amsfonts}

\usepackage{pgfplots}

\pgfplotsset{compat=1.10}

\usepackage{multicol}

\usepackage{caption}

\usepackage{floatrow}

\everymath{\displaystyle}

% Header / footer settings

\pagestyle{fancy}
\fancyhf{}
\renewcommand{\headrulewidth}{0.2mm}
\fancyhead[C]{Funktionen}
\renewcommand{\footrulewidth}{0.2mm}
\fancyfoot[L]{Peter Goldsborough}
\fancyfoot[C]{\thepage}
\fancyfoot[R]{\today}

\fancypagestyle{plain}{%
	\fancyhf{}
	\renewcommand{\headrulewidth}{0mm}%
	\renewcommand{\footrulewidth}{0.2mm}%
	\fancyfoot[L]{Peter Goldsborough}
	\fancyfoot[C]{\thepage}
	\fancyfoot[R]{\today}
}


\setlength{\headheight}{15pt}

\setlength{\parindent}{0pt}

\addtolength{\parskip}{\baselineskip}


\newcommand{\overbar}[1]{\mkern 1.5mu\overline{\mkern-1.5mu#1\mkern-1.5mu}\mkern 1.5mu}

\newcommand{\heading}[1]{\begin{center}\Huge \textbf{#1}\end{center}\par}

\newcommand{\sub}[1]{\vspace{\parskip}{\LARGE\textbf{#1}}}

\newcommand{\subsub}[1]{{\Large \textbf{#1}}}

\newcommand{\subsubsub}[1]{\textbf{#1}}

\newcommand{\colvec}[1]{\begin{pmatrix}#1\end{pmatrix}}

\newcommand{\extrapar}{\par\vspace{\baselineskip}}

\newcommand{\zitat}[1]{\foreignquote{german}{#1}}

\newcommand{\bolditem}[1]{\item \textbf{#1}}

\newcommand{\titleitem}[1]{\bolditem{#1}\par}

\newcommand{\defas}{ \dots \,\,}

\begin{document}
\thispagestyle{plain}

\heading{Funktionen}

\sub{Definition}

Eine Funktion $f$ ist eine eindeutige Zuordnung zwischen einer unabh\"{a}ngigen \textbf{Definitionsmenge} $D$ und einer von dieser abh\"{a}ngigen \textbf{Wertemenge} $W$. F\"{u}r jeden zul\"{a}ssigen, unabh\"{a}ngigen Eingabewert $x$ legt eine Funktion $f(x)$ eindeutig einen von $x$ abh\"{a}ngigen Funktions- bzw. Ausgabewert $y$ fest.

F\"{u}r eine Funktion $f(x)$ gilt somit:

\textbf{Unabh\"{a}ngige Variable} \defas $x$

\textbf{Abh\"{a}ngige Variable} \defas $y$ bzw. $f(x)$

\textbf{Definitionsmenge} $D_{f}$ \defas Menge aller zul\"{a}ssigen, unabh\"{a}ngigen Eingabewerte $x$

\textbf{Wertemenge} $W_{f}$ \defas Menge aller auftretenden, von $x$ bzw. $D_{f}$ abh\"{a}ngigen Funktionswerte $y$

\textbf{Funktionsgleichung} \defas

$y = f(x)$

oder

$f: x \rightarrow y$ ($f$ bildet Werte die Menge der $x$ auf die Menge der $y$ ab).

\sub{Darstellungsweisen}

Eine Funktion $f(x)$ kann als Funktionsterm bzw. -gleichung, als Wertetabelle oder als Funktionsgraph dargestellt werden. Beispiel:

\begin{multicols}{2}

\textbf{Funktionsterm} \defas $2x + 1$

\textbf{Funktionsgleichung} \defas $f(x) = 2x + 1$

\textbf{Wertetabelle} \defas
\begin{tabular}{c | c}

	$x$ & $y$
	\\ \hline
	-1 & -1
	\\
	0 & 1
	\\
	1 & 3
	\\
	2 & 5

\end{tabular}

\begin{tikzpicture}
	\begin{axis}
	[
		axis lines = middle,
		xlabel = $x$,
		ylabel = {$f(x) = 2x + 1$},
		ytick = {1, 2, 3, 4, 5}
	]

	\addplot
	[
		domain=-1:2,
		samples=100,
		mark=*
	]
	coordinates {(-1, -1) (0, 1) (1, 3) (2, 5)};

	\end{axis}
\end{tikzpicture}

\begin{center}\textbf{Funktionsgraph}\end{center}

\end{multicols}

\sub{Wichtige Begriffe}

Vor der Beschreibung bzw. Diskussion wichtiger Begriffe im Zusammenhang mit Funktionen sollte angemerkt werden, worin der Unterschied zwischen einer \textbf{Stelle} und einem \textbf{Punkt} einer Funktion liegt. Mit einer \emph{Stelle} ist immer nur der Wert der Definitionsmenge bzw. die unab\"{a}ngige Variable --- also $x$ --- gemeint. Ein \emph{Punkt} bezeichnet dagegen ein Koordinatentupel bestehend aus der unab\"{a}ngigen \emph{und} abh\"{a}ngigen Variable --- also $(x, y)$.

\begin{figure}[h!]
	\centering
	\begin{tikzpicture}
	\begin{axis}
	[
		axis lines = middle,
		xlabel = $x$,
		ylabel = {$f(x)$},
		legend pos = outer north east
	]

	\addplot
	[
		domain=-1:1,
		samples=100
	]
	{-2*x + 1};

	\addlegendentry{$f(x) = -2x + 1$}

	\addplot
	[
		domain=-1:1,
		samples=100,
		mark=*,
		forget plot
	]
	coordinates {(0, 1) (0.333, 0.333) (0.5, 0)}
	node [pos = 0, right] {(N), (S)}
	node [pos = 0.69, above] {(F)}
	node [pos = 1, above right] {(S)};

	\addplot
	[
		domain=-1:1,
		samples=100,
		red,
		dashed
	]
	{x};

	\addlegendentry{1. Mediane}

	\end{axis}
\end{tikzpicture}
\end{figure}

\textbf{Nullstelle} (N) \defas jene Stelle einer Funktion, an welcher gilt $f(x) = 0 \rightarrow$ die Funktion schneidet die $x$-Achse

\textbf{Spurpunkt} (S) \defas jener Punkt einer Funktion, an welcher sie eine der beiden Achsen schneidet ($x$- oder $y$-Achse)

\textbf{Fixpunkt} (F) \defas jener Punkt einer Funktion, an welcher gilt $f(x) = x \rightarrow$ die Funktion schneidet die \emph{1. Mediane}



\end{document}