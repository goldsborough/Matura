% Funktionen

\documentclass[11pt]{article}

\usepackage[a4paper, margin=1in]{geometry}

\usepackage{amsmath}

\usepackage{amssymb}

\usepackage[german]{babel}

\usepackage[autostyle=true]{csquotes}

\usepackage{libertine}

\usepackage[libertine]{newtxmath}

\usepackage{tikz}

\usepackage{gensymb}

\usepackage{fancyhdr}

\usepackage{amsfonts}

\usepackage{pgfplots}

\pgfplotsset{compat=1.10}

\usepackage{multicol}

\usepackage{caption}

\usepackage{floatrow}

\everymath{\displaystyle}

% Header / footer settings

\pagestyle{fancy}
\fancyhf{}
\renewcommand{\headrulewidth}{0.2mm}
\fancyhead[C]{Funktionen}
\renewcommand{\footrulewidth}{0.2mm}
\fancyfoot[L]{Peter Goldsborough}
\fancyfoot[C]{\thepage}
\fancyfoot[R]{\today}

\fancypagestyle{plain}{%
	\fancyhf{}
	\renewcommand{\headrulewidth}{0mm}%
	\renewcommand{\footrulewidth}{0.2mm}%
	\fancyfoot[L]{Peter Goldsborough}
	\fancyfoot[C]{\thepage}
	\fancyfoot[R]{\today}
}


\setlength{\headheight}{15pt}

\setlength{\parindent}{0pt}

\addtolength{\parskip}{\baselineskip}


\newcommand{\overbar}[1]{\mkern 1.5mu\overline{\mkern-1.5mu#1\mkern-1.5mu}\mkern 1.5mu}

\newcommand{\heading}[1]{\begin{center}\Huge \textbf{#1}\end{center}\par}

\newcommand{\sub}[1]{\vspace{\parskip}{\LARGE\textbf{#1}}}

\newcommand{\subsub}[1]{{\Large \textbf{#1}}}

\newcommand{\subsubsub}[1]{\textbf{#1}}

\newcommand{\colvec}[1]{\begin{pmatrix}#1\end{pmatrix}}

\newcommand{\extrapar}{\par\vspace{\baselineskip}}

\newcommand{\zitat}[1]{\foreignquote{german}{#1}}

\newcommand{\bolditem}[1]{\item \textbf{#1}}

\newcommand{\titleitem}[1]{\bolditem{#1}\par}

\newcommand{\defas}{ \dots \,\,}

\begin{document}
\thispagestyle{plain}

\heading{Funktionen}

\sub{Definition}

Eine Funktion $f$ ist eine eindeutige Zuordnung zwischen einer unabh\"{a}ngigen \textbf{Definitionsmenge} $D$ und einer von dieser abh\"{a}ngigen \textbf{Wertemenge} $W$. F\"{u}r jeden zul\"{a}ssigen, unabh\"{a}ngigen Eingabewert $x$ legt eine Funktion $f(x)$ eindeutig einen von $x$ abh\"{a}ngigen Funktions- bzw. Ausgabewert $y$ fest.

F\"{u}r eine Funktion $f(x)$ gilt somit:

\textbf{Unabh\"{a}ngige Variable} \defas $x$

\textbf{Abh\"{a}ngige Variable} \defas $y$ bzw. $f(x)$

\textbf{Definitionsmenge} $D_{f}$ \defas Menge aller zul\"{a}ssigen, unabh\"{a}ngigen Eingabewerte $x$

\textbf{Wertemenge} $W_{f}$ \defas Menge aller auftretenden, von $x$ bzw. $D_{f}$ abh\"{a}ngigen Funktionswerte $y$

\textbf{Funktionsgleichung} \defas

$y = f(x)$

oder

$f: x \rightarrow y$ ($f$ bildet Werte die Menge der $x$ auf die Menge der $y$ ab).

\sub{Darstellungsweisen}

Eine Funktion $f(x)$ kann als Funktionsterm bzw. -gleichung, als Wertetabelle oder als Funktionsgraph dargestellt werden. Beispiel:

\begin{multicols}{2}

\textbf{Funktionsterm} \defas $2x + 1$

\textbf{Funktionsgleichung} \defas $f(x) = 2x + 1$

\textbf{Wertetabelle} \defas
\begin{tabular}{c | c}

	$x$ & $y$
	\\ \hline
	-1 & -1
	\\
	0 & 1
	\\
	1 & 3
	\\
	2 & 5

\end{tabular}

\begin{tikzpicture}
	\begin{axis}
	[
		axis lines = middle,
		xlabel = $x$,
		ylabel = {$f(x) = 2x + 1$},
		ytick = {1, 2, 3, 4, 5}
	]

	\addplot
	[
		domain=-1:2,
		samples=100,
		mark=*
	]
	coordinates {(-1, -1) (0, 1) (1, 3) (2, 5)};

	\end{axis}
\end{tikzpicture}

\begin{center}\textbf{Funktionsgraph}\end{center}

\end{multicols}

\sub{Wichtige Begriffe}

Vor der Beschreibung bzw. Diskussion wichtiger Begriffe im Zusammenhang mit Funktionen sollte angemerkt werden, worin der Unterschied zwischen einer \textbf{Stelle} und einem \textbf{Punkt} einer Funktion liegt. Mit einer \emph{Stelle} ist immer nur der Wert der Definitionsmenge bzw. die unab\"{a}ngige Variable --- also $x$ --- gemeint. Ein \emph{Punkt} bezeichnet dagegen ein Koordinatentupel bestehend aus der unab\"{a}ngigen \emph{und} abh\"{a}ngigen Variable --- also $(x, y)$.

\textbf{Nullstelle} (N) \defas jene Stelle einer Funktion, an welcher gilt $f(x) = 0 \rightarrow$ die Funktion schneidet die $x$-Achse

\textbf{Spurpunkt} (S) \defas jener Punkt einer Funktion, an welcher sie eine der beiden Achsen schneidet ($x$- oder $y$-Achse)

\textbf{Fixpunkt} (F) \defas jener Punkt einer Funktion, an welcher gilt $f(x) = x \rightarrow$ die Funktion schneidet die \emph{1. Mediane}

\textbf{Steigung} \defas jener Wert $k$, um welchen sich eine Funktion pro $x$-Wert auf der $y$-Achse ver\"{a}ndert. Berechenbar als $k = \frac{\Delta y}{\Delta x}$ f\"{u}r jedes beliebige Intervall. $k$ ist somit der \emph{Differenzenquotient} $\rightarrow$ die durschnittliche Ver\"{a}nderung der Funktion pro $x$-Wert in diesem Intervall.

\begin{figure}[h!]
	\centering
	\begin{tikzpicture}
	\begin{axis}
	[
		axis lines = middle,
		xlabel = $x$,
		ylabel = {$f(x)$},
		legend pos = outer north east
	]

	\addplot
	[
		domain=-1:1,
		samples=100
	]
	{-2*x + 1};

	\addlegendentry{$f(x) = -2x + 1$}

	\addplot
	[
		domain=-1:1,
		samples=100,
		mark=*,
		forget plot
	]
	coordinates {(0, 1) (0.333, 0.333) (0.5, 0)}
	node [pos = 0, right] {(S)}
	node [pos = 0.69, above] {(F)}
	node [pos = 1, above right] {(N), (S)};

	\addplot
	[
		domain=-1:1,
		samples=100,
		red,
		dashed
	]
	{x};

	\addlegendentry{1. Mediane}

	\end{axis}

	\draw [dashed]
	      (2.0, 4.0)
	   -- (0.0, 4.0) node [midway, below] {$\Delta x = 0.5$}
	   -- (0.0, 5.7) node [midway, left] {$\Delta y = -1$}
	   			     node [pos=0.4, right] {\hspace{0.4cm}$k$};
\end{tikzpicture}
\end{figure}

\textbf{Monotonie} \defas Beschreibung des Steigungsverhalten einer Funktion. Eine Funktion $f(x)$ ist in einem beliebigen Intervall $[x_{1}; x_{2}]$

\begin{itemize}
	\bolditem{monoton wachsend}, wenn der Funktionswert jedes $x$-Wertes in diesem Intervall \emph{nicht kleiner} ist als jener des vorhergehenden $x$-Wertes: $f(x) \geq f(x-1) \text{ f\"{u}r } x \in [x_{1}; x_{2}]$

	\bolditem{monoton fallend}, wenn der Funktionswert jedes $x$-Wertes in diesem Intervall \emph{nicht gr\"{o}\ss{}er} ist als jener des vorhergehenden $x$-Wertes: $f(x) \leq f(x-1) \text{ f\"{u}r } x \in [x_{1}; x_{2}]$

	\bolditem{streng monoton wachsend}, wenn der Funktionswert jedes $x$-Wertes in diesem Intervall \emph{stets gr\"{o}\ss{}er} ist als jener des vorhergehenden $x$-Wertes: $f(x) > f(x-1) \text{ f\"{u}r } x \in [x_{1}; x_{2}]$

	\bolditem{streng monoton fallend}, wenn der Funktionswert jedes $x$-Wertes in diesem Intervall \emph{stets kleiner} ist als jener des vorhergehenden $x$-Wertes: $f(x) < f(x-1) \text{ f\"{u}r } x \in [x_{1}; x_{2}]$

\end{itemize}

\pagebreak

\begin{figure}[h!]
\centering
	\begin{tikzpicture}
	% x axis
	\draw [->] (-4, 0) -- (4, 0) node [below] {x};

	% y axis
	\draw [->] (0, -4) -- (0, 3) node [right] {y};

	% SMW
	\draw [red]
	      (-3.5, -4)
	   -- (-2.5, -1) node [pos=0.7, left] {Streng Monoton\par Wachsend};

	% MW
	\draw [blue]
	      (-2.5, -1)
	   -- (-1, -1)
	   -- (-0.5,  2) node [midway, left] {Monoton Wachsend}
	   -- (0.2, 2);

	% MF
	\draw [orange]
	      ( 0.2,  2)
	   -- ( 1, -1) node [midway, right] {Monoton Fallend}
	   -- ( 2, -1);

	% SMF
	\draw [teal]
	      ( 2, -1)
	   -- ( 4, -4) node [pos=0.7, right] {Streng Monoton Fallend};
	\end{tikzpicture}
\end{figure}

\textbf{Extrempunkt} \defas jener Punkt einer Funktion, an welchem sich das Monotonieverhalten ver\"{a}ndert. Ein Extrempunkt kann entweder ein Hochpunkt (\emph{Maximum}) oder ein Tiefpunkt (\emph{Minimum}) sein. An einer Extremstelle gilt f\"{u}r die Steigung der Funktion $k = 0$. Daher ist die \emph{Tangente} an diesen Punkt waagrecht.

\textbf{Kr\"{u}mmung} \defas die Ver\"{a}nderung der Steigung einer Funktion $f(x)$ in einem bestimmten Intervall $[x_{1}; x_{2}]$. Es gibt zwei M\"{o}glichkeiten:

\begin{itemize}
	\bolditem{positiv bzw. linksgekr\"{u}mmt}, wenn die Steigung der Funktion in dem Intervall $[x_{1}; x_{2}]$ zunehmend ansteigt: $f'(x) > f'(x-1), \text{ f\"{u}r } x \in [x_{1}; x_{2}]$, f\"{u}r die zweite Ableitung gilt: $f''(x) > 0$

	\bolditem{negativ bzw. rechtsgekr\"{u}mmt}, wenn die Steigung der Funktion in dem Intervall $[x_{1}; x_{2}]$ zunehmend f\"{a}llt: $f'(x) < f'(x-1), \text{ f\"{u}r } x \in [x_{1}; x_{2}]$, f\"{u}r die zweite Ableitung gilt: $f''(x) < 0$
\end{itemize}

\textbf{Wendepunkt} \defas jener Punkt einer Funktion, in welchem sich ihre Kr\"{u}mmung ver\"{a}ndert. In dem Punkt selbst ist die Kr\"{u}mmung der Funktion gleich null: $f''(x_{W}) = 0$

\begin{figure}[h!]
\centering
	\begin{tikzpicture}
		\begin{axis}
		[
			mark=left,
			xlabel=$x$,
			ylabel={$f(x)$},
			axis lines = middle,
			legend pos = outer north east
		]

		\addplot
		[domain=-0.2:4.2, samples=100]
		{x^3-6*x^2+9*x-2}
		node [pos=0.1, right] {\hspace{0.8cm}$f''(x < 0)$}
		node [pos=0.285, right] {\hspace{0.1cm}$f'(x) = 0$}
		node [pos=0.04, right] {\hspace{4cm}$f'(x) = 0$}
		node [pos=0.9, left] {$f''(x > 0)\,\,\,\,\,\,\,\,\,\,\,$};

		\addplot
		[mark=*]
		coordinates {(1, 2)}
		node [above] {Hochpunkt};

		\addplot
		[mark=*]
		coordinates {(2, 0)}
		node [above right] {Wendepunkt};

		\addplot
		[mark=*]
		coordinates {(3, -2)}
		node [below] {Tiefpunkt};


		\end{axis}
	\end{tikzpicture}
\end{figure}

\textbf{Sattel- bzw. Terassenpunkt} \defas jener Punkt einer Funktion, welcher sowohl Wende- als auch Extrempunkt ist, sodass gilt $f'(x) = 0 \land f''(x) = 0$

\pagebreak

\textbf{Symmetrie} \defas einer Funktion beschreibt ihr Symmetrieverhalten, also ob und wie sich Variablen untereinander austauschen lassen. Eine Funktion $f(x)$ kann folgendes Symmetrieverhalten aufweisen:

\begin{itemize}
	\bolditem{gerade bzw. achsensymmetrisch}, wenn die Funktion an der $y$-Achse gespiegelt ist, sodass gilt $f(-x) = f(x)$

	\bolditem{ungerade bzw. punktsymmetrisch}, wenn die Funktion in jedem Punkt gespiegelt ist, sodass gilt $f(-x) = -f(x)$
\end{itemize}

\begin{figure}[h!]
	\begin{floatrow}
		\ffigbox
		{
			\begin{tikzpicture}
				\begin{axis}
				[
					xlabel=$x$,
					ylabel={$f(x)$},
					axis lines = middle
				]

				\addplot
				[domain=-1:1]
				{x^2};

				\end{axis}
			\end{tikzpicture}
		} {\caption*{Achsensymmetrisch}}

		\ffigbox
		{
			\begin{tikzpicture}
				\begin{axis}
				[
					xlabel=$x$,
					ylabel={$f(x)$},
					axis lines = middle
				]

				\addplot
				[domain=-1:1]
				{x^3};

				\addplot
				[domain=-1:1, dashed, red]
				{x};

				\end{axis}
			\end{tikzpicture}
		} {\caption*{Punktsymmetrisch}}
	\end{floatrow}
\end{figure}

\textbf{Periodizit\"{a}t} \defas eine Funktion $f(x)$ ist periodisch mit einer Periode $p$, wenn sich die Werte der Funktion im Abstand $p$ stets wiederholen, sodass gilt: $f(x) = f(x + p)$

\begin{figure}[h!]
	\centering
	\begin{tikzpicture}

    	\draw [->]
    	      (-4, 0)
    	   -- ( 4, 0) node [right] {$x$}
    	   			  node [midway, below right] {0};

    	\draw [->]
    	      (0, -2)
    	   -- (0,  2) node [right] {$f(x)$};

    	\draw [domain=-4:4, samples=1000]
    		   plot (\x, {sin(pi * \x r)});

    	\draw [red, dashed] 
    	      (-4, 2)
    	   -- (-4, -2) node [below] {$-2p$};

    	\draw [red, dashed] 
    	      (-2, 2)
    	   -- (-2, -2) node [below] {$-p$};

    	\draw [red, dashed] 
    	      (2, 2)
    	   -- (2, -2) node [below] {$p$};

    	\draw [red, dashed] 
    	      (4, 2)
    	   -- (4, -2) node [below] {$2p$};

	\end{tikzpicture}
\end{figure}

\textbf{Asymptote} \defas jene Gerade $a$, die einer Funktion $f$ beliebig nahe kommt, ohne sie jemals zu ber\"{u}hren.

\begin{figure}[h!]
	\centering
	\begin{tikzpicture}
    	\begin{axis}
    	[
    		y=1.2cm,
    		xlabel=$x$,
			ylabel={$f(x)$},
			ymin=0, ymax=3,
			ytick={1, 2},
			xmin=-8, xmax=8,
			axis lines = middle,
			legend pos = outer north east
    	]

    	\addplot
    	[domain=-8:8, samples=100]
    	{1/(x^2) + 0.4};

    	\addlegendentry{$x^{-2} + 1$}

    	\addplot
    	[domain=-8:8, red, dashed]
    	{0.3};

    	\addlegendentry{$Asymptote$};

    	\end{axis}
	\end{tikzpicture}
\end{figure}

\pagebreak

\sub{Funktionstypen}

Generell kann eine Funktion in einer von zwei Formen angeschrieben sein, entweder in der \textbf{Normal bzw. Hauptform} oder in der \textbf{Allgemeinen Form}. Diese Formen unterscheiden sich je nach Funktionstyp.

\textbf{Abszisse} \defas horizontale Achse eines Funktionsgraphen ($x$-Achse)

\textbf{Ordinate} \defas vertikale Achse eines Funktionsgraphen ($y$-Achse)

\subsub{Lineare Funktionen}

Eine lineare Funktion $f(x)$ ist eine linear wachsende oder fallende Funktion, mit einer festgelegten, konstanten Steigung $k$, sodass gilt: $f(x + 1) = f(x) + k$. Ebenso kann eine lineare Funktion einen Abstand vom Ursprung $d$ haben, um welchen alle Werte auf der Ordinate verschoben sind.

\textbf{Hauptform} \defas $y = kx + d$

\textbf{Allgemeine Form} \defas $ax + by + c = 0$

\textbf{Homogene Lineare Funktion} \defas eine Lineare Funktion $y = kx\,|\,k \neq 0$ ohne Abstand vom Ursprung $d$, wobei zwischen $y$- und $x$-Werten ein direktes Verh\"{a}ltnis (direkte Proportionalit\"{a}t) besteht, sodass jeder Wert $f(x)$ den Faktor $k$ mit dem $x$-Wert gemeinsam hat.

\textbf{Inhomogene Lineare Funktion} \defas eine Lineare Funktion $y = kx + d \,|\, k \neq 0 \land d \neq 0$ ohne direktem Verh\"{a}ltnis zwischen $x$- und $y$-Werten.

Eine Lineare Funktion hat ihre Spurpunkte in $(0|d)$ sowie $\left(-\frac{d}{k}|0\right)$.

\begin{figure}[h!]
\centering
	\begin{tikzpicture}
		\begin{axis}
		[
			xlabel = $x$,
			ylabel = $f(x)$,
			axis lines = middle,
			legend pos = outer north east
		]

		\addplot
		[domain=-4:4, blue]
		{4*x + 1};

		\addlegendentry{$4x + 1$};

		\addplot
		[domain=-4:4, red]
		{(-5*x/2) - 3};

		\addlegendentry{$-2.5x - 3$};
		\end{axis}
	\end{tikzpicture}
\end{figure}


\pagebreak

\subsub{Potenzfunktionen}

Eine Potenzfunktion $f(x)$ w\"{a}chst oder f\"{a}llt nicht-linear. Auch eine Potenzfunktion kann einen Abstand $c$ vom Ursprung haben, um welchen alle Funktionswerte auf der Ordinate verschoben sind. Je nachdem ob die Potenz $z$ gerade oder ungerade, positiv oder negativ ist, hat der Graph einer Potenfunktion verschiedene Formen, welche entweder punkt- oder achsensymmetrisch sind. Ist der Exponent $z$ einer Potenzfunktion $\in \mathbb{Z^+}$, liegt eine direkte Proportionalit\"{a}t vor, ist er $\in \mathbb{Z^-}$ sind $x$ und $y$ indirekt proportional. Gilt $z \in \mathbb{Q}$, handelt es sich um eine \emph{Wurzelfunktion}, da jede rationale Potenz $z$ einer Zahl $x$ als Bruch $x^{\frac{m}{n}}$ und somit als Wurzel $\sqrt[n]{x^m}$ dargestellt werden kann.

\textbf{Hauptform} \defas $ax^z + b$

\extrapar

\begin{figure}[h!]
	\begin{floatrow}
		\ffigbox
		{
			\begin{tikzpicture}
				\begin{axis}
				[
					samples=1000,
					xlabel=$x$,
					ylabel={$f(x)$},
					restrict y to domain=0:6,
					axis lines = middle,
					legend pos = north west
				]

				\addplot [blue] {x^2};

				\addlegendentry{$x^2$};

				\addplot [magenta] {x^4};

				\addlegendentry{$x^4$};

				\addplot [cyan] {x^8};

				\addlegendentry{$x^8$};

				\end{axis}
			\end{tikzpicture}
		} {\caption*{$z$ gerade und positiv $\Rightarrow f(x)$ achsensymmetrisch und $> 0$}}

		\ffigbox
		{
			\begin{tikzpicture}
				\begin{axis}
				[
					samples=1000,
					xlabel=$x$,
					ylabel={$f(x)$},
					restrict y to domain=-11:11,
					axis lines = middle,
					legend pos = north west
				]

				\addplot [blue] {x^3};

				\addlegendentry{$x^3$};

				\addplot [magenta] {x^5};

				\addlegendentry{$x^5$};

				\addplot [cyan] {x^7};

				\addlegendentry{$x^7$};

				\end{axis}
			\end{tikzpicture}
		} {\caption*{$z$ ungerade und positiv $\Rightarrow f(x)$ punktsymmetrisch}}
	\end{floatrow}

	\begin{floatrow}
		\ffigbox
		{
			\begin{tikzpicture}
				\begin{axis}
				[
					samples=1000,
					xlabel=$x$,
					ylabel={$f(x)$},
					restrict y to domain=0:6,
					restrict x to domain=-2.5:2.5,
					axis lines = middle,
					legend pos = north west
				]

				\addplot [blue] {x^-2};

				\addlegendentry{$x^{-2}$};

				\addplot [magenta] {x^-4};

				\addlegendentry{$x^{-4}$};

				\addplot [cyan] {x^-8};

				\addlegendentry{$x^{-8}$};

				\end{axis}
			\end{tikzpicture}
		} {\caption*{$z$ gerade und negativ $\Rightarrow f(x)$ achsensymmetrisch und $> 0$}}

		\ffigbox
		{
			\begin{tikzpicture}
				\begin{axis}
				[
					samples=1000,
					xlabel=$x$,
					ylabel={$f(x)$},
					restrict y to domain=-12:12,
					restrict x to domain=-2.5:2.5,
					axis lines = middle,
					legend pos = north west
				]

				\addplot [blue] {x^-3};

				\addlegendentry{$x^{-3}$};

				\addplot [magenta] {x^-5};

				\addlegendentry{$x^{-5}$};

				\addplot [cyan] {x^-7};

				\addlegendentry{$x^{-7}$};

				\end{axis}
			\end{tikzpicture}
		} {\caption*{$z$ ungerade und negativ $\Rightarrow f(x)$ punktsymmetrisch}}
	\end{floatrow}
\end{figure}

\pagebreak

\subsub{Wurzelfunktion}

Eine Wurzelfunktion ist eine besondere Form der Potenzfunktion, da eine rationale Potenz $z$ einer Zahl $x$ als Bruch $x^{\frac{m}{n}}$ und somit als Wurzel $\sqrt[n]{x^m}$ dargestellt werden kann. Wurzelfunktionen sind daher ebenso nicht-linear wachsend oder fallend. Gilt f\"{u}r die Definitionsmenge $D_{f}$ einer Wurzelfunktion $f(x)$, dass sie $\in \mathbb{R}$, so ist die Wurzelfunktion nur f\"{u}r positive Werte der Definitionsmenge ($x$-Werte) definiert, also $\mathbb{R^+}$.

\textbf{Hauptform} \defas $ax^{\frac{m}{n}} + b = a\sqrt[n]{x^m} + b = 0$

\begin{figure}[h!]
\centering
	\begin{tikzpicture}
		\begin{axis}
		[
			samples=1000,
			xlabel=x,
			ylabel={f(x)},
			domain=0:20,
			axis lines = middle,
			legend pos = outer north east
		]

		\addplot [blue] {x^(1/2)};

		\addlegendentry{$\sqrt{x}$};

		\addplot [magenta] {x^(2/3)};

		\addlegendentry{$\sqrt[3]{2}$};

		\addplot [cyan] {x^(1/5) + 1};

		\addlegendentry{$\sqrt[5]{x}$};

		\end{axis}
	\end{tikzpicture}
\end{figure}

\subsub{Polynomfunktion}

Eine Polynomfunktion ist eine Potenzfunktion bestehend aus mehreren Termen, in denen jeweils die unabh\"{a}ngige Variable $x$ mit verschiedenen Potenzen vorkommt. Sie kann einen sehr komplexen Verlauf haben und ihre Funktionswerte k\"{o}nnen durch einen Abstand vom Ursprung auf der Ordinate verschoben werden.
Spricht man von einer Polynomfunktion $n$-ten Grades, so ist die h\"{o}chste vorkommende Potenz der Polynome $n$.

\textbf{Allgemeine Form} \defas $\sum_{i=0}^{n} a_{i} \cdot x^i$

Hierbei ist der erste Term $a_{0}x^0$ der Abstand vom Ursprung, da $x^0 = 1$ und somit nur $a_{0}$ als Koeffizient ohne Variable \"{u}brig bleibt.

F\"{u}r eine Polynomfunktion $n$-ten Grades gibt es folgende Zusammenh\"{a}nge:

\begin{itemize}
	\item Anzahl der Nullstellen (N) =
	      $\begin{cases}
	      	0 \leq N \leq n\text{, wenn n gerade}\\
	      	1 \leq N \leq n\text{, wenn n ungerade}
	      \end{cases}$
	\item Anzahl der Extremstellen (E) = $1 \leq E < n$
	\item Anzahl der Wendepunkte (W) = $E - 1$
\end{itemize}

\pagebreak

\begin{figure}[t!]
\centering
	\begin{tikzpicture}
		\begin{axis}
		[
			samples=1000,
			xlabel=$x$,
			ylabel={$f(x)$},
			domain=-3:7,
			restrict y to domain=-10:6,
			axis lines = middle,
			legend pos = outer north east
		]

		\addplot [blue] {x^3 - 5*x - 4};

		\addlegendentry{$x^3 - 5x + 1$};

		\addplot [red] {(x^5 - 12*x^4 + 35*x^3 + 20*x^2 -156*x + 112)/56};

		\addlegendentry{$(x^5 - 12x^4 + 35x^3 + 20x^2 -156x + 112) \div 56$};

		\end{axis}
	\end{tikzpicture}
\caption*{Polynomfunktionen 3. und 5. Grades}
\end{figure}

\subsubsub{Linearfaktoren und der Satz des Vieta}

Jede beliebige Polynomfunktion $n$-ten Grades kann in $n$ Linearfaktoren aufgespalten werden. Linearfaktoren sind jene unabh\"{a}ngigen Variablenwerte, welche man durch L\"{o}sen der Funktion erh\"{a}lt. Pro Grad der Funktion gibt es eine L\"{o}sung. Hat man $n$ Linearfaktoren $x_{1}, x_{2}, ..., x_{n}$ einer Polynomfunktion $n$-ten Grades, erh\"{a}lt man deren Funktionsgleichung auf folgende Weise: $$f(x) = \prod_{i = 1}^{n} (x - x_{i})$$

Ebenso kann man mittels der beiden Linearfaktoren $x_{1}$ und $x_{2}$ einer quadratischen Polynomfunktion $f(x) = x^2 + px + q$ die Koeffizienten $p$ und $q$ direkt berechnen. Dazu verwendet man den \emph{Satz des Vieta}: $$-p = x_{1} + x_{2}$$ $$q = x_{1} \cdot x_{2}$$

\subsub{Exponentialfunktion}

Die Funktionswerte einer Exponentialfunktion wachsen oder fallen exponentiell. Daher gilt f\"{u}r eine Exponentialfunktion $f(x) = a \cdot b^x$, dass $f(x + 1) = f(x) \cdot b$. Oftmals ist die Basis $b$ gleich der Euler'schen Zahl $e$, was vor allem bei Wachstums- und Zerfallsmodellen von Bedeutung ist.

\textbf{Allgemeine Form} \defas $a \cdot b^x$

Hierbei bestimmt $b$ die Steigung der Funktion. Der Koeffizient $a$ ist der Abstand vom Ursprung, da f\"{u}r $x = 0$ anf\"{a}nglich $b^0 = 1$ gilt, somit ist der Abstand vom Ursprung $a \cdot 1$. Gilt f\"{u}r eine Exponentialfunktion $a = 1$, so hat sie ihren Abstand vom Ursprung bei $y =1$, da $1 \cdot b^0 = 1$.

\pagebreak

\begin{figure}[t!]
	\begin{floatrow}
		\ffigbox
		{
			\begin{tikzpicture}
				\begin{axis}
				[
					samples=1000,
					xlabel=$x$,
					ylabel={$f(x)$},
					domain=-3:4,
					restrict y to domain=0:10,
					axis lines = middle,
					legend pos = north west
				]

				\addplot [blue] {2^x};

				\addlegendentry{$2^x$};

				\addplot [magenta] {4^x};

				\addlegendentry{$4^x$};

				\addplot [cyan] {2*3^x};

				\addlegendentry{$2 \cdot 3^x$};

				\end{axis}
			\end{tikzpicture}
		} {\caption*{Exponentialfunktionen mit Basen $b$ gr\"{o}\ss{}er 1}}

		\ffigbox
		{
			\begin{tikzpicture}
				\begin{axis}
				[
					samples=1000,
					xlabel=$x$,
					ylabel={$f(x)$},
					domain=-5:7,
					restrict y to domain=0:10,
					axis lines = middle,
					legend pos = north east
				]

				\addplot [blue] {0.5^x};

				\addlegendentry{$0.5^x$};

				\addplot [magenta] {0.25^x};

				\addlegendentry{$0.25^x$};

				\addplot [cyan] {2 * 0.75^x};

				\addlegendentry{$2 \cdot 0.75^x$};

				\end{axis}
			\end{tikzpicture}
		} {\caption*{Exponentialfunktionen mit Basen $b$ kleiner 1}}
	\end{floatrow}
\end{figure}

\pagebreak

F\"{u}r Modellierungen von Wachstum und Zerfall biologischer oder sonstiger nat\"{u}rlicher Vorg\"{a}nge wird oft die Euler'sche Zahl $e$ als Basis verwendet. Die Funktionsgleichung einer Wachstums- oder Zerfallsfunktion beeinschlie\ss{}t neben der unabh\"{a}ngigen Variable $t$ (= Zeit) noch eine Wachstums- bzw. Zerfallskonstante $\lambda$ als Exponent der Basis $e$. Der Koeffizient $a$ ist gleich der urspr\"{u}nglichen Menge des Wachstums- / Zerfallsprozesses und wird $N_{0}$ bezeichnet. 

\textbf{Wachstums- bzw. Zerfallsfunktionsgleichung} \defas $N(t) = N_{0} \cdot e^{\lambda \cdot t}$

\begin{figure}[h!]
\centering
	\begin{tikzpicture}
		\begin{axis}
		[
			samples=1000,
			xlabel=$x$,
			ylabel={$f(x)$},
			restrict y to domain=-11:11,
			axis lines = middle,
			legend pos = north west
		]

		\addplot [red] {e^x};

		\addlegendentry{$e^{1 \cdot x}$};

		\end{axis}
	\end{tikzpicture}
\caption*{Exponentielles Wachstum mit $e$ als Basis}
\end{figure}

\pagebreak

\subsub{Sinus- und Cosinusfunktion}

Sinus- und Cosinusfunktionen sind periodsch fallend und wachsende Funktionen, dessen Definitionswerte (oft \emph{Phase} genannt) meist in Grad oder in Radien angegeben wird. Der Phasenunterschied zwischen Cosinus- und Sinusfunktion betr\"{a}gt $90\degree$ bzw. $\frac{\pi}{2}$ Radien.

\textbf{Allgemeine Sinufunktion} \defas $f(x) = a \cdot \sin(bx + c) + d$

\textbf{Allgemeine Cosinusfunktion} \defas $f(x) = a \cdot \cos(bx + c) + d$

Aus oben gennanter Beschreibung l\"{a}sst sich schlie\ss{}en: $\cos(x) = \sin(x + \frac{\pi}{2})$

Der Faktor $a$ bestimmt die maximale Amplitude (= H\"{o}he) der Funktion. $b$ bestimmt die Frequenz sowie indirekt die Periode $p$. Da Sinus- und Cosinusfunktionen im Normalfall, also wenn gilt $b=1$, eine Periode von $2\pi$ haben, gilt allgemein $p = \frac{1}{b \cdot 2\pi}$. Die Konstante $c$ ist eine beliebige Phasenverschiebung auf der Abszisse. $d$ ist ein Abstand vom Ursprung auf der Ordinate.

Einige Bemerkungen:

\begin{itemize}
	\item Nullstellen der Sinusfunktion liegen bei $\frac{p}{2} - c$ sowie $p - c$, jene der Cosinusfunktion bei $\frac{p}{4} + c$ sowie $\frac{3p}{4} - c$. Generell haben sie immer einen Abstand von $\frac{p}{2}$.

	\item Extremstellen der Sinusfunktion liegen bei $\frac{p}{4} + c$ sowie $\frac{3p}{4} + c$, jene der Cosinusfunktion bei $c$ sowie $\frac{p}{2} + c$
	
	\item $[\sin(x)]' = \cos(x)$, $[\cos(x)]' = -\sin(x)$

	\item $[\sin(kx)]' = k \cdot \cos(kx)$, $[\cos(kx)]' = k \cdot [-\sin(kx)]$
\end{itemize}

\begin{figure}[h!]
	\begin{tikzpicture}
		\begin{axis}
		[
			samples=1000,
			xlabel=$x$,
			ylabel={$f(x)$},
			domain=-8:8,
			xmin=-6, xmax=6,
			ymin=-4, ymax=4,
			axis lines = middle,
			legend pos = outer north east
		]

		\addplot [blue] {sin(deg(2 * x)};

		\addlegendentry{$\sin(2x)$}

		\addplot [red] {3 * cos(deg(x))};

		\addlegendentry{$3 \cdot \cos(x)$}

		\end{axis}
	\end{tikzpicture}
\end{figure}

\end{document}