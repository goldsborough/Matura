% Zahlenmengen

\documentclass[11pt]{article}

\usepackage[german]{babel}

\usepackage[autostyle=true]{csquotes}

\usepackage[a4paper, margin=1in]{geometry}

\usepackage{libertine}

\setlength{\parindent}{0pt}

\addtolength{\parskip}{\baselineskip}

\newcommand{\extrapar}{\par\vspace{\baselineskip}}

\newcommand{\heading}[1]{\begin{center}\Huge \textbf{#1} \end{center}}

\newcommand{\sub}[1]{{\Large \textbf{#1}}\par}

\newcommand{\subsub}[1]{{\large \textbf{#1}}\par}

\newcommand{\zitat}[1]{\emph{\foreignquote{german}{#1}}}

\newcommand{\titleitem}[1]{\item \textbf{#1} \par}

\begin{document}
\thispagestyle{plain}

\heading{Zahlenmengen}

\sub{Nat\"{u}rliche Zahlen}

Die Menge der nat\"{u}rlichen Zahlen $\mathbb{N}$ ist die exklusivste und fundamentalste Zahlenmenge, welche nur positive, ganze Zahlen beeinschlie\ss{}t. $\mathbb{N}$ ist bez\"{u}glich der \textbf{Addition} und \textbf{Multiplikation} abgeschlossen, nicht aber bez\"{u}glich der Subtraktion und Division.

$\mathbb{N} = \{0, 1, 2, 3, 4, 5, ...\}$

$\mathbb{N^*} = \{1, 2, 3, 4, 5, ...\}$

$\mathbb{N_\textnormal{g}} = \{2, 4, 6, 8, ...\}$

$\mathbb{N_\textnormal{u}} = \{1, 3, 5, 7, ...\}$

Eigenschaften der nat\"{u}rlichen Zahlen:

\begin{itemize}
	\item Es gibt eine kleinste nat\"{u}rliche Zahl: 0
	\item Jede nat\"{u}rliche Zahl $n$ au\ss{}er 0 hat einen Vorg\"{a}nger $n - 1$
	\item Jede nat\"{u}rliche Zahl $n$ hat einen Nachfolger $n + 1$
	\item Es gibt keine gr\"{o}\ss{}te nat\"{u}rliche Zahl
	\item Zwischen zwei nat\"{u}rlichen Zahlen gibt es keine weitere nat\"{u}rliche Zahl
\end{itemize}

\sub{Ganze Zahlen}

Die Menge der ganzen Zahlen $\mathbb{Z}$ schlie\ss{}t die Menge der nat\"{u}rlichen Zahlen $\mathbb{N}$ ein, erweitert sie aber auf negative ganze Zahlen. Somit ist $\mathbb{Z}$ bez\"{u}glich der \textbf{Addition}, \textbf{Mulitplikation} und nun auch der \textbf{Subtraktion}, jedoch noch nicht bez\"{u}glich der Division, abgeschlossen.

\begin{figure}[h!]
	\begin{tikzpicture}
		% N circle and node
		\draw (0, 0) circle [radius=0.4cm]
					 node {$\mathbb{N}$};

		% Z circle
		\draw (0, 0) circle [radius=1cm];

		% Z node
		\draw (0.5, 0.5) node  {$\mathbb{Z}$};
	\end{tikzpicture}
\end{figure}

Eigenschaften der ganzen Zahlen:

\begin{itemize}
	\item Jede ganze Zahl $z$ hat einen Vorg\"{a}nger $z - 1$ und einen Nachfolger $z + 1$
	\item Es gibt weder eine gr\"{o}\ss{}te noch eine kleinste ganze Zahl
	\item Zwischen zwei ganzen Zahlen gibt es keine weitere ganze Zahl
\end{itemize}

\pagebreak

\sub{Rationale Zahlen}

Die Menge der rationalen Zahlen $\mathbb{Q}$ erweitert die Menge der ganzen Zahlen $\mathbb{Z}$ auf jene \emph{endlichen} oder \emph{unendlichen, periodischen} Dezimahlzahlen, welche als Bruch $\frac{z}{n}$ mit $z, n \in \mathbb{Z}$ und $n \neq 0$ darstellbar sind: $$\mathbb{Q} = \{\, \frac{z}{n} \,|\, z,n \in \mathbb{Z}, n \neq 0\,\}$$ Die Menge $\mathbb{Q}$ ist bez\"{u}glich der \textbf{Addition}, \textbf{Subtraktion} und \textbf{Mulitplikation} abgeschlossen. Ohne null, also $\mathbb{Q}\backslash\{0\}$, ist sie auch bez\"{u}glich der \textbf{Division} abgeschlossen.

\begin{figure}[h!]
	\begin{tikzpicture}
		% N circle and node
		\draw (0, 0) circle [radius=0.4cm]
					 node {$\mathbb{N}$};

		% Z circle
		\draw (0, 0) circle [radius=1cm];

		% Z node
		\draw (0.5, 0.5) node  {$\mathbb{Z}$};


		% Q circle
		\draw (0, 0) circle [radius=1.5cm];

		% Q node
		\draw (0.9, 0.9) node  {$\mathbb{Q}$};
	\end{tikzpicture}
\end{figure}
Eigenschaften der rationalen Zahlen:
\begin{itemize}
	\item Zwischen zwei rationalen Zahlen l\"{a}sst sich stets eine weitere rationale Zahl einf\"{u}gen
	\item Daher hat eine rationale Zahl $r$ weder einen definitiven Vorg\"{a}nger noch Nachfolger
	\item Eine rationale Zahl $r$ l\"{a}sst sich als Bruch $\frac{z}{n}$ darstellen, wo gilt $z, n \in \mathbb{Z}$ und $n \neq 0$
	\item Auch unendliche Zahlen k\"{o}nnen rational sein, wenn sie sich als Bruch darstellen lassen: $\frac{1}{3} = 0.\dot{3}$
	\item Es gibt keine gr\"{o}\ss{}te oder kleinste rationale Zahl
\end{itemize}

\sub{Reelle Zahlen}

Die Menge der reellen Zahlen $\mathbb{R}$ erweitert jene der rationalen Zahlen $\mathbb{Q}$ auf unendliche, nicht-periodische Zahlen, welche sich nicht als Bruch darstellen lassen. Beispiele daf\"{u}r w\"{a}ren Konstanten wie $\pi, e$ oder $\sqrt{2}$. Formal gesehen vereinigt $\mathbb{R}$ die Menge der rationalen Zahlen $\mathbb{Q}$ mit der Menge der \emph{irrationalen} Zahlen $\mathbb{I}$. 

\begin{figure}[h!]
	\begin{tikzpicture}
		% N circle and node
		\draw (0, 0) circle [radius=0.4cm]
					 node {$\mathbb{N}$};

		% Z circle
		\draw (0, 0) circle [radius=1cm];

		% Z node
		\draw (0.5, 0.5) node  {$\mathbb{Z}$};


		% Q circle
		\draw (0, 0) circle [radius=1.5cm];

		% Q node
		\draw (0.9, 0.9) node  {$\mathbb{Q}$};


		% R circle
		\draw (0, 0) circle [radius=2cm];

		% R node
		\draw (1.3, 1.3) node  {$\mathbb{R}$};
	\end{tikzpicture}
\end{figure}
$$
\mathbb{R} =
\begin{cases}
	\text{rationale Zahlen } \mathbb{Q}
	\begin{cases}
		\text{ endliche Dezimalzahlen: } 1.2, \frac{3}{4}, 45, \sqrt{4}, \ldots
		\\
		\text{ unendliche, periodische Dezimalzahlen: } \frac{1}{3}, \frac{7}{9}, \ldots
	\end{cases}
	\\
	\text{irrationale Zahlen } \mathbb{I} \rightarrow
	\text{ unendliche, nicht-periodische Dezimalzahlen } \sqrt{5}, \pi, e, \ldots
\end{cases}
$$

\end{document}