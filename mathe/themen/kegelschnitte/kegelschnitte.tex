\documentclass[11pt]{article}

\usepackage[german]{babel}

\usepackage[autostyle=true]{csquotes}

\usepackage[a4paper, margin=1in]{geometry}

\usepackage{libertine}

\setlength{\parindent}{0pt}

\addtolength{\parskip}{\baselineskip}

\newcommand{\extrapar}{\par\vspace{\baselineskip}}

\newcommand{\heading}[1]{\begin{center}\Huge \textbf{#1} \end{center}}

\newcommand{\sub}[1]{{\Large \textbf{#1}}\par}

\newcommand{\subsub}[1]{{\large \textbf{#1}}\par}

\newcommand{\zitat}[1]{\emph{\foreignquote{german}{#1}}}

\newcommand{\titleitem}[1]{\item \textbf{#1} \par}

\begin{document}

\heading{Kegelschnitte}
\thispagestyle{plain}

Kegelschnitte sind geometrische Figuren, die entstehen, wenn man Kegel mit Ebenen auf verschiedene Weisen schneidet. Bei der Diskussion von Kegelschnitten sind vor allem die Konstruktion der Figuren sowie das Schneiden mit Geraden oder anderen Kegelschnitten interessant.  

\sub{Kreis}

Im Koordinatensystem ist ein Kreis durch einen Mittelpunkt $M$ und einen Radius $r$ exakt definiert: $$k:\,\,[\,M(x_{M}|y_{M}),\, r\,]$$

Jeder Punkt $X$ auf dem Kreis $k$ hat vom Mittelpunkt $M$ den Abstand $r$, sodass gilt: $\overbar{MX} = r$

Man kann mittels dem Satz des Pythagoras jeden Punkt $X$ auf dem Kreis $k$ berechnen, da die $x$- und $y$-Koordinaten des Punktes $X$ als Katheten, zusammen mit dem Radius $r$ als Hypotenuse, ein rechtwinkliges Dreieck im Kreis bilden:

\begin{figure}[h!]
	\centering
	\begin{tikzpicture}

		% circle
		\draw [red] 
			  (2.5, 2.5) circle [radius=2.5];

		% circle label
		\draw [red]
			  (-0.2, 1.3) node [] {$k$};

		\draw [fill=black] 
			  (2.5, 2.5) circle [radius=1pt]
			             node [left] {$M (x_{M} | y_{M})$}

		   -- (4.5, 2.5) circle [radius=1pt]
		   			     node [midway, below] {$x - x_{M}$}
		             
		   -- (4.5, 4.0) circle [radius=1pt]
		   			     node [pos=0.4, right] {$y\,- y_{M}$}
		   			     node [right] {$X (x | y)$}

		   -- (2.5, 2.5) node [midway, above] {$r$};

	\end{tikzpicture}
\end{figure}

Durch Entnahme der korrekten Variablen aus der Grafik und Einsetzen in den Pythagor\"{a}ischen Lehrsatz erh\"{a}lt man so die \textbf{Kreisgleichung in Koordinatenform}:

\begin{center}
	$a^2 + b^2 = c^2$\\
	$\Downarrow$\\
	$k:\, (x - x_{M})^2 + (y - y_{M})^2 = r^2$
\end{center}

Liegt der Mittelpunkt eines Kreises mit Radius $r = 1$ im Koordinatenursprung $(0, 0)$, erh\"{a}lt man die sehr kompakte und einpr\"{a}gsame Kreisgleichung des \emph{Einheitskreises}:

$$k_{E}:\,x^2 + y^2 = 1$$

Multipliziert man die Kreisgleichung aus, erh\"{a}lt man die \textbf{allgemeine Kreisgleichung}:

\begin{center}
	$k:\, (x - x_{M})^2 + (y - y_{M})^2 = r^2$\\
	$\Downarrow$\\
	$x^2 + y^2 + ax + by + c = 0$
\end{center}

\pagebreak

Um von der ausmultiplizierten, allgemeinen Kreisgleichung auf die Kreisgleichung in Koordinatenform zur\"{u}ckzukommen, muss man die allgmeine Kreisgleichung auf ein volles Quadrat erg\"{a}nzen. Beispiel: \par

\emph{Die Gleichung $x^2 + y^2 + 4x - 2y - 20 = 0$ beschreibt einen Kreis. Ermittle den Mittelpunkt $M$ und den Radius $r$ des Kreises.}

\begin{tabular}{l l}
	I. Allgemeine Kreisgleichung: & $x^2 + y^2 + 4x - 2y - 20 = 0$
	\\
	II. Umformung um die Erg\"{a}nzung zu erleichtern: & $(x^2 + 4x + a^2) + (y^2 - 2y + b^2) = 20$
	\\
	III. Finden der passenden Variablen: & $a = 2,\, b = -1$
	\\
	IV. Addition der Quadrate auf beiden Seiten der Gleichung: & $(x^2 + 4x + 4) + (y^2 - 2y + 1) = 20 + 4 + 1$
	\\
	V. Zu binomischen Formeln umformen: & $(x + 2)^2 + (y - 1)^2 = 25$
	\\
	VI. Vergleich mit der Kreisgleichung: & $(x - x_{M})^2 + (y - y_{M})^2 = r^2$
	\\
	VII. Entnahme des Mittelpunktes: & $M (-2 | 1)$
	\\
	VIII. Entnahme des Radius: & $r^2 = 25 \Rightarrow r = 5$
\end{tabular}

\subsub{Schnitt Kreis - Gerade}

Um eine Gerade $g$ mit einem Kreis $k$ zu schneiden dr\"{u}ckt man eine Variable ($x$ oder $y$) aus der Geradengleichung aus und setzt sie in die Kreisgleichung von $k$ ein. Dabei erh\"{a}lt man eine quadratische Gleichung nach der nicht ausgedr\"{u}ckten Variable, welche einem entweder zwei (A), einen (B) oder keinen (C) gemeinsame(n) Punkt(e) liefert. Hierbei muss man beachten, dass es drei m\"{o}gliche Lagebeziehungen zwischen der Gerade und dem Kreis geben kann. Die Gerade kann n\"{a}mlich sein:

\begin{enumerate}

\renewcommand{\theenumi}{\Alph{enumi}}

	\titleitem{Sekante}

	Die Gerade schneidet den Kreis in zwei Punkten und bildet so zwei Schnittpunkte.

	\titleitem{Tangente}

	Die Gerade ber\"{u}hrt den Kreis in einem Punkt --- dem \emph{Ber\"{u}hrpunkt}.

	\titleitem{Passante}

	Der Kreis wird von der Gerade weder geschnitten noch ber\"{u}hrt. 

\end{enumerate}

\subsub{Schnitt Kreis - Kreis}

Zwei Kreise $k_{1}$ und $k_{2}$ schneidet man nach \"{u}blicher Methode in einem Gleichungssystem. Hierbei gibt es auch wieder verschiedene Lagebeziehung zwischen den beiden Kreisen:
\begin{enumerate}

\renewcommand{\theenumi}{\Alph{enumi}}

	\titleitem{Ident}

	$k_{1}$ und $k_{2}$ sind gleich, ber\"{u}hren einander also in unendlich vielen Punkten: $k_{1} = k_{2}$. Diese Lagebeziehung gilt, wenn sowohl Mittelpunkt als auch Radius der beiden Kreise gleich sind. 

	\titleitem{Zwei Schnittpunkte}

	$k_{1}$ und $k_{2}$ \"{u}berlappen so, dass sie zwei Schnittpunkte haben.

	\titleitem{Ein Ber\"{u}hrpunkt}

	$k_{1}$ und $k_{2}$ ber\"{u}hren einander in genau einem einzigen Punkt.

	\titleitem{Keine gemeinsamen Punkte}

	$k_{1}$ und $k_{2}$ liegen entweder nebeneinander oder ineinander, schneiden oder ber\"{u}hren sich jedoch nicht. 

\end{enumerate}

Haben zwei Kreise einen Ber\"{u}hr- oder Schnittpunkt, kann man auch die Winkel zwischen den Tangenten der beiden Kreise in diesem Punkt berechnen. Die Richtungsvektoren $\vec{t_{1}}$ und $\vec{t_{2}}$ der Tangentengleichungen berechnet man als Normalvektoren der Vektoren von jeweils einem Mittelpunkt ($M_{1}$ , $M_{2}$) zu dem Ber\"{u}hr- oder Schnittpunkt $P$, also:

\begin{figure}[h!]
\centering
	\begin{tikzpicture}

		% k_{1}
		\draw [red] (2.5, 2.5) circle [radius=2.5];

		% M_{1}P
		\draw [fill=black] 
			  (2.5, 2.5) circle [radius=1pt]
			  			 node [below] {$M_{1}$}

		   -- (4.8, 3.52) circle [radius=1pt]
		   				 node [midway, below] {$\vec{M_{1}P}$}
		   				 node [pos=0.95, below] {P};

		% t_{1}
		\draw [<->]
			  (2.87, 8.0)
		   -- (6.35, 0.0) node [pos=0.13, right] {$t_{1} = \vec{n_{M_{1}P}}$ };


		% k_{2}
		\draw [blue] (5, 6) circle [radius=2.5];

		% M_{2}P
		\draw [fill=black] 
			  (5.0, 6.0) circle [radius=1pt]
			  			 node [right] {$M_{2}$}

		   -- (4.8, 3.52) node [midway, right] {$\vec{M_{2}P}$};

		% t_{2}
		\draw [<->]
			  (0.0, 3.52)
		   -- (8.0, 3.52) node [pos=0.25, above] {$t_{2} = \vec{n_{M_{2}P}}$};

		% alpha
		\draw (6, 3.52) arc [radius=1, start angle = 0, end angle = -76];

		\draw (5.4, 3.15) node [] {$\alpha$}; 

	\end{tikzpicture}
\end{figure}

\extrapar

\begin{table}[h!]
\centering
	\begin{tabular}{c c}
		$ \vec{M_{1}P}$ & $\vec{M_{2}P}$\\
		$ \Downarrow$ & $\Downarrow$\\
		$\vec{t_{1}} = \vec{n_{M_{1}P}}$ & $\vec{t_{2}} = \vec{n_{M_{2}P}}$
	\end{tabular}
\end{table}

\extrapar

Den Winkel $\alpha$ berechnet man dann mittels der Vektoriellen Winkelformel: $$\cos \alpha = \frac{\vec{t_{1}} \cdot{} \vec{t_{2}}}{|\vec{t_{1}}| \cdot{} |\vec{t_{2}}| }$$

\pagebreak

\sub{Ellipse}

Eine Ellipse kann als eine ovale Kurve oder als ein Kreis mit zwei verschiedenen Radii $a$ und $b$ f\"{u}r die x- bzw. y-Achse, anstatt nur einem Radius $r$, gesehen werden. Sie ist definiert druch zwei Brennpunkte $F_{1}$ und $F_{2}$ sowie einer Zahl $a > 0$. Eine vollst\"{a}ndig beschriftete Ellipse $ell$ mit einem Punkt $X$ sieht so aus:

\begin{figure}[h!]
\centering
	\begin{tikzpicture}

		% ellipsis
		\draw [red] 
		      (4, 4) circle [x radius = 3, y radius = 2];

		% ellipsis label
		\draw [red]
			  (6.5, 2.2) node [] {$ell$};

		% x axis
		\draw [fill=black] 
			  (0.0, 4.0)

		   -- (1.0, 4.0) circle [radius=1.2pt]
		   			     node [below left] {B}

		   -- (2.0, 4.0) circle [radius=1.2pt]
		   			     node [below left] {$F_{2}$}

		   -- (4.0, 4.0) circle [radius=1.2pt]
		   			     node [below left] {0}

		   -- (6.0, 4.0) circle [radius=1.2pt]
		    		     node [below right] {$F_{1}$}

		   -- (7.0, 4.0) circle [radius=1.2pt] 
		   			     node [below right] {A};

	    % arrow tip of x axis
	    \draw [->] (7.0, 4.0)
	            -- (8.0, 4.0) node [below, pos = 1] {x};

		% y axis
		\draw [fill=black]
		      (4.0, 1,0)
		   -- (4.0, 2.0) circle [radius=1.2pt]
		   			     node [above right] {D}

		   -- (4.0, 6.0) circle [radius=1.2pt] 
		   				 node [above right] {C};

		% arrow tip of y axis
	    \draw [->] (4.0, 6.0)
	            -- (4.0, 7.0) node [right, pos = 1] {y};

	    % X 
	    \draw [fill=black]
	    	  (2.0, 4.0)
	       -- (5.5, 5.73) circle [radius=1.2pt]
	       			      node [pos=1.07] {X}
	       -- (6.0, 4.0);

	    % a
	    \draw [<->]
	          (4.0, 6.6)
	       -- (1.0, 6.6) node [midway, above] {a};

	    % b
	    \draw [<->]
	    	  (0.5, 4.0)
	       -- (0.5, 6.0) node [midway, right] {b};

	    % e
	    \draw [<->]
 	          (4.0, 3.2)
	       -- (6.0, 3.2) node [midway, below] {e};


	\end{tikzpicture}
\end{figure}

Hierbei sind:

\begin{tabular}{l c l}
	$A, B$ & \ldots & Schnittpunkte mit der x-Achse $\Rightarrow$ Hauptscheitel\\
	$C, D$ & \ldots &  Schnittpunkte mit der y-Achse $\Rightarrow$ Nebenscheitel\\ 
	$a$ & \ldots & gro\ss{}e Halbachse\\
	$b$ & \ldots & kleine Halbachse \\ 
	$F_{1}, F_{2}$ & \ldots & Brennpunkte\\
	$e$ & \ldots & Lineare Exzentrizit\"{a}t\\
\end{tabular}

Es gibt folgende Zusammenh\"{a}nge zwischen diesen Variablen:

\begin{enumerate}
	\item $\overbar{XF_{1}} + \overbar{XF_{2}} = 2a$
	\item $a = |\vec{0A}| = |\vec{0B}| \Rightarrow A (a | 0),\, B (-a | 0)$
	\item $b = |\vec{0C}| = |\vec{0D}| \Rightarrow C (0 | b),\, D (0 | -b)$
	\item $e = |\vec{0F_{1}}| = |\vec{0F_{2}}| \Rightarrow F_{1} (e | 0)\, F_{2} (-e | 0)$
	\item $e^2 = a^2 - b^2$
\end{enumerate}

Hat eine Ellipse, wie jene oben, ihre Brennpunkte symmetrisch zum Koordinatenursprung auf der $x$-Achse ($F_{1} (-e | 0),\, F_{2} (e | 0)$), sowie die gro\ss{}e Halbachse auf der $x$-Achse liegend ($A (a | 0),\, B (-a | 0)$), so spricht man von einer Ellipse in \textbf{\emph{1. Hauptlage}}. Liegen die Brennpunkte auf der y-Achse, befindet sich die Ellipse in der \emph{2. Hauptlage}. 

\pagebreak

Eine Ellipse $ell$ in 1. Hauptlage mit den Halbachsen $a$ und $b$ kann geometrisch durch folgende Gleichung beschrieben werden:

\begin{table}[h!]
	\begin{center}
		$ell:\, b^2x^2 + a^2y^2 = a^2b^2$\\
		$\Downarrow$\\
		$ell:\, \frac{x^2}{a^2} + \frac{y^2}{b^2} = 1$
	\end{center}
\end{table}


\subsub{Tangenten an eine Ellipse}

Um eine Tangente $t$ an eine Ellipse $ell$ in einem Punkt $P$ zu legen, muss man die Ellipse implizit differenzieren und in das Resultat dessen den Punkt $P$ einsetzen, um die Steigung $k$ der Ellipse in diesem Punkt zu erhalten. Die $x$ und $y$ Koordinaten des Punktes $P$ sowie die Steigung $k$ der Ellipse in diesem Punkt setzt man dann in die Normalform der linearen Gleichung $y = kx + d$ einsetzen, um somit die volle Gleichung der Geraden bzw. der Tangente $t$ zu bestimmen. Beispiel:

\emph{Gegeben sind die Ellipse $ell: x^2 + 3y^2 = 28$ und der Punkt $P(4 | 2)$ der auf der Ellipse liegt. Bestimmte die Gleichung der Tangente $t$ an die Ellipse im Punkt P.}

{ \renewcommand{\arraystretch}{2}

\begin{tabular}{l l}
	I. Ellipsengleichung: & $x^2 + 3y^2 = 28$
	\\
	II Implizites Differenzieren: & $2x + 3 \cdot{} 2y \cdot{} y' = 0 \Rightarrow 2x + 6yy' = 0$
	\\
	III. Umformung nach $y'$: & $y' = \frac{-2x}{6y} = \frac{-x}{3y}$
	\\
	IV. Einsetzung von $P$: & $y' = \frac{-4}{3 \cdot{} 2} = \frac{-2}{3}$
	\\
	V. $y'$ ist die Steigung $k$ der Tangentengleichung $t$: & $y = kx + d$
	\\
	VI. Einsetzung von $k$ ($ = y'$) sowie $P$ in $t$: & $2 = 4 * \frac{-2}{3} + d$
	\\
	VII. L\"{o}sen nach $d$: & $d = 2 - \frac{-8}{3} = \frac{14}{3}$
	\\
	VIII. Fertige Tangentengleichung $t$: & $y = \frac{-2}{3}x + \frac{14}{3}$
\end{tabular}

} %

\pagebreak

\sub{Hyperbel}

Eine Hyperbel besteht aus zwei B\"{o}gen bzw. \zitat{\"{A}sten}, deren Form durch zwei Brennpunkte $F_{1}$ und $F_{2}$ sowie einer L\"{a}nge $a$ exakt definiert ist. Sind die Brennpunkte symmetrisch zum Koordinatenursprung und liegen auf der $x$-Achse ($F_{1} (-e | 0),\, F_{2} (e | 0)$), spricht man von einer Hyperbel in \textbf{1. Hauptlage}. Liegen die Brennpunkte auf der $y$-Achse, befindet sich die Hyperbel in \emph{2. Hauptlage}. Geometrisch bzw. grafisch sieht eine Hyperbel in 1. Hauptlage so aus:

\begin{figure}[h!]
\centering
	\begin{tikzpicture}

		% hyperbola, left branch
		\draw [red, domain=1.8:6.2, rotate=90]
			  plot (\x, {0.5 * \x*\x - 4 * \x + 5});

		% hyperbola, right branch
		\draw [red, domain=-6.2:-1.8, rotate=-90]
		      plot (\x, {0.5 * \x*\x + 4 * \x + 13});

	    % hyperbola label
		\draw [red]
			  (1.0, 2.5) node [] {$hyp$};

		% x axis
		\draw [fill=black] 
			  (0.0, 4.0)

		   -- (2.0, 4.0) circle [radius=1.2pt]
		   			     node [below left] {$F_{2}$}

		   -- (3.0, 4.0) circle [radius=1.2pt]
		   			     node [below left] {$B$}

		   -- (4.0, 4.0) circle [radius=1.2pt]
		   			     node [below left] {$0$}

		   -- (5.0, 4.0) circle [radius=1.2pt] 
		   			     node [below right] {$A$}

		   -- (6.0, 4.0) circle [radius=1.2pt]
		    		     node [below right] {$F_{1}$};

	    % arrow tip of x axis
	    \draw [->] 
	          (6.0, 4.0)

	       -- (8.0, 4.0) node [below, pos = 1] {x};

		% y axis
		\draw [fill=black, ->]
		      (4.0, 1,0)

		   -- (4.0, 3.0) circle [radius=1.2pt]
		   			     node [right] {$D$}

		   -- (4.0, 5.0) circle [radius=1.2pt]
		   				 node [right] {$C$};

		% arrow tip of y axis
		\draw [->]
			  (4.0, 5.0)

		   -- (4.0, 7.0) node [right, pos = 1] {$y$};

	    % X 
	    \draw [fill=black]
	    	  (2.0, 4.0)

	       -- (5.5, 5.0) circle [radius=1.2pt]
	       		     	 node [pos=1.05, right] {$X$}
	       -- (6.0, 4.0);

	    % a
	    \draw [<->]
	          (4.0, 4.2)

	       -- (5.0, 4.2) node [midway, above] {$a$};

	    % b
	    \draw [<->]
	    	  (3.5, 4.0)

	       -- (3.5, 5.0) node [pos=0.6, left] {$b$};

	    % e
	    \draw [<->]
 	          (4.0, 3.3)

	       -- (2.0, 3.3) node [midway, below right] {$e$};

	\end{tikzpicture}
\end{figure}

Hierbei sind:

\begin{tabular}{l c l}
	$A, B$ & \ldots & Schnittpunkte mit der x-Achse $\Rightarrow$ Hauptscheitel\\
	$C, D$ & \ldots &  Schnittpunkte mit der y-Achse $\Rightarrow$ Nebenscheitel\\ 
	$a$ & \ldots & gro\ss{}e Halbachse\\
	$b$ & \ldots & kleine Halbachse \\ 
	$F_{1}, F_{2}$ & \ldots & Brennpunkte\\
	$e$ & \ldots & Lineare Exzentrizit\"{a}t\\
\end{tabular}

Es gibt folgende Zusammenh\"{a}nge zwischen diesen Variablen:

\begin{enumerate}
	\item $|\overbar{XF_{1}} - \overbar{XF_{2}}| = 2a$
	\item $a = |\vec{0A}| = |\vec{0B}| \Rightarrow A (a | 0),\, B (-a | 0)$
	\item $b = |\vec{0C}| = |\vec{0D}| \Rightarrow C (0 | b),\, D (0 | -b)$
	\item $e = |\vec{0F_{1}}| = |\vec{0F_{2}}| \Rightarrow F_{1} (-e | 0)\, F_{2} (e | 0)$
	\item $e^2 = a^2 + b^2$
\end{enumerate}

Eine Hyperbel $hyp$ in 1. Hauptlage mit den Halbachsen $a$ und $b$ kann geometrisch durch folgende Gleichung beschrieben werden, welche sich von der Ellipsengleichung nur im Vorzeichen unterscheidet:

\begin{table}[h!]
	\begin{center}
		$hyp:\, b^2x^2 - a^2y^2 = a^2b^2$\\
		$\Downarrow$\\
		$hyp:\, \frac{x^2}{a^2} - \frac{y^2}{b^2} = 1$
	\end{center}
\end{table}

Die Asymptoten $as_{1, 2}$ einer Hyperbel sind jene Geraden, welche den beiden \"{A}sten unendlich nahe kommen ohne sie je wirklich zu ber\"{u}hren. Sie sind definiert als:

$$as_{1, 2}: y = \pm \frac{b}{a}\cdot{}x$$

\begin{figure}[h!]
\centering
	\begin{tikzpicture}

		% hyperbola, left branch
		\draw [red, domain=1.8:6.2, rotate=90] 
			  plot (\x, {0.5 * \x*\x - 4 * \x + 5});

		% hyperbola, right branch
		\draw [red, domain=-6.2:-1.8, rotate=-90]
			   plot (\x, {0.5 * \x*\x + 4 * \x + 13});

		% hyperbola label
		\draw [red]
			  (1.0, 2.5) node [] {$hyp$};

		% x axis
		\draw [fill=black] 
			  (0.0, 4.0)

		   -- (2.0, 4.0) circle [radius=1.2pt]
		   			     node [below left] {$F_{2}$}

		   -- (3.0, 4.0) circle [radius=1.2pt]
		   			     node [below left] {$B$}

		   -- (4.0, 4.0) circle [radius=1.2pt]
		   			     node [below left] {$0$}

		   -- (5.0, 4.0) circle [radius=1.2pt] 
		   			     node [below right] {$A$}

		   -- (6.0, 4.0) circle [radius=1.2pt]
		    		     node [below right] {$F_{1}$};

	    % arrow tip of x axis
	    \draw [->] 
	          (6.0, 4.0)

	       -- (8.0, 4.0) node [below, pos = 1] {x};

		% y axis
		\draw [fill=black, ->]
		      (4.0, 1,0)

		   -- (4.0, 3.0) circle [radius=1.2pt]
		   			     node [right] {$D$}

		   -- (4.0, 5.0) circle [radius=1.2pt]
		   				 node [right] {$C$};

		% arrow tip of y axis
		\draw [->]
			  (4.0, 5.0)

		   -- (4.0, 7.0) node [right, pos = 1] {$y$};


		% as_{1}
		\draw [blue, dashed]
			  (0, 1)
		   -- (8, 7) node [pos=0.8, above left] {$as_{1}$};

		% as_{2}
		\draw [blue, dashed]
			  (0, 7)
		   -- (8, 1) node [pos=0.8, below left] {$as_{2}$};
	\end{tikzpicture}
\end{figure}

\pagebreak

\sub{Parabel}

Eine Parabel ist eine Hyperbel mit nur einem Bogen bzw. mit nur einem Ast. Die Kurve einer Parabel ist exakt definiert durch einen Brennpunkt $F$ sowie eine Leitlinie $l$. Liegt der Brennpunkt $F$ auf der positiven $x$-Achse, der Scheitel der Parabel im Koordinatenursprung und ist die Leitlinie $l$ parallel zur $y$-Achse, handelt es sich um eine Parabel in \textbf{1. Hauptlage}. In diesem Fall ist die Parabel schon durch den Parameter $p$ exakt definiert.

\begin{figure}[h!]
\centering
	\begin{tikzpicture}

		% parabola
		\draw [red, domain=-6.2:-1.8, rotate=-90]
		       plot (\x, {0.5 * \x*\x + 4 * \x + 12});

		% parabola label
		\draw [red]
			   (5.5, 1.8) node [] {$par$};

		% x axis
		\draw [fill=black] 
			  (0.0, 4.0)

		   -- (4.0, 4.0) circle [radius=1.2pt]
		   			     node [below left] {$0$}

		   -- (5.0, 4.0) circle [radius=1.2pt]
		    		     node [below right] {$F$};

	    % arrow tip of x axis
	    \draw [->] 
	          (5.0, 4.0)

	       -- (8.0, 4.0) node [below, pos = 1] {x};

		% y axis
		\draw [fill=black, ->]
		      (4.0, 1,0)

		   -- (4.0, 7.0) node [right, pos = 1] {$y$};

	    % X 
	    \draw [fill=black]
	          (4.5, 5.0) circle [radius=1.2pt]
	       		     	 node [above] {$X$}

	       -- (5.0, 4.0);

	    % p
	    \draw [<->]
	          (3.0, 3.2)

	       -- (5.0, 3.2) node [midway, below left] {$p$};

	    % l
	    \draw (3.0, 1.0)

	       -- (3.0, 7.0) node [pos=0.6, left] {$l$};

	\end{tikzpicture}
\end{figure}

Hierbei sind:

\begin{tabular}{l c l}
	$l$ & \ldots & Leitlinie\\
	$F$ & \ldots & Brennpunkt\\
	$p$ & \ldots & Abstand zwischen Leitlinie und Brennpunkt\\ 
\end{tabular}

Es gibt folgende Zusammenh\"{a}nge zwischen diesen Variablen:

\begin{enumerate}
	\item $\overbar{XF} = \overbar{XI}$
	\item $p = \overbar{FI}$
	\item $F\,(\frac{p}{2} | 0)$
	\item $l(y) = -\frac{p}{2}$
\end{enumerate}

Die Gleichung einer Parabel $par$ in 1. Hauptlage mit dem Parameter $p$ lautet:

$$ y^2 = 2px$$

\end{document}
