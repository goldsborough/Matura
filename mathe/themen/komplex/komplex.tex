% Komplexe Zahlen

\documentclass[11pt]{article}

\usepackage[german]{babel}

\usepackage[autostyle=true]{csquotes}

\usepackage[a4paper, margin=1in]{geometry}

\usepackage{libertine}

\setlength{\parindent}{0pt}

\addtolength{\parskip}{\baselineskip}

\newcommand{\extrapar}{\par\vspace{\baselineskip}}

\newcommand{\heading}[1]{\begin{center}\Huge \textbf{#1} \end{center}}

\newcommand{\sub}[1]{{\Large \textbf{#1}}\par}

\newcommand{\subsub}[1]{{\large \textbf{#1}}\par}

\newcommand{\zitat}[1]{\emph{\foreignquote{german}{#1}}}

\newcommand{\titleitem}[1]{\item \textbf{#1} \par}

\begin{document}
% no header on first page
\thispagestyle{plain}

\heading{Komplexe Zahlen}

Die Menge der Komplexen Zahlen $\mathbb{C}$ ist nach den Mengen $\mathbb{N, Z, Q}$ und $\mathbb{R}$ die letzte uns erschlossene Zahlenmenge, mit welcher vorher unl\"{o}sbare Gleichungen und Probleme nun gel\"{o}st werden k\"{o}nnen.

\sub{Definitionen}

Die Menge der Komplexen Zahlen besch\"{a}ftigt sich mit der Verwendung der \textbf{imagin\"{a}ren Einheit} $i$, f\"{u}r welche gilt $i^2 = -1$. Dadurch k\"{o}nnen vorher nicht l\"{o}sbare Gleichungen gel\"{o}st werden:

$$x = 5 + \sqrt{-4} \Rightarrow \text{in }\mathbb{N, Z, Q, R} \text{ nicht l\"{o}sbar}$$

$$x = 5 + \sqrt{4} \cdot \sqrt{-1} = 5 + 2 \cdot i \Rightarrow \text{in } \mathbb{C} \text{ l\"{o}sbar}$$

Generell hat eine komplexe Zahl die Form $a + b \cdot i$, wobei $a$ als der \textbf{Realteil} und $b$ als der \textbf{Imagin\"{a}rteil} bezeichnet wird. Zahlen, die nur aus dem Imagin\"{a}rteil bestehen, also $b \cdot i$, werden \textbf{imagin\"{a}re Zahlen} genannt. Das negative Gegenst\"{u}ck zu einer komplexen Zahl $z = a + b \cdot i$ nennt man \textbf{konjugiert komplexe Zahl}: $\bar{z} = a - b \cdot i$

Hierbei ist es wichtig anzumerken, dass alle Zahlenmengen unter der Menge der komplexen Zahlen dennoch in $\mathbb{C}$ enthalten sind, da jede reelle Zahl $a$ als $a + 0 \cdot i$ angeschrieben werden kann.

Den \textbf{Betrag} $|z|$ einer komplexen Zahl $z = a + b \cdot i$ berechnet man gleich wie jenen eines Vektors: $|z| = \sqrt{a^2 + b^2}$. Der Grund daf\"{u}r ist, dass eine komplexe Zahl grafisch bzw. geometrisch \"{a}hnlich wie ein Vektor ein Koordinatentupel darstellt, wo $a$ die Koordinate auf der \textbf{reellen Zahlenachse} und $b$ die Koordinate auf der \textbf{imagin\"{a}ren Zahlenachse} ist:

\begin{figure}[h!]
	\centering
	\begin{tikzpicture}

		% real axis
		\draw [->] 
		      (0, 0) node [below left] {0}
		   -- (1, 0) node [below] {1}
		   -- (2, 0) node [below] {2}
		   -- (3, 0) node [below] {3}
		   -- (4, 0) node [below] {4}
		   		     node [pos=0.9, above] {reelle Achse};

		 % imaginary axis
		\draw [->]
			  (0, 0) 
		   -- (0, 1) node [left] {i}
		   -- (0, 2) node [left] {2i}
		   -- (0, 3) node [left] {3i}
		   -- (0, 4) node [left] {4i}
		   	         node [pos=0.9, right] {imagin\"{a}re Achse};

		 % z
		\draw [->, red]
		      (0, 0)
		   -- (2.94, 2.94) node [pos=0.6, above left] {$|z|$};

		\draw [fill=black]
		      (3, 3) circle [radius=1.2pt]
		      	     node [right] {$z$};

	\end{tikzpicture}
\end{figure}

\pagebreak

\sub{Potenzen von $i$}

Der Wert der imagin\"{a}ren Einheit $i$ hoch einer Potenz $n$ ver\"{a}ndert sich in Abh\"{a}ngigkeit von $n$:

$i^0 = 1$

$i^1 = i$

$i^2 = -1$

$i^3 = i^2 \cdot i = -1 \cdot i = -i$

$i^4 = i^2 \cdot i^2 = -1 \cdot (-1) = 1$

$i^5 = i^4 \cdot i = 1 \cdot i = i$

$i^6 = i^4 \cdot i^2 = 1 \cdot (-1) = -1$

$\dots$

F\"{u}r eine potenzierte imagin\"{a}re Einheit $i^n$ mit beliebigem $n$ kann man somit sagen:

$$
i^n = 
\begin{cases}
1 \text{, wenn } n \text{ mod } 4 = 0
\\ 
i \text{, wenn } n \text{ mod } 4 = 1
\\
-1 \text{, wenn } n \text{ mod } 4 = 2
\\
-i \text{, wenn } n \text{ mod } 4 = 3  
\end{cases}
$$

\sub{Grundrechnungsarten}

\subsub{Addition und Subtraktion}

Zwei komplexe Zahlen $z_{1} = a + b \cdot i$ und $z_{2} = c + d \cdot i$ werden addiert bzw. subtrahiert, in dem man die jeweilige Operation f\"{u}r die Real- und Imagin\"{a}rteile der beiden komplexen Zahlen, also f\"{u}r $a$ und $c$ bzw. $b$ und $d$ durchf\"{u}hrt: $$ z_{1} + z_{2} = (a + c) + (b + d) \cdot i$$ $$ z_{1} - z_{2} = (a - c) + (b - d) \cdot i$$

\subsub{Multiplikation}

Bei der Multiplizierung zweier komplexer Zahlen $z_{1} = a + b \cdot i$ und $z_{2} = c + d \cdot i$ werden die beiden Zahlen miteinander ausmultipliziert: $$z_{1} \cdot z_{2} = (a + b \cdot i) \cdot (c + d \cdot i)$$

Wobei man diese Multiplikation immer auf die folgende Formel reduzieren kann: $$z_{1} \cdot z_{2} = (ac - bd) + (ad + bc) \cdot i$$

\pagebreak

\subsub{Division}

Um zwei komplexe Zahlen $z_{1} = a + b \cdot i$ und $z_{2} = c + d \cdot i$ zu dividieren, muss man die Division im Nenner als auch im Z\"{a}hler um die konjugiert komplexe Zahl des Nenners, also $\bar{z_{2}}$, erweitern. Dadurch wird der Nenner wieder zu einer reellen Zahl: 

$$ 
\frac{z_{1}}{z_{2}} = \frac{z_{1} \cdot \bar{z_{2}}}{z_{2} \cdot \bar{z_{2}}} = \frac{(a + bi) \cdot (c - di)}{(c + di) \cdot (c - di)}
= \frac{(ac + bd) \cdot (bc - ad) \cdot i}{c^2 + d^2}
= \frac{ac + bd}{c^2 + d^2} + \frac{bc -ad}{c^2 + b^2} \cdot i
$$

\sub{Quadratische Gleichungen}

Quadratische Gleichungen der Form $x^2 + p \cdot x + q = 0$ sind in der Menge der komplexen Zahlen immer l\"{o}sbar, da nun auch negative Diskriminanten behandelt werden k\"{o}nnen. Beispiel:

\emph{L\"{o}se die quadratische Gleichung $x^2 + 4x + 13 = 0$ in $\mathbb{C}$}!

\begin{tabular}{l l}
	I. Wir verwenden die kleine L\"{o}sungsformel: & $x_{1,2} = \frac{-p}{2} \pm \sqrt{\frac{p^2}{4} - q}$
	\\ & \\
	II. Einsetzen der Gleichung: & $x_{1,2} = \frac{-4}{2} \pm \sqrt{\frac{16}{2} - 13}$
	\\ & \\
	III. Vereinfachung der Gleichung nach x: & $x_{1,2} = -2 \pm \sqrt{-9}$
	\\ & \\
	IV. Einsetzen der imagin\"{a}ren Einheit $i$: & $x_{1,2} = -2 \pm \sqrt{9} \cdot i$
	\\ & \\
	V. L\"{o}sung von $x_{1, 2}$: & $x_{1, 2} = -2 \pm 3i$
\end{tabular}

\extrapar

Eine quadratische Gleichung $x^2 + p \cdot x + q = 0$ mit p, q $\in \mathbb{R}$ besitzt somit in $\mathbb{C}$

\begin{itemize}
	\item zwei reelle L\"{o}sungen, wenn $D > 0$
	\item eine reelle L\"{o}sung, wenn $D = 0$
	\item zwei zueinander konjugiert komplexe L\"{o}sungen, wenn $D < 0$
\end{itemize}

wobei $D$ die Diskriminante unter der Wurzel der kleinen $$ x_{1, 2} = \frac{-p}{2} \pm \sqrt{D}$$ oder der gro\ss{}en $$ x_{1, 2} = \frac{-b \pm \sqrt{D}}{2a}$$ L\"{o}sungsformel ist.

\end{document}